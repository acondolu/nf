%!TEX root = main.tex

\TODO{DISCORSO SU:
\begin{itemize}
  \item fare esempi di combinazioni di operatori
  \item uno potrebbe definire il dominio come la free algebra degli operatori (tipo induttivo). Ma non si puo' fare, perche' il singoletto obbliga un dipendenza mutua della membership con l'uguaglianza, e definire l'uguaglianza semplicemente in modo extensionale rende la definizione non ben fondata/strutturalmente ricorsiva.
  \item allora bisogna ragionare sulle forme normali, e definire quanto puo' vicino a forme normali cosi' da rendere extensional equality as intensional as possible.
\end{itemize}}

\subsection{Sets}

\TODO{PRIMA COSA DA FARE DOMANI: togliere parametro implicito da set NF2}
\XXX

\begin{coqdoccode}
  \coqdocnoindent
  \coqdockw{Inductive} \coqdefRef{NF2.Model.SET}{SET}{\coqdocinductive{SET}} :=\coqdoceol
  \coqdocindent{1.00em}
  \ensuremath{|} \coqdef{NF2.Model.Low}{Low}{\coqdocconstructor{Low}} : \coqdocnotation{\ensuremath{\forall}} \coqdocvar{X}, \coqexternalref{::type scope:x '->' x}{http://coq.inria.fr/distrib/V8.11.0/stdlib//Coq.Init.Logic}{\coqdocnotation{(}}\coqdocvariable{X} \coqexternalref{::type scope:x '->' x}{http://coq.inria.fr/distrib/V8.11.0/stdlib//Coq.Init.Logic}{\coqdocnotation{\ensuremath{\rightarrow}}} \coqref{NF2.Model.SET}{\coqdocinductive{SET}}\coqexternalref{::type scope:x '->' x}{http://coq.inria.fr/distrib/V8.11.0/stdlib//Coq.Init.Logic}{\coqdocnotation{)}} \coqexternalref{::type scope:x '->' x}{http://coq.inria.fr/distrib/V8.11.0/stdlib//Coq.Init.Logic}{\coqdocnotation{\ensuremath{\rightarrow}}} \coqref{NF2.Model.SET}{\coqdocinductive{SET}}\coqdoceol
  \coqdocindent{1.00em}
  \ensuremath{|} \coqdef{NF2.Model.High}{High}{\coqdocconstructor{High}} : \coqdocnotation{\ensuremath{\forall}} \coqdocvar{X}, \coqexternalref{::type scope:x '->' x}{http://coq.inria.fr/distrib/V8.11.0/stdlib//Coq.Init.Logic}{\coqdocnotation{(}}\coqdocvariable{X} \coqexternalref{::type scope:x '->' x}{http://coq.inria.fr/distrib/V8.11.0/stdlib//Coq.Init.Logic}{\coqdocnotation{\ensuremath{\rightarrow}}} \coqref{NF2.Model.SET}{\coqdocinductive{SET}}\coqexternalref{::type scope:x '->' x}{http://coq.inria.fr/distrib/V8.11.0/stdlib//Coq.Init.Logic}{\coqdocnotation{)}} \coqexternalref{::type scope:x '->' x}{http://coq.inria.fr/distrib/V8.11.0/stdlib//Coq.Init.Logic}{\coqdocnotation{\ensuremath{\rightarrow}}} \coqref{NF2.Model.SET}{\coqdocinductive{SET}}
  .\coqdoceol
  \coqdocemptyline
\end{coqdoccode}

\begin{example}[Empty set and Universe]aaa

  \begin{coqdoccode}
    \coqdocnoindent
    \coqdockw{Definition} \coqdocdefinition{Ø} := \coqdocconstructor{Low} \coqexternalref{False}{http://coq.inria.fr/distrib/V8.11.0/stdlib//Coq.Init.Logic}{\coqdocinductive{False}} (\coqdockw{fun} \coqdocvar{x} \ensuremath{\Rightarrow} \coqdockw{match} \coqdocvariable{x} \coqdockw{with} \coqdockw{end}).\coqdoceol
    \coqdocemptyline
  \end{coqdoccode}

  In a fully specular way, the universal set is defined as a \emph{high} set with empty index:

  \begin{coqdoccode}
    \coqdocnoindent
    \coqdockw{Definition} \coqdocdefinition{U} := \coqdocconstructor{High} \coqexternalref{False}{http://coq.inria.fr/distrib/V8.11.0/stdlib//Coq.Init.Logic}{\coqdocinductive{False}} (\coqdockw{fun} \coqdocvar{x} \ensuremath{\Rightarrow} \coqdockw{match} \coqdocvariable{x} \coqdockw{with} \coqdockw{end}).\coqdoceol
    \coqdocemptyline
  \end{coqdoccode}
\end{example}

\begin{example}[Set union]

\end{example}

a

\begin{coqdoccode}
  \coqdocnoindent
\coqdockw{Fixpoint} \coqdefRef{NF2.Model.EQ}{EQ}{\coqdocdefinition{EQ}} \coqdocvar{a} \coqdocvar{b} := \coqdockw{match} \coqdocvariable{a},\coqdocvariable{b} \coqdockw{with}\coqdoceol
\coqdocindent{1.00em}
\ensuremath{|} \coqref{NF2.Model.Low}{\coqdocconstructor{Low}} \coqdocvar{\_} \coqdocvar{\_}, \coqref{NF2.Model.High}{\coqdocconstructor{High}} \coqdocvar{\_} \coqdocvar{\_} \ensuremath{\Rightarrow} \coqexternalref{False}{http://coq.inria.fr/distrib/V8.11.0/stdlib//Coq.Init.Logic}{\coqdocinductive{False}}\coqdoceol
\coqdocindent{1.00em}
\ensuremath{|} \coqref{NF2.Model.High}{\coqdocconstructor{High}} \coqdocvar{\_} \coqdocvar{\_}, \coqref{NF2.Model.Low}{\coqdocconstructor{Low}} \coqdocvar{\_} \coqdocvar{\_} \ensuremath{\Rightarrow} \coqexternalref{False}{http://coq.inria.fr/distrib/V8.11.0/stdlib//Coq.Init.Logic}{\coqdocinductive{False}}\coqdoceol
\coqdocindent{1.00em}
\ensuremath{|} \coqref{NF2.Model.Low}{\coqdocconstructor{Low}} \coqdocvar{\_} \coqdocvar{f}, \coqref{NF2.Model.Low}{\coqdocconstructor{Low}} \coqdocvar{\_} \coqdocvar{g} \ensuremath{\Rightarrow} \coqdocvar{f} $\AEQ$ \coqdocvar{g} \coqdoceol
\coqdocindent{1.00em}
\ensuremath{|} \coqref{NF2.Model.High}{\coqdocconstructor{High}} \coqdocvar{\_} \coqdocvar{f}, \coqref{NF2.Model.High}{\coqdocconstructor{High}} \coqdocvar{\_} \coqdocvar{g} \ensuremath{\Rightarrow} \coqdocvar{f} $\AEQ$ \coqdocvar{g} \coqdoceol
\coqdocnoindent
\coqdockw{end}.\coqdoceol

\end{coqdoccode}

\TODO{Dire che la notazione $\AEQ$ nasconde la dipendenza, ma che espandendo e' tutto ancora safe}