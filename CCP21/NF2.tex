%!TEX root = main.tex

\renewcommand\INX{\coqref{NF2.Model.IN}{{\IN}}~}
\renewcommand\EQX{\coqref{NF2.Model.EQ}{{\EQ}}~}

\TODO{DISCORSO SU:
\begin{itemize}
  \item fare esempi di combinazioni di operatori
  \item uno potrebbe definire il dominio come la free algebra degli operatori (tipo induttivo). Ma non si puo' fare, perche' il singoletto obbliga un dipendenza mutua della membership con l'uguaglianza, e definire l'uguaglianza semplicemente in modo extensionale rende la definizione non ben fondata/strutturalmente ricorsiva.
  \item allora bisogna ragionare sulle forme normali, e definire quanto puo' vicino a forme normali cosi' da rendere extensional equality as intensional as possible.
\end{itemize}}

\subsection{The Type \texorpdfstring{\coqdocinductive{SET}}{SET}}

\begin{coqdoccode}
  \coqdocnoindent
  \coqdockw{Inductive} \coqdefRef{NF2.Model.SET}{SET}{\coqdocinductive{SET}} :=\coqdoceol
  \coqdocindent{1.00em}
  \ensuremath{|} \coqdef{NF2.Model.Low}{Low}{\coqdocconstructor{Low}} : \coqdocnotation{\ensuremath{\forall}} \coqdocvar{X}, \coqexternalref{::type scope:x '->' x}{http://coq.inria.fr/distrib/V8.11.0/stdlib//Coq.Init.Logic}{\coqdocnotation{(}}\coqdocvariable{X} \coqexternalref{::type scope:x '->' x}{http://coq.inria.fr/distrib/V8.11.0/stdlib//Coq.Init.Logic}{\coqdocnotation{\ensuremath{\rightarrow}}} \coqref{NF2.Model.SET}{\coqdocinductive{SET}}\coqexternalref{::type scope:x '->' x}{http://coq.inria.fr/distrib/V8.11.0/stdlib//Coq.Init.Logic}{\coqdocnotation{)}} \coqexternalref{::type scope:x '->' x}{http://coq.inria.fr/distrib/V8.11.0/stdlib//Coq.Init.Logic}{\coqdocnotation{\ensuremath{\rightarrow}}} \coqref{NF2.Model.SET}{\coqdocinductive{SET}}\coqdoceol
  \coqdocindent{1.00em}
  \ensuremath{|} \coqdef{NF2.Model.High}{High}{\coqdocconstructor{High}} : \coqdocnotation{\ensuremath{\forall}} \coqdocvar{X}, \coqexternalref{::type scope:x '->' x}{http://coq.inria.fr/distrib/V8.11.0/stdlib//Coq.Init.Logic}{\coqdocnotation{(}}\coqdocvariable{X} \coqexternalref{::type scope:x '->' x}{http://coq.inria.fr/distrib/V8.11.0/stdlib//Coq.Init.Logic}{\coqdocnotation{\ensuremath{\rightarrow}}} \coqref{NF2.Model.SET}{\coqdocinductive{SET}}\coqexternalref{::type scope:x '->' x}{http://coq.inria.fr/distrib/V8.11.0/stdlib//Coq.Init.Logic}{\coqdocnotation{)}} \coqexternalref{::type scope:x '->' x}{http://coq.inria.fr/distrib/V8.11.0/stdlib//Coq.Init.Logic}{\coqdocnotation{\ensuremath{\rightarrow}}} \coqref{NF2.Model.SET}{\coqdocinductive{SET}}
  .\coqdoceol
  \coqdocemptyline
\end{coqdoccode}

\begin{example}[Empty set and Universe]aaa

  \begin{coqdoccode}
    \coqdocnoindent
    \coqdockw{Definition} \coqdocdefinition{Ø} := \coqdocconstructor{Low} \coqexternalref{False}{http://coq.inria.fr/distrib/V8.11.0/stdlib//Coq.Init.Logic}{\coqdocinductive{False}} (\coqdockw{fun} \coqdocvar{x} \ensuremath{\Rightarrow} \coqdockw{match} \coqdocvariable{x} \coqdockw{with} \coqdockw{end}).\coqdoceol
    \coqdocemptyline
  \end{coqdoccode}

  In a fully specular way, the universal set is defined as a \emph{high} set with empty index:

  \begin{coqdoccode}
    \coqdocnoindent
    \coqdockw{Definition} \coqdocdefinition{U} := \coqdocconstructor{High} \coqexternalref{False}{http://coq.inria.fr/distrib/V8.11.0/stdlib//Coq.Init.Logic}{\coqdocinductive{False}} (\coqdockw{fun} \coqdocvar{x} \ensuremath{\Rightarrow} \coqdockw{match} \coqdocvariable{x} \coqdockw{with} \coqdockw{end}).\coqdoceol
    \coqdocemptyline
  \end{coqdoccode}
\end{example}

\TODO{We'll see more sets in upcoming section}

\subsection{Set Equality}

a

\begin{coqdoccode}
  \coqdocnoindent
\coqdockw{Fixpoint} \coqdefRef{NF2.Model.EQ}{EQ}{\coqdocdefinition{EQ}} \coqdocvar{s} \coqdocvar{t} := \coqdockw{match} \coqdocvariable{s},\coqdocvariable{t} \coqdockw{with}\coqdoceol
\coqdocindent{1.00em}
\ensuremath{|} \coqref{NF2.Model.Low}{\coqdocconstructor{Low}} \coqdocvar{\_} \coqdocvar{\_}, \coqref{NF2.Model.High}{\coqdocconstructor{High}} \coqdocvar{\_} \coqdocvar{\_} \ensuremath{\Rightarrow} \coqexternalref{False}{http://coq.inria.fr/distrib/V8.11.0/stdlib//Coq.Init.Logic}{\coqdocinductive{False}}\coqdoceol
\coqdocindent{1.00em}
\ensuremath{|} \coqref{NF2.Model.High}{\coqdocconstructor{High}} \coqdocvar{\_} \coqdocvar{\_}, \coqref{NF2.Model.Low}{\coqdocconstructor{Low}} \coqdocvar{\_} \coqdocvar{\_} \ensuremath{\Rightarrow} \coqexternalref{False}{http://coq.inria.fr/distrib/V8.11.0/stdlib//Coq.Init.Logic}{\coqdocinductive{False}}\coqdoceol
\coqdocindent{1.00em}
\ensuremath{|} \coqref{NF2.Model.Low}{\coqdocconstructor{Low}} \coqdocvar{\_} \coqdocvar{f}, \coqref{NF2.Model.Low}{\coqdocconstructor{Low}} \coqdocvar{\_} \coqdocvar{g} \ensuremath{\Rightarrow} \coqdocvar{f} $\AEQ$ \coqdocvar{g} \coqdoceol
\coqdocindent{1.00em}
\ensuremath{|} \coqref{NF2.Model.High}{\coqdocconstructor{High}} \coqdocvar{\_} \coqdocvar{f}, \coqref{NF2.Model.High}{\coqdocconstructor{High}} \coqdocvar{\_} \coqdocvar{g} \ensuremath{\Rightarrow} \coqdocvar{f} $\AEQ$ \coqdocvar{g} \coqdoceol
\coqdocnoindent
\coqdockw{end}.\coqdoceol

\end{coqdoccode}

\TODO{Dire che la notazione $\AEQ$ nasconde la dipendenza, ma che espandendo e' tutto ancora safe}

\TODO{equivalence goes smoothly, like for \ZF{}, only the cases are doubled because of the two constructors.}

\begin{coqdoccode}
  \coqdocnoindent
  \coqdockw{Instance} \coqdefRef{NF2.Model.nf2 setoid}{nf2\_setoid}{\coqdocinstance{nf2\_setoid}} : \coqexternalref{Equivalence}{http://coq.inria.fr/distrib/V8.11.0/stdlib//Coq.Classes.RelationClasses}{\coqdocclass{Equivalence}} \coqref{NF2.Model.EQ}{\coqdocdefinition{EQ}}.
\end{coqdoccode}

\subsection{Set Membership}

\begin{coqdoccode}
  \coqdocnoindent
\coqdockw{Definition} \coqdefRef{NF2.Model.IN}{IN}{\coqdocdefinition{IN}} \coqdocvar{s} \coqdocvar{t} := \coqdockw{match} \coqdocvariable{t} \coqdockw{with}\coqdoceol
\coqdocindent{1.00em}
\ensuremath{|} \coqref{NF2.Model.Low}{\coqdocconstructor{Low}} \coqdocvar{\_} \coqdocvar{f} \ensuremath{\Rightarrow} \coqdocvariable{s} \coqref{Internal.Common.in aczel}{\AIN} \coqdocvariable{f} \coqdoceol
\coqdocindent{1.00em}
\ensuremath{|} \coqref{NF2.Model.High}{\coqdocconstructor{High}} \coqdocvar{\_} \coqdocvar{f} \ensuremath{\Rightarrow} \coqdocvariable{s} \coqref{Internal.Common.in aczel}{$\not\AIN$} \coqdocvariable{f} \coqdoceol
\coqdocnoindent
\coqdockw{end}.\coqdoceol
\end{coqdoccode}

\TODO{morfismo uguale a ZF}

\begin{coqdoccode}
  \coqdocnoindent
\coqdockw{Add} \coqdockw{Morphism} \coqref{NF2.Model.IN}{\coqdocdefinition{IN}} \coqdockw{with} \coqdockw{signature} \coqref{NF2.Model.EQ}{\coqdocdefinition{EQ}} \coqexternalref{ProperNotations.::signature scope:x '==>' x}{http://coq.inria.fr/distrib/V8.11.0/stdlib//Coq.Classes.Morphisms}{\coqdocnotation{==>}} \coqref{NF2.Model.EQ}{\coqdocdefinition{EQ}} \coqexternalref{ProperNotations.::signature scope:x '==>' x}{http://coq.inria.fr/distrib/V8.11.0/stdlib//Coq.Classes.Morphisms}{\coqdocnotation{==>}} \coqexternalref{iff}{http://coq.inria.fr/distrib/V8.11.0/stdlib//Coq.Init.Logic}{\coqdocdefinition{iff}} \coqdockw{as} \coqdocvar{IN\_mor}.\coqdoceol
\coqdocnoindent
\end{coqdoccode}

\subsection{Set Operators}

\begin{coqdoccode}
  \coqdockw{Definition} \coqdef{NF2.Sets.compl}{compl}{\coqdocdefinition{compl}} : \coqref{NF2.Model.SET}{\coqdocinductive{SET}} \coqexternalref{::type scope:x '->' x}{http://coq.inria.fr/distrib/V8.11.0/stdlib//Coq.Init.Logic}{\coqdocnotation{\ensuremath{\rightarrow}}} \coqref{NF2.Model.SET}{\coqdocinductive{SET}} :=\coqdoceol
\coqdocindent{1.00em}
\coqdockw{fun} \coqdocvar{s} \ensuremath{\Rightarrow} \coqdockw{match} \coqdocvariable{s} \coqdockw{with}\coqdoceol
\coqdocindent{1.00em}
\ensuremath{|} \coqref{NF2.Model.Low}{\coqdocconstructor{Low}} \coqdocvar{\_} \coqdocvar{f} \ensuremath{\Rightarrow} \coqref{NF2.Model.High}{\coqdocconstructor{High}} \coqdocvar{\_} \coqdocvar{f}\coqdoceol
\coqdocindent{1.00em}
\ensuremath{|} \coqref{NF2.Model.High}{\coqdocconstructor{High}} \coqdocvar{\_} \coqdocvar{f} \ensuremath{\Rightarrow} \coqref{NF2.Model.Low}{\coqdocconstructor{Low}} \coqdocvar{\_} \coqdocvar{f}\coqdoceol
\coqdocindent{1.00em}
\coqdockw{end}\coqdoceol
\coqdocnoindent
.\coqdoceol
\coqdocemptyline
\coqdocnoindent
\coqdockw{Lemma} \coqdef{NF2.Sets.compl ok}{compl\_ok}{\coqdoclemma{compl\_ok}} : \coqdockw{\ensuremath{\forall}} \coqdocvar{s} \coqdocvar{t}, \coqdocvariable{s} \coqref{NF2.Model.:::x 'xE2x88x88' x}{\coqdocnotation{∈}} \coqref{NF2.Sets.compl}{\coqdocdefinition{compl}} \coqdocvariable{t} \coqexternalref{::type scope:x '<->' x}{http://coq.inria.fr/distrib/V8.11.0/stdlib//Coq.Init.Logic}{\coqdocnotation{\ensuremath{\leftrightarrow}}} \coqexternalref{::type scope:x '<->' x}{http://coq.inria.fr/distrib/V8.11.0/stdlib//Coq.Init.Logic}{\coqdocnotation{(}}\coqdocvariable{s} \coqref{NF2.Model.:::x 'xE2x88x88' x}{\coqdocnotation{∈}} \coqdocvariable{t} \coqexternalref{::type scope:x '->' x}{http://coq.inria.fr/distrib/V8.11.0/stdlib//Coq.Init.Logic}{\coqdocnotation{\ensuremath{\rightarrow}}} \coqexternalref{False}{http://coq.inria.fr/distrib/V8.11.0/stdlib//Coq.Init.Logic}{\coqdocinductive{False}}\coqexternalref{::type scope:x '<->' x}{http://coq.inria.fr/distrib/V8.11.0/stdlib//Coq.Init.Logic}{\coqdocnotation{)}}.\coqdoceol
\coqdocnoindent
\coqdockw{Proof}.\coqdoceol
\coqdocindent{1.00em}
\coqdoctac{intros} \coqdocvar{s} \coqdocvar{t}. \coqdoctac{destruct} \coqdocvar{t}; \coqdoctac{simpl} \coqdocvar{compl}; \coqdoctac{simpl} \coqdocvar{IN}; \coqdoctac{split}; \coqdoctac{firstorder}.\coqdoceol
\coqdocindent{1.00em}
\coqdoctac{apply} \coqexternalref{not all not ex}{http://coq.inria.fr/distrib/V8.11.0/stdlib//Coq.Logic.Classical\_Pred\_Type}{\coqdoclemma{not\_all\_not\_ex}}. \coqdoctac{assumption}.\coqdoceol
\coqdocnoindent
\coqdockw{Qed}.\coqdoceol
\end{coqdoccode}

\begin{coqdoccode}
  \coqdocnoindent
\coqdockw{Definition} \coqdef{NF2.Sets.sing}{sing}{\coqdocdefinition{sing}} : \coqref{NF2.Model.SET}{\coqdocinductive{SET}} \coqexternalref{::type scope:x '->' x}{http://coq.inria.fr/distrib/V8.11.0/stdlib//Coq.Init.Logic}{\coqdocnotation{\ensuremath{\rightarrow}}} \coqref{NF2.Model.SET}{\coqdocinductive{SET}} :=\coqdoceol
\coqdocindent{1.00em}
\coqdockw{fun} \coqdocvar{s} \ensuremath{\Rightarrow} \coqref{NF2.Model.Low}{\coqdocconstructor{Low}} \coqexternalref{unit}{http://coq.inria.fr/distrib/V8.11.0/stdlib//Coq.Init.Datatypes}{\coqdocinductive{unit}} (\coqdockw{fun} \coqdocvar{\_} \ensuremath{\Rightarrow} \coqdocvariable{s}).\coqdoceol
\coqdocemptyline
\coqdocnoindent
\coqdockw{Definition} \coqdef{NF2.Sets.sing ok}{sing\_ok}{\coqdocdefinition{sing\_ok}} : \coqdockw{\ensuremath{\forall}} \coqdocvar{s} \coqdocvar{t}, \coqdocvariable{s} \coqref{NF2.Model.:::x 'xE2x88x88' x}{\coqdocnotation{∈}} \coqref{NF2.Sets.sing}{\coqdocdefinition{sing}} \coqdocvariable{t} \coqexternalref{::type scope:x '<->' x}{http://coq.inria.fr/distrib/V8.11.0/stdlib//Coq.Init.Logic}{\coqdocnotation{\ensuremath{\leftrightarrow}}} \coqdocvariable{t} \coqref{NF2.Model.:::x 'xE2x89xA1' x}{\coqdocnotation{≡}} \coqdocvariable{s}.\coqdoceol
\coqdocnoindent
\coqdockw{Proof}.\coqdoceol
\coqdocindent{1.00em}
\coqdoctac{intros}. \coqdoctac{simpl}. \coqdoctac{split}; \coqdoctac{intros}. \coqdoctac{destruct} \coqdocvar{H}; \coqdoctac{firstorder}. \coqdoctac{apply} \coqref{Internal.Misc.ex unit}{\coqdoclemma{ex\_unit}}. \coqdoctac{auto}.\coqdoceol
\coqdocnoindent
\coqdockw{Qed}.\coqdoceol
\end{coqdoccode}

\TODO{complement, union, singleton}

\begin{coqdoccode}
  \coqdockw{Definition} \coqdef{NF2.Sets.cup}{cup}{\coqdocdefinition{cup}} \coqdocvar{s} \coqdocvar{s'} := \coqdockw{match} \coqdocvariable{s}, \coqdocvariable{s'} \coqdockw{with}\coqdoceol
\coqdocindent{1.00em}
\ensuremath{|} \coqref{NF2.Model.Low}{\coqdocconstructor{Low}} \coqdocvar{\_} \coqdocvar{f}, \coqref{NF2.Model.High}{\coqdocconstructor{High}} \coqdocvar{\_} \coqdocvar{g} \ensuremath{\Rightarrow} \coqref{NF2.Model.High}{\coqdocconstructor{High}} \coqdocvar{\_} (\coqref{NF2.Sets.minus}{\coqdocdefinition{minus}} \coqdocvar{g} \coqdocvar{f})\coqdoceol
\coqdocindent{1.00em}
\ensuremath{|} \coqref{NF2.Model.High}{\coqdocconstructor{High}} \coqdocvar{\_} \coqdocvar{f}, \coqref{NF2.Model.Low}{\coqdocconstructor{Low}} \coqdocvar{\_} \coqdocvar{g} \ensuremath{\Rightarrow} \coqref{NF2.Model.High}{\coqdocconstructor{High}} \coqdocvar{\_} (\coqref{NF2.Sets.minus}{\coqdocdefinition{minus}} \coqdocvar{f} \coqdocvar{g})\coqdoceol
\coqdocindent{1.00em}
\ensuremath{|} \coqref{NF2.Model.Low}{\coqdocconstructor{Low}} \coqdocvar{\_} \coqdocvar{f}, \coqref{NF2.Model.Low}{\coqdocconstructor{Low}} \coqdocvar{\_} \coqdocvar{g} \ensuremath{\Rightarrow} \coqref{NF2.Model.Low}{\coqdocconstructor{Low}} \coqdocvar{\_} (\coqref{NF2.Sets.join}{\coqdocdefinition{join}} \coqdocvar{f} \coqdocvar{g})\coqdoceol
\coqdocindent{1.00em}
\ensuremath{|} \coqref{NF2.Model.High}{\coqdocconstructor{High}} \coqdocvar{\_} \coqdocvar{f}, \coqref{NF2.Model.High}{\coqdocconstructor{High}} \coqdocvar{\_} \coqdocvar{g} \ensuremath{\Rightarrow} \coqref{NF2.Model.Low}{\coqdocconstructor{Low}} \coqdocvar{\_} (\coqref{NF2.Sets.meet}{\coqdocdefinition{meet}} \coqdocvar{f} \coqdocvar{g})\coqdoceol
\coqdocnoindent
\coqdockw{end}.\coqdoceol
\coqdocnoindent
\coqdockw{Notation} \coqdef{NF2.Sets.:::x 'xE2x88xAA' x}{"}{"}A ∪ B" := (\coqref{NF2.Sets.cup}{\coqdocdefinition{cup}} \coqdocvar{A} \coqdocvar{B}) (\coqdoctac{at} \coqdockw{level} 85).\coqdoceol
\coqdocemptyline
\end{coqdoccode}

\TODO{define minus, meet, join??? For instace, minus is:}

\begin{coqdoccode}
  \coqdocnoindent
  \coqdockw{Definition} \coqdef{NF2.Sets.minus}{minus}{\coqdocdefinition{minus}} \{\coqdocvar{X} \coqdocvar{Y}\} \coqdocvar{f} \coqdocvar{g} \coqdoceol
  \coqdocindent{1.00em}
  : \coqexternalref{::type scope:'x7B' x ':' x 'x26' x 'x7D'}{http://coq.inria.fr/distrib/V8.11.0/stdlib//Coq.Init.Specif}{\coqdocnotation{\{}} \coqdocvar{x} \coqexternalref{::type scope:'x7B' x ':' x 'x26' x 'x7D'}{http://coq.inria.fr/distrib/V8.11.0/stdlib//Coq.Init.Specif}{\coqdocnotation{:}} \coqdocvariable{X} \coqexternalref{::type scope:'x7B' x ':' x 'x26' x 'x7D'}{http://coq.inria.fr/distrib/V8.11.0/stdlib//Coq.Init.Specif}{\coqdocnotation{\&}} \coqdockw{\ensuremath{\forall}} \coqdocvar{y} : \coqdocvariable{Y}, \coqexternalref{::type scope:'x7E' x}{http://coq.inria.fr/distrib/V8.11.0/stdlib//Coq.Init.Logic}{\coqdocnotation{\ensuremath{\lnot}}} \coqexternalref{::type scope:'x7E' x}{http://coq.inria.fr/distrib/V8.11.0/stdlib//Coq.Init.Logic}{\coqdocnotation{(}}\coqdocvariable{g} \coqdocvariable{y} \coqref{NF2.Model.:::x 'xE2x89xA1' x}{\coqdocnotation{≡}} \coqdocvariable{f} \coqdocvariable{x}\coqexternalref{::type scope:'x7E' x}{http://coq.inria.fr/distrib/V8.11.0/stdlib//Coq.Init.Logic}{\coqdocnotation{)}} \coqexternalref{::type scope:'x7B' x ':' x 'x26' x 'x7D'}{http://coq.inria.fr/distrib/V8.11.0/stdlib//Coq.Init.Specif}{\coqdocnotation{\}}} \coqexternalref{::type scope:x '->' x}{http://coq.inria.fr/distrib/V8.11.0/stdlib//Coq.Init.Logic}{\coqdocnotation{\ensuremath{\rightarrow}}} \coqref{NF2.Model.SET}{\coqdocinductive{SET}} \coqdoceol
\coqdocindent{1.00em}:=
\coqref{Internal.Misc.select}{\coqdocdefinition{select}} \coqdocvariable{f} (\coqdockw{fun} \coqdocvar{x} \ensuremath{\Rightarrow} \coqdockw{\ensuremath{\forall}} \coqdocvar{y}, \coqexternalref{::type scope:'x7E' x}{http://coq.inria.fr/distrib/V8.11.0/stdlib//Coq.Init.Logic}{\coqdocnotation{\ensuremath{\lnot}}} \coqexternalref{::type scope:'x7E' x}{http://coq.inria.fr/distrib/V8.11.0/stdlib//Coq.Init.Logic}{\coqdocnotation{(}}\coqdocvariable{g} \coqdocvariable{y} \coqref{NF2.Model.:::x 'xE2x89xA1' x}{\coqdocnotation{≡}} \coqdocvariable{f} \coqdocvariable{x}\coqexternalref{::type scope:'x7E' x}{http://coq.inria.fr/distrib/V8.11.0/stdlib//Coq.Init.Logic}{\coqdocnotation{)}}).\coqdoceol
\coqdocemptyline
\end{coqdoccode}

where selects:
'select f P' restricts the domain of a function f according to a predicate P:

\begin{coqdoccode}
  \coqdocnoindent
  \coqdockw{Definition} \coqdef{Internal.Misc.select}{select}{\coqdocdefinition{select}} \{\coqdocvar{X} \coqdocvar{Y}\} (\coqdocvar{f}: \coqdocvariable{X} \coqexternalref{::type scope:x '->' x}{http://coq.inria.fr/distrib/V8.11.0/stdlib//Coq.Init.Logic}{\coqdocnotation{\ensuremath{\rightarrow}}} \coqdocvariable{Y}) (\coqdocvar{P}: \coqdocvariable{X} \coqexternalref{::type scope:x '->' x}{http://coq.inria.fr/distrib/V8.11.0/stdlib//Coq.Init.Logic}{\coqdocnotation{\ensuremath{\rightarrow}}} \coqdockw{Prop}) \coqdoceol
  \coqdocindent{1.00em} : \coqexternalref{::type scope:'x7B' x ':' x 'x26' x 'x7D'}{http://coq.inria.fr/distrib/V8.11.0/stdlib//Coq.Init.Specif}{\coqdocnotation{\{}}\coqdocvar{x}\coqexternalref{::type scope:'x7B' x ':' x 'x26' x 'x7D'}{http://coq.inria.fr/distrib/V8.11.0/stdlib//Coq.Init.Specif}{\coqdocnotation{:}} \coqdocvariable{X} \coqexternalref{::type scope:'x7B' x ':' x 'x26' x 'x7D'}{http://coq.inria.fr/distrib/V8.11.0/stdlib//Coq.Init.Specif}{\coqdocnotation{\&}} \coqdocvariable{P} \coqdocvariable{x}\coqexternalref{::type scope:'x7B' x ':' x 'x26' x 'x7D'}{http://coq.inria.fr/distrib/V8.11.0/stdlib//Coq.Init.Specif}{\coqdocnotation{\}}} \coqexternalref{::type scope:x '->' x}{http://coq.inria.fr/distrib/V8.11.0/stdlib//Coq.Init.Logic}{\coqdocnotation{\ensuremath{\rightarrow}}} \coqdocvariable{Y}\coqdoceol
  \coqdocindent{1.00em}
  := \coqdockw{fun} \coqdocvar{x} \ensuremath{\Rightarrow} \coqdocvariable{f} (\coqexternalref{projT1}{http://coq.inria.fr/distrib/V8.11.0/stdlib//Coq.Init.Specif}{\coqdocdefinition{projT1}} \coqdocvariable{x}).\coqdoceol
  \coqdocemptyline
\end{coqdoccode}

\begin{coqdoccode}
\coqdocnoindent
\coqdockw{Lemma} \coqdef{NF2.Sets.cup ok}{cup\_ok}{\coqdoclemma{cup\_ok}} : \coqdockw{\ensuremath{\forall}} \coqdocvar{s} \coqdocvar{s'} \coqdocvar{t}, \coqdocvariable{t} \coqref{NF2.Model.:::x 'xE2x88x88' x}{\coqdocnotation{∈}} \coqref{NF2.Model.:::x 'xE2x88x88' x}{\coqdocnotation{(}}\coqdocvariable{s} \coqref{NF2.Sets.:::x 'xE2x88xAA' x}{\coqdocnotation{∪}} \coqdocvariable{s'}\coqref{NF2.Model.:::x 'xE2x88x88' x}{\coqdocnotation{)}} \coqexternalref{::type scope:x '<->' x}{http://coq.inria.fr/distrib/V8.11.0/stdlib//Coq.Init.Logic}{\coqdocnotation{\ensuremath{\leftrightarrow}}} \coqexternalref{::type scope:x 'x5C/' x}{http://coq.inria.fr/distrib/V8.11.0/stdlib//Coq.Init.Logic}{\coqdocnotation{(}}\coqdocvariable{t} \coqref{NF2.Model.:::x 'xE2x88x88' x}{\coqdocnotation{∈}} \coqdocvariable{s}\coqexternalref{::type scope:x 'x5C/' x}{http://coq.inria.fr/distrib/V8.11.0/stdlib//Coq.Init.Logic}{\coqdocnotation{)}} \coqexternalref{::type scope:x 'x5C/' x}{http://coq.inria.fr/distrib/V8.11.0/stdlib//Coq.Init.Logic}{\coqdocnotation{\ensuremath{\lor}}} \coqexternalref{::type scope:x 'x5C/' x}{http://coq.inria.fr/distrib/V8.11.0/stdlib//Coq.Init.Logic}{\coqdocnotation{(}}\coqdocvariable{t} \coqref{NF2.Model.:::x 'xE2x88x88' x}{\coqdocnotation{∈}} \coqdocvariable{s'}\coqexternalref{::type scope:x 'x5C/' x}{http://coq.inria.fr/distrib/V8.11.0/stdlib//Coq.Init.Logic}{\coqdocnotation{)}}.\coqdoceol
\end{coqdoccode}

\subsection{Extensionality}

\begin{itemize}
  \item One direction is is easy (?) (follows from \coqdocdefinition{in\_sound\_right})
  \item The other, we proceed by cases over $a, b$. When their constructor matches, the proof proceeds in like for ZF (the case high - high is completely specular). The cases high vs low are impossible, and we need to show that it is never the case that a low set and a high set cannot have the same extension.
\end{itemize}
aaa

Informally, if $p \approx \Complement{q}$ for booleans $p,q$, then $p \lor q \approx \top$.

\begin{coqdoccode}
  \coqdocnoindent
  \coqdockw{Lemma} \coqdefRef{NF2.Ext.pos univ}{pos\_univ}{\coqdoclemma{pos\_univ}}: \coqdockw{\ensuremath{\forall}} \coqdocvar{X} \coqdocvar{f} \coqdocvar{Y} \coqdocvar{g},\coqdoceol
\coqdocindent{1.00em}
\coqexternalref{::type scope:x '->' x}{http://coq.inria.fr/distrib/V8.11.0/stdlib//Coq.Init.Logic}{\coqdocnotation{(}}\coqdockw{\ensuremath{\forall}} \coqdocvar{s}, \coqdocvariable{s} \INX \coqref{NF2.Model.Low}{\coqdocconstructor{Low}} \coqdocvariable{X} \coqdocvariable{f} \coqexternalref{::type scope:x '<->' x}{http://coq.inria.fr/distrib/V8.11.0/stdlib//Coq.Init.Logic}{\coqdocnotation{\ensuremath{\leftrightarrow}}} \coqdocvariable{s} \INX \coqref{NF2.Model.High}{\coqdocconstructor{High}} \coqdocvariable{Y} \coqdocvariable{g}\coqexternalref{::type scope:x '->' x}{http://coq.inria.fr/distrib/V8.11.0/stdlib//Coq.Init.Logic}{\coqdocnotation{)}}\coqdoceol
\coqdocindent{1.00em}
\coqexternalref{::type scope:x '->' x}{http://coq.inria.fr/distrib/V8.11.0/stdlib//Coq.Init.Logic}{\coqdocnotation{\ensuremath{\rightarrow}}} \coqdockw{\ensuremath{\forall}} \coqdocvar{s}, \coqdocvariable{s} \INX \coqref{NF2.Model.Low}{\coqdocconstructor{Low}} (\coqdocvariable{X} \coqexternalref{::type scope:x '+' x}{http://coq.inria.fr/distrib/V8.11.0/stdlib//Coq.Init.Datatypes}{\coqdocnotation{+}} \coqdocvariable{Y}) (\coqdocvariable{f} \coqref{Internal.Misc.:::x 'xE2xA8x81' x}{\coqdocnotation{⨁}} \coqdocvariable{g}).\coqdoceol
\end{coqdoccode}
\begin{proof}
  Warning! Classical logic is required to prove this lemma.
  Assume $\coqdocvariable{s} \colon \coqdocinductive{SET}$. To show that \coqdocvariable{s} \INX \coqref{NF2.Model.Low}{\coqdocconstructor{Low}} (\coqdocvariable{X} \coqexternalref{::type scope:x '+' x}{http://coq.inria.fr/distrib/V8.11.0/stdlib//Coq.Init.Datatypes}{\coqdocnotation{+}} \coqdocvariable{Y}) (\coqdocvariable{f} \coqref{Internal.Misc.:::x 'xE2xA8x81' x}{\coqdocnotation{⨁}} \coqdocvariable{g}), we need to supply an element of type \coqdocvariable{X} \coqexternalref{::type scope:x '+' x}{http://coq.inria.fr/distrib/V8.11.0/stdlib//Coq.Init.Datatypes}{\coqdocnotation{+}} \coqdocvariable{Y}. Choosing the left or right injection into \coqdocvariable{X} \coqexternalref{::type scope:x '+' x}{http://coq.inria.fr/distrib/V8.11.0/stdlib//Coq.Init.Datatypes}{\coqdocnotation{+}} \coqdocvariable{Y} is equivalent to deciding whether \coqdocvariable{s} \INX \coqref{NF2.Model.Low}{\coqdocconstructor{Low}} \coqdocvariable{X} \coqdocvariable{f} holds.
\end{proof}

\TODO{The axiom of regularity is a \dots of ZF that rules out \dots Regularity does not hold in \NFTWO{}, since set membership is not well-founded (for instance, $\Universe \in \Universe \in \ldots$)}. 

\begin{coqdoccode}
  \coqdocnoindent
\coqdockw{Theorem} \coqdefRef{NF2.Ext.weak regularity}{weak\_regularity}{\coqdoclemma{weak\_regularity}}: \coqdockw{\ensuremath{\forall}} \coqdocvar{s}, \coqref{NF2.Model.low}{\coqdocdefinition{low}} \coqdocvariable{s} \coqexternalref{::type scope:x '->' x}{http://coq.inria.fr/distrib/V8.11.0/stdlib//Coq.Init.Logic}{\coqdocnotation{\ensuremath{\rightarrow}}} \coqdocvariable{s} \INX \coqdocvariable{s} \coqexternalref{::type scope:x '->' x}{http://coq.inria.fr/distrib/V8.11.0/stdlib//Coq.Init.Logic}{\coqdocnotation{\ensuremath{\rightarrow}}} \coqexternalref{False}{http://coq.inria.fr/distrib/V8.11.0/stdlib//Coq.Init.Logic}{\coqdocinductive{False}}.\coqdoceol
\end{coqdoccode}
\begin{proof}
  By structural induction on \coqdocvariable{s}. Let \coqdocvariable{s} = \coqref{NF2.Model.Low}{\coqdocconstructor{Low}} \coqdocvariable{X} \coqdocvariable{f}. From \coqdocvariable{s} \INX \coqdocvariable{s} it follows that there exists \coqdocvariable{x} such that \coqdocvariable{s} \EQX \coqdocvariable{f}~\coqdocvariable{x}. Note that \coqdocvariable{f}~\coqdocvariable{x} must be a low set by the definition of \coqdocdefinition{EQ}. Since \coqdocdefinition{IN} is a \coqdocdefinition{EQ}-morphism, it follows that \coqdocvariable{f}~\coqdocvariable{x} \INX \coqdocvariable{f}~\coqdocvariable{x}. We conclude by the inductive hypothesis.
\end{proof}

As an easy corollary, we prove that a low set and a high set cannot have the same extension:

\begin{coqdoccode}
  \coqdocnoindent
\coqdockw{Lemma} \coqdefRef{NF2.Ext.pos neg ext neq}{pos\_neg\_ext\_neq}{\coqdoclemma{pos\_neg\_ext\_neq}}: \coqdockw{\ensuremath{\forall}} \coqdocvar{X} \coqdocvar{f} \coqdocvar{Y} \coqdocvar{g},\coqdoceol
\coqdocindent{1.00em}
\coqexternalref{::type scope:x '->' x}{http://coq.inria.fr/distrib/V8.11.0/stdlib//Coq.Init.Logic}{\coqdocnotation{(}}\coqdockw{\ensuremath{\forall}} \coqdocvar{s}, \coqdocvariable{s} \INX \coqref{NF2.Model.Low}{\coqdocconstructor{Low}} \coqdocvariable{X} \coqdocvariable{f} \coqexternalref{::type scope:x '<->' x}{http://coq.inria.fr/distrib/V8.11.0/stdlib//Coq.Init.Logic}{\coqdocnotation{\ensuremath{\leftrightarrow}}} \coqdocvariable{s} \INX \coqref{NF2.Model.High}{\coqdocconstructor{High}} \coqdocvariable{Y} \coqdocvariable{g} \coqexternalref{::type scope:x '->' x}{http://coq.inria.fr/distrib/V8.11.0/stdlib//Coq.Init.Logic}{\coqdocnotation{)}} \coqexternalref{::type scope:x '->' x}{http://coq.inria.fr/distrib/V8.11.0/stdlib//Coq.Init.Logic}{\coqdocnotation{\ensuremath{\rightarrow}}} \coqexternalref{False}{http://coq.inria.fr/distrib/V8.11.0/stdlib//Coq.Init.Logic}{\coqdocinductive{False}}.\coqdoceol
\end{coqdoccode}
\begin{proof}
  By \coqref{NF2.Ext.pos univ}{\coqdoclemma{pos\_univ}} and \coqref{NF2.Ext.weak regularity}{\coqdoclemma{weak\_regularity}}.
\end{proof}

We are now ready to prove extensionality for \NFTWO:

\begin{coqdoccode}
  \coqdocnoindent
\coqdockw{Theorem} \coqdefRef{NF2.Ext.ext}{ext}{\coqdoclemma{ext}}: \coqdockw{\ensuremath{\forall}} \coqdocvar{s} \coqdocvar{s'}, \coqdocvariable{s} \EQX \coqdocvariable{s'} \coqexternalref{::type scope:x '<->' x}{http://coq.inria.fr/distrib/V8.11.0/stdlib//Coq.Init.Logic}{\coqdocnotation{\ensuremath{\leftrightarrow}}} \coqdockw{\ensuremath{\forall}} \coqdocvar{t}, \coqdocvariable{t} \INX \coqdocvariable{s} \coqexternalref{::type scope:x '<->' x}{http://coq.inria.fr/distrib/V8.11.0/stdlib//Coq.Init.Logic}{\coqdocnotation{\ensuremath{\leftrightarrow}}} \coqdocvariable{t} \INX \coqdocvariable{s'}.\coqdoceol
\end{coqdoccode}
\begin{proof}
  The direction $(\Rightarrow)$ of the double implication follows from \coqdocdefinition{IN} being an \coqdocdefinition{EQ}-morphism. As for the other direction, we proceed by cases on \coqdocvariable{s} and \coqdocvariable{s'}:
  \begin{itemize}
    \item If \coqdocvariable{s} and \coqdocvariable{s'} are both low sets or both high sets, then the proof carries on exactly as in \ZF.
    \item The cases when one is a low set and the other one is a high set are ruled out by lemma \coqref{NF2.Ext.pos neg ext neq}{\coqdoclemma{pos\_neg\_ext\_neq}}.
    \qedhere
  \end{itemize}
\end{proof}
