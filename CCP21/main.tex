\documentclass[sigplan,10pt,anonymous,review]{acmart}%
%\settopmatter{printfolios=true,printccs=false,printacmref=false}
% ??? Commented out?

% !TEX root = main.tex

\usepackage[color]{coqdoc}
\usepackage{amsmath,amssymb}
\usepackage{bbold}
\usepackage{hyperref}
\usepackage{marginnote}
\usepackage{xcolor}
\usepackage{stmaryrd}
\usepackage{graphicx}
\usepackage{ulem}
\normalem
% \usepackage{fancyvrb}

\newcommand\margincoq[1]{\marginpar{\rotatebox{270}{\footnotesize $\star$ [\coqid{#1}]}}}
\newcommand{\coqdefRef}[3]{\margincoq{#1}\coqdef{#1}{#2}{#3}}
\newcommand{\coqid}[1]{\texttt{#1}}

\newcommand\AEQ{\equiv_A}

% AUX
\newcommand\TODO[1]{\textcolor{red}{{\bf TODO} #1}}

\newcommand\Placeholder{\textcolor{gray}{\bullet}}

\DeclareUnicodeCharacter{1D4E5}{\ensuremath{\mathcal{V}}}
\DeclareUnicodeCharacter{2261}{\ensuremath{\equiv}}
\DeclareUnicodeCharacter{27E6}{\ensuremath{\llbracket}}
\DeclareUnicodeCharacter{27E7}{\ensuremath{\rrbracket}}
\DeclareUnicodeCharacter{2218}{\ensuremath{\circ}}
\DeclareUnicodeCharacter{2A01}{\ensuremath{\oplus}}
\DeclareUnicodeCharacter{2A00}{\ensuremath{\odot}}

% LOGIC
\newcommand\grameq{::=}
\newcommand\defeq{:=}
\newcommand\ie{\textit{i.e.}}
\newcommand\fv[1]{\mathrm{fv}(#1)}

% NF
\newcommand\ZF{{\sf ZF}}
\newcommand\NF{{\sf NF}}
\newcommand\NFTWO{{\tt NF2}}
\newcommand\NFO{{\tt NFO}}
\newcommand\Coq{{\tt Coq}}
\newcommand\SetCompr{{\textsc{SetCompr}}}
\newcommand\SetExt{{\textsc{SetExt}}}

\newcommand\sctag[1]{\tag{\textsc{#1}}}

\newcommand\Essence[1]{\Verb{Ess}\,#1}
% \newcommand\Sing[1]{\Verb{\{}#1\Verb{\}}}
% \newcommand\Sing[1]{\Verb{Sing}\,#1}
\newcommand\Sing[1]{\Verb{\{} #1 \Verb{\}}}
\newcommand\Complement[1]{\overline{#1}}
% \newcommand\Complement[1]{\Verb{Compl}\,#1}
\newcommand\Char[1]{\Verb{Char}_{#1}} % \mathbb{1}
\newcommand\Compr[2]{\Verb{\{} #1 ~\Verb{|}~ #2 \Verb{\}}}
\newcommand\Universe{U}
\newcommand\Verb[1]{\text{\texttt{#1}}}
% \newcommand\Union[2]{#1 \mathrel{\Verb{U}} #2}
\newcommand\Union[2]{#1 \cup #2}

% COQ DOC
\newcommand\Definition{\coqdockw{Definition}}
\newcommand\Math[1]{\coqdocnotation{\ensuremath{#1}}}
\newcommand\Var\coqdocvariable


%%
%% \BibTeX command to typeset BibTeX logo in the docs
\AtBeginDocument{%
  \providecommand\BibTeX{{%
    \normalfont B\kern-0.5em{\scshape i\kern-0.25em b}\kern-0.8em\TeX}}}

%% Rights management information.  This information is sent to you
%% when you complete the rights form.  These commands have SAMPLE
%% values in them; it is your responsibility as an author to replace
%% the commands and values with those provided to you when you
%% complete the rights form.
\setcopyright{acmcopyright}
\copyrightyear{2018}
\acmYear{2018}
\acmDOI{10.1145/1122445.1122456}

%% These commands are for a PROCEEDINGS abstract or paper.
\acmConference[Woodstock '18]{Woodstock '18: ACM Symposium on Neural
  Gaze Detection}{June 03--05, 2018}{Woodstock, NY}
\acmBooktitle{Woodstock '18: ACM Symposium on Neural Gaze Detection,
  June 03--05, 2018, Woodstock, NY}
\acmPrice{15.00}
\acmISBN{978-1-4503-XXXX-X/18/06}


%%
%% Submission ID.
%% Use this when submitting an article to a sponsored event. You'll
%% receive a unique submission ID from the organizers
%% of the event, and this ID should be used as the parameter to this command.
%%\acmSubmissionID{123-A56-BU3}

%%
%% The majority of ACM publications use numbered citations and
%% references.  The command \citestyle{authoryear} switches to the
%% "author year" style.
%%
%% If you are preparing content for an event
%% sponsored by ACM SIGGRAPH, you must use the "author year" style of
%% citations and references.
%% Uncommenting
%% the next command will enable that style.
%%\citestyle{acmauthoryear}

%%
%% end of the preamble, start of the body of the document source.
\begin{document}

%%
%% The "title" command has an optional parameter,
%% allowing the author to define a "short title" to be used in page headers.
\title{Universal Sets in Coq}

%%
%% The "author" command and its associated commands are used to define
%% the authors and their affiliations.
%% Of note is the shared affiliation of the first two authors, and the
%% "authornote" and "authornotemark" commands
%% used to denote shared contribution to the research.
\author{Andrea Condoluci}
\email{andreacondoluci@gmail.com}
\orcid{1234-5678-9012}
\affiliation{%
  \institution{Institute for Clarity in Documentation}
  \streetaddress{P.O. Box 1212}
  \city{Dublin}
  \state{Ohio}
  \postcode{43017-6221}
}

%%
%% By default, the full list of authors will be used in the page
%% headers. Often, this list is too long, and will overlap
%% other information printed in the page headers. This command allows
%% the author to define a more concise list
%% of authors' names for this purpose.
% \renewcommand{\shortauthors}{Trovato and Tobin, et al.}

%%
%% The abstract is a short summary of the work to be presented in the
%% article.
\begin{abstract}
New Foundations (\NF) is a somewhat bizarre alternative to mainstream set theories like Zermelo–Fraenkel (\ZF). The distinguishing feature of \NF{} is permitting the construction of a \emph{universal set}, \ie{} a set that contains all sets (itself, too). Though counterintuitive, \NF{} is capable of expressing ordinary mathematical concepts like well-orders, ordinals, cardinals, natural numbers, and so on. Its adoption, however, is hampered by the fact that after more than 80 years since its introduction, the community has not yet agreed upon its logical consistency.
  
In this paper, we consider two weak sub-theories of \NF{} whose consistency has already been established: \NFTWO{} and \NFO. These theories are obtained from \NF{} by restricting the axiom of comprehension to quantifier-free formulas only. We provide an encoding of \NFTWO{} and \NFO{} in the Coq proof assistant, adapting the classic encoding of \ZF{} in Coq by Benjamin Werner. To our knowledge, this is the first mechanized consistency proof concerning New Foundations.
\end{abstract}

% %%
% %% The code below is generated by the tool at http://dl.acm.org/ccs.cfm.
% %% Please copy and paste the code instead of the example below.
% %%
% \begin{CCSXML}
% <ccs2012>
%  <concept>
%   <concept_id>10010520.10010553.10010562</concept_id>
%   <concept_desc>Computer systems organization~Embedded systems</concept_desc>
%   <concept_significance>500</concept_significance>
%  </concept>
%  <concept>
%   <concept_id>10010520.10010575.10010755</concept_id>
%   <concept_desc>Computer systems organization~Redundancy</concept_desc>
%   <concept_significance>300</concept_significance>
%  </concept>
%  <concept>
%   <concept_id>10010520.10010553.10010554</concept_id>
%   <concept_desc>Computer systems organization~Robotics</concept_desc>
%   <concept_significance>100</concept_significance>
%  </concept>
%  <concept>
%   <concept_id>10003033.10003083.10003095</concept_id>
%   <concept_desc>Networks~Network reliability</concept_desc>
%   <concept_significance>100</concept_significance>
%  </concept>
% </ccs2012>
% \end{CCSXML}

% \ccsdesc[500]{Computer systems organization~Embedded systems}
% \ccsdesc[300]{Computer systems organization~Redundancy}
% \ccsdesc{Computer systems organization~Robotics}
% \ccsdesc[100]{Networks~Network reliability}

%%
%% Keywords. The author(s) should pick words that accurately describe
%% the work being presented. Separate the keywords with commas.
\keywords{set theory, New Foundations, universal set}

%%
%% This command processes the author and affiliation and title
%% information and builds the first part of the formatted document.
\maketitle

\section{Introduction}

\TODO{The history of the foundations of mathematics has been \dots}

\paragraph{Set theories.}
A \emph{set theory} is a first-order theory having the usual logical connectives ($\neg$, $\land$, $\lor$, $\to$, $\leftrightarrow$, $\exists$, $\forall$) plus two relation symbols: \emph{equality} ($=$) and \emph{membership} ($\in$). Variables (like $x,y,z,\ldots$) range over sets, and a set is meant to represent a collection of sets: $x \in y$ means that $x$ is a \emph{member} or an \emph{element} of $y$.

%Formulas of set theory are thus described by the following grammar: 

% \[\begin{array}{rrl}
%   \varphi & \grameq & \bot \mid \neg \varphi \mid \varphi \land \varphi \mid \varphi \lor \varphi \\
%   & & \mid \forall x.\, \varphi \mid \exists x.\, \varphi \\
%   & & \mid x = y \mid x \in y .
% % \TODO{Oh no, we need $\leftrightarrow$!!!}
% \end{array}\]

Sets do not posses any internal structure other than their \emph{extension}, \ie{} the elements they contain. The \emph{axiom of extensionality} formalizes this fact by requiring that sets having the same extension are the same set:

\[\forall x y.\, x = y \leftrightarrow \forall z. \, z \in x \leftrightarrow z \in y \quad \tag{\SetExt} \]

A natural way to form a collection of sets is by gathering all sets that satisfy a given predicate; this collection is expressed in symbols by $\Compr x \varphi$, reading ``the collection of all $x$'s such that $\varphi$ holds''. \TODO{this is called comprehension, and very powerful. for instance, ordinals, \dots} A very desirable ??? would be that collections specified by comprehension are sets themselves, as the \emph{axiom of comprehension} requires:

\[ \exists y.\, \forall x.\, x \in y \leftrightarrow \varphi  \quad \tag{\SetCompr} \]

\TODO{another point about extension is the identification between sets (citizens) and unary predicates over sets:}

Where $y$ does not occur in $\varphi$. \TODO{denote with $\Compr x \varphi$ the set which witnesses the existential $\exists y\ldots$, which is then unique by extensionality \dots }
Set comprehension is actually an axiom schema: one adds an axiom for every $\varphi$. \TODO{but which $\varphi$s are allowed? NOT ALL!}
When all $\varphi$s, unrestricted comprehension. \TODO{But Russel's paradox $\varphi \defeq x \not\in x$.\footnotemark{}}

\footnotetext{By $x\not\in x$ we mean $\neg {(x \in x)}$. }

Problem is circularity.

There are different ways to restrict \SetCompr{} so to avoid Russel's paradox. \ZF's way is to only allow comprehension of \emph{guarded} propositions, of the form $x \in y \land \varphi$ for some $y\neq x$ and an arbitrary $\varphi$. \TODO{$\Compr {x \in y} \varphi$} No self-reference because carve only subsets of already existing sets, and other axioms of ZF force sets to be well-founded (\dots).

NF's way is \dots to restrict the axiom of comprehension to only \emph{stratified} formulas \ldots
The axiom of stratified comprehension does not force the collection $\Compr x {x \not\in x}$ to be a set, because the formula $x \not\in x$ is not stratified. As a consequence, Russel's paradox is avoided.

\subsection{Quantifier-free Comprehension}

In this paper we consider a weaker fragment of full-fledged \NF{}, obtained by restricting the axiom of stratified comprehension to only quantifier-free formulas.

\TODO{dire che: in this sub-case, we can actually replace stratified comprehension with an equivalent axiomatization with a few set operators ($\Universe, \Complement \Placeholder, \Union \Placeholder \Placeholder, \Sing\Placeholder, \Essence\Placeholder$) and charachterizing axioms.}

\TODO{Dire che si puo' fare per Skolemization/ some Skolem theorem?}

By a rapid inspection, we note that when $\varphi$ is quantifier-free, $\varphi$ is stratified if and only if it does not contain ``$x \in x$'' as a subformula.
\TODO{This means that we can drop the definition of stratified, and use a much simplier syntactic restriction on $\varphi$'s}

To understand what operations on sets are necessary to , we 
We show that $\Compr x \varphi$ for $\varphi$ stratified and quantifier-free can . We blah by cases on the structure of $\varphi$:

\[\begin{array}{lcll}
  \Compr x {x = x} & = & \Universe \\
  \Compr x {x = y} & = & \Sing y \\
  \Compr x {y = x} & = & \Sing y \\
  \Compr x {y = z} & = &
    \begin{cases}
      \Universe & \text{if } y = z\\
      \Complement \Universe & \text{otherwise}
    \end{cases}\\
  \Compr x {x \in y} & = & y \\
  \Compr x {y \in x} & = & \Essence y \\
  \Compr x {y \in z} & = &
    \begin{cases}
      \Universe & \text{if } y \in z\\
      \Complement \Universe & \text{otherwise}
    \end{cases}\\
  \Compr x {\neg\varphi} & = & \Complement {\Compr x \varphi} \\
  \Compr x {\varphi \lor \varphi'} & = & \Union {\Compr x \varphi} {\Compr x {\varphi'}} \\
\end{array}\]

where $x\neq y$ and $x \neq z$ above.

where
\begin{enumerate} \renewcommand\labelitemi{--}
  \item $\Universe$ is the universal set, \ie{} the set containing all sets:
    \[ \forall z.\, z \in \Universe \sctag{Universe} \]
  \item $\Complement \Placeholder$ is the set complement:
    \[ \forall z.\, z \in \Complement x \leftrightarrow \neg (z \in x) \sctag{Complement}\]
  \item $\Union \Placeholder \Placeholder$ is the set union:
    \[ \forall z.\, z \in \Union x y \leftrightarrow z \in x \lor z \in y \sctag{Union} \]
  \item $\Sing\Placeholder$ is singleton:
    \[ \forall z.\, z \in \Sing x \leftrightarrow x = z \sctag{Singleton} \]
  \item $\Essence\Placeholder$ is the essence:
    \[ \forall z.\, z \in \Essence{x} \leftrightarrow x \in z \sctag{Essence} \]
\end{enumerate}

\subsection{Sets in Coq}
\TODO{Descrivere Coq e la formalizzazione di ZF in Coq.}

\TODO{We use a \emph{shallow embedding} of \NF{} in the Calculus of Constructions: the logic formulas of \NF{} are interpreted in \Coq{} in such a way that quantifiers and logical connectives of \NF{} are mapped to the corresponding ones in \Coq{}}

\section{Roadmap}
Dire:
\begin{itemize}
  \item Come si fara' in ogni sezione: prima definire il dominio, poi l'uguaglianza, poi membership, poi morphism, poi extensionality e set operators independently.
  \item Contenuto delle sezioni sotto
\end{itemize}

\section{\ZF}
%!TEX root = main.tex

Let us start by summarizing the classic encoding of \ZF{} in Coq by Benjamin Werner \cite{DBLP:conf/tacs/Werner97}\footnotemark{}, which \TODO{as already mention Aczel.}

\footnotetext{The source code of the encoding of \ZF{} is included in the repository of Coq users' contributions, and can be found at the address \url{https://github.com/coq-contribs/zfc/}.}

The steps are the following:
\begin{itemize}
  \item \emph{Type of sets:} defined \coqdocinductive{SET}\footnotemark{}
  \item \emph{(Extensional) equality:} define \coqdocdefinition{EQ}. setoid, etc.
  \item \emph{Set membership:} define \coqdocdefinition{IN} morphism
  \item \emph{Axiom of extensionality}
  \item \emph{Remaining axiom / set operators} and their properties
\end{itemize}

\footnotetext{We use the uppercase identifiers \coqdocinductive{SET}, \coqdocdefinition{EQ}, and \coqdocdefinition{IN} in order to avoid shadowing, respectively, the kind \coqdocdefinition{Set}, the standard (intensional) equality \coqdocdefinition{eq}, and the reserved keyword \coqdockw{in}.}

\subsection{Type of sets}
The type representing sets

\begin{coqdoccode}
\coqdocnoindent
\coqdockw{Inductive} \coqdocinductive{SET} : \coqdockw{Type} :=\coqdoceol
\coqdocindent{2.00em}
\coqdocconstructor{Sup} : \coqdockw{\ensuremath{\forall}} \coqdocvar{X} : \coqdockw{Type}, (\coqdocvar{X} \ensuremath{\rightarrow} \coqdocinductive{SET}) \ensuremath{\rightarrow} \coqdocinductive{SET}.\coqdoceol
\coqdocemptyline
\end{coqdoccode}

\TODO{index type $X$ plus an indexing function}

\begin{example}[Empty set]aaa

  \begin{coqdoccode}
    \coqdocnoindent
    \coqdockw{Definition} \coqdocdefinition{Ø} := \coqdocconstructor{Sup} \coqexternalref{False}{http://coq.inria.fr/distrib/V8.11.0/stdlib//Coq.Init.Logic}{\coqdocinductive{False}} (\coqdockw{fun} \coqdocvar{x} \ensuremath{\Rightarrow} \coqdockw{match} \coqdocvariable{x} \coqdockw{with} \coqdockw{end}).\coqdoceol
    \coqdocemptyline
  \end{coqdoccode}
  \TODO{False is just a type with no inhabitants, explain empty pattern matching}
\end{example}

\begin{example}[Set union]
  Index type is the disjoint union of the indexes, and the indexing is the sum of the indexings:
  \begin{coqdoccode}
    \coqdocemptyline
    \coqdocnoindent
    \coqdockw{Definition} \coqdocdefinition{union} \coqdocvar{a} \coqdocvar{b} := \coqdockw{match} \coqdocvar{a}, \coqdocvar{b} \coqdockw{with}\coqdoceol
    \coqdocindent{1.00em}
    \coqdocconstructor{Sup} \coqdocvar{X} \coqdocvar{f}, \coqdocconstructor{Sup} \coqdocvar{Y} \coqdocvar{g} \ensuremath{\Rightarrow} \coqdocvar{Sup} (\coqdocvar{X} + \coqdocvar{Y}) (\coqdocvar{f} \coqref{Internal.Misc.:::x 'xE2xA8x81' x}{\coqdocnotation{⨁}} \coqdocvar{g})\coqdoceol
    \coqdocnoindent
    \coqdockw{end}.\coqdoceol
  \end{coqdoccode}

  \TODO{+ is the disjoint sum (union) of types, and ⨁ is the sum of functions, defined as follows:}

  \begin{coqdoccode}
    \coqdocnoindent
    \coqdockw{Definition} \coqdef{Internal.Misc.sumF}{sumF}{\coqdocdefinition{sumF}} \{\coqdocvar{X} \coqdocvar{Y} \coqdocvar{Z}\} \coqdoceol
    \coqdocindent{1.00em}: \coqexternalref{::type scope:x '->' x}{http://coq.inria.fr/distrib/V8.11.0/stdlib//Coq.Init.Logic}{\coqdocnotation{(}}\coqdocvariable{X} \coqexternalref{::type scope:x '->' x}{http://coq.inria.fr/distrib/V8.11.0/stdlib//Coq.Init.Logic}{\coqdocnotation{\ensuremath{\rightarrow}}} \coqdocvariable{Z}\coqexternalref{::type scope:x '->' x}{http://coq.inria.fr/distrib/V8.11.0/stdlib//Coq.Init.Logic}{\coqdocnotation{)}} \coqexternalref{::type scope:x '->' x}{http://coq.inria.fr/distrib/V8.11.0/stdlib//Coq.Init.Logic}{\coqdocnotation{\ensuremath{\rightarrow}}} \coqexternalref{::type scope:x '->' x}{http://coq.inria.fr/distrib/V8.11.0/stdlib//Coq.Init.Logic}{\coqdocnotation{(}}\coqdocvariable{Y} \coqexternalref{::type scope:x '->' x}{http://coq.inria.fr/distrib/V8.11.0/stdlib//Coq.Init.Logic}{\coqdocnotation{\ensuremath{\rightarrow}}} \coqdocvariable{Z}\coqexternalref{::type scope:x '->' x}{http://coq.inria.fr/distrib/V8.11.0/stdlib//Coq.Init.Logic}{\coqdocnotation{)}} \coqexternalref{::type scope:x '->' x}{http://coq.inria.fr/distrib/V8.11.0/stdlib//Coq.Init.Logic}{\coqdocnotation{\ensuremath{\rightarrow}}} \coqdocvariable{X} \coqexternalref{::type scope:x '+' x}{http://coq.inria.fr/distrib/V8.11.0/stdlib//Coq.Init.Datatypes}{\coqdocnotation{+}} \coqdocvariable{Y} \coqexternalref{::type scope:x '->' x}{http://coq.inria.fr/distrib/V8.11.0/stdlib//Coq.Init.Logic}{\coqdocnotation{\ensuremath{\rightarrow}}} \coqdocvariable{Z} \coqdoceol
    \coqdocindent{1.00em} := \coqdockw{fun} \coqdocvar{f} \coqdocvar{g} \coqdocvar{xy} \ensuremath{\Rightarrow} \coqdockw{match} \coqdocvariable{xy} \coqdockw{with}\coqdoceol
    \coqdocindent{8.00em}
    \ensuremath{|} \coqexternalref{inl}{http://coq.inria.fr/distrib/V8.11.0/stdlib//Coq.Init.Datatypes}{\coqdocconstructor{inl}} \coqdocvar{x} \ensuremath{\Rightarrow} \coqdocvariable{f} \coqdocvar{x}\coqdoceol
    \coqdocindent{8.00em}
    \ensuremath{|} \coqexternalref{inr}{http://coq.inria.fr/distrib/V8.11.0/stdlib//Coq.Init.Datatypes}{\coqdocconstructor{inr}} \coqdocvar{y} \ensuremath{\Rightarrow} \coqdocvariable{g} \coqdocvar{y}\coqdoceol
    \coqdocindent{8.00em}
    \coqdockw{end}.\coqdoceol
    \coqdocnoindent
    \coqdockw{Infix} \coqdef{Internal.Misc.:::x 'xE2xA8x81' x}{"}{"}⨁" := \coqref{Internal.Misc.sumF}{\coqdocdefinition{sumF}} (\coqdoctac{at} \coqdockw{level} 80, \coqdoctac{right} \coqdockw{associativity}).\coqdoceol
    \coqdocemptyline
    \end{coqdoccode}
\end{example}

\subsection{Equality}

Recursive Definition of the extentional equality on sets

\begin{coqdoccode}
  \coqdocnoindent
  \coqdockw{Fixpoint} \coqdocdefinition{EQ} \coqdocvar{a} \coqdocvar{b} : \coqdockw{Prop} :=\coqdoceol
  \coqdocindent{1.00em}
  \coqdockw{match} \coqdocvar{a}, \coqdocvar{b} \coqdockw{with}\coqdoceol
  \coqdocindent{1.00em}
  \ensuremath{|} \coqdocconstructor{Sup} \coqdocvar{X} \coqdocvar{f}, \coqdocconstructor{Sup} \coqdocvar{Y} \coqdocvar{g} \ensuremath{\Rightarrow}\coqdoceol
  \coqdocindent{2.00em}
  (\coqdockw{\ensuremath{\forall}} \coqdocvar{x}, \coqdoctac{\ensuremath{\exists}} \coqdocvar{y}, \coqdocvar{EQ} (\coqdocvar{f} \coqdocvar{x}) (\coqdocvar{g} \coqdocvar{y})) \ensuremath{\land} (\coqdockw{\ensuremath{\forall}} \coqdocvar{y}, \coqdoctac{\ensuremath{\exists}} \coqdocvar{x}, \coqdocvar{EQ} (\coqdocvar{f} \coqdocvar{x}) (\coqdocvar{g} \coqdocvar{y}))\coqdoceol
  \coqdocindent{1.00em}
  \coqdockw{end}.\coqdoceol
  \coqdocnoindent
\coqdockw{Notation} ``X ≡ Y'' := (\coqref{NF2.Model.EQ}{\coqdocdefinition{EQ}} \coqdocvar{X} \coqdocvar{Y}).\coqdoceol
\coqdocemptyline
\end{coqdoccode}

\TODO{parlare di Coq fixpoint}

Aczel's equality


\begin{coqdoccode}
\coqdocnoindent
\coqdockw{Definition} \coqdefRef{Internal.Misc.eq aczel}{eq\_aczel}{\coqdocdefinition{eq\_aczel}} \{\coqdocvar{X} \coqdocvar{Y} \coqdocvar{Z}\} (\coqdocvar{R}: \coqdocvariable{Z} \coqexternalref{::type scope:x '->' x}{http://coq.inria.fr/distrib/V8.11.0/stdlib//Coq.Init.Logic}{\coqdocnotation{\ensuremath{\rightarrow}}} \coqdocvariable{Z} \coqexternalref{::type scope:x '->' x}{http://coq.inria.fr/distrib/V8.11.0/stdlib//Coq.Init.Logic}{\coqdocnotation{\ensuremath{\rightarrow}}} \coqdockw{Prop}) \coqdocvar{f} \coqdocvar{g} :=\coqdoceol
\coqdocindent{1.00em}
\coqexternalref{::type scope:x '/x5C' x}{http://coq.inria.fr/distrib/V8.11.0/stdlib//Coq.Init.Logic}{\coqdocnotation{(}}\coqdockw{\ensuremath{\forall}} \coqdocvar{x}: \coqdocvariable{X}, \coqexternalref{::type scope:'exists' x '..' x ',' x}{http://coq.inria.fr/distrib/V8.11.0/stdlib//Coq.Init.Logic}{\coqdocnotation{\ensuremath{\exists}}} \coqdocvar{y}\coqexternalref{::type scope:'exists' x '..' x ',' x}{http://coq.inria.fr/distrib/V8.11.0/stdlib//Coq.Init.Logic}{\coqdocnotation{,}} \coqdocvariable{R} (\coqdocvariable{f} \coqdocvariable{x}) (\coqdocvariable{g} \coqdocvariable{y})\coqexternalref{::type scope:x '/x5C' x}{http://coq.inria.fr/distrib/V8.11.0/stdlib//Coq.Init.Logic}{\coqdocnotation{)}}\coqdoceol
\coqdocindent{1.00em}
\coqexternalref{::type scope:x '/x5C' x}{http://coq.inria.fr/distrib/V8.11.0/stdlib//Coq.Init.Logic}{\coqdocnotation{\ensuremath{\land}}} \coqdockw{\ensuremath{\forall}} \coqdocvar{y}: \coqdocvariable{Y}, \coqexternalref{::type scope:'exists' x '..' x ',' x}{http://coq.inria.fr/distrib/V8.11.0/stdlib//Coq.Init.Logic}{\coqdocnotation{\ensuremath{\exists}}} \coqdocvar{x}\coqexternalref{::type scope:'exists' x '..' x ',' x}{http://coq.inria.fr/distrib/V8.11.0/stdlib//Coq.Init.Logic}{\coqdocnotation{,}} \coqdocvariable{R} (\coqdocvariable{f} \coqdocvariable{x}) (\coqdocvariable{g} \coqdocvariable{y}).\coqdoceol
\coqdocemptyline
\end{coqdoccode}

\coqdocconstructor{Sup} \coqdocvar{X} \coqdocvar{f} $\equiv$ \coqdocconstructor{Sup} \coqdocvar{Y} \coqdocvar{g} $~\Leftrightarrow$ \coqdocdefinition{eq\_aczel} \coqdocvar{EQ} \coqdocvar{f} \coqdocvar{g}

\begin{coqdoccode}
  \coqdocnoindent
  \coqdockw{Infix} ``\ensuremath{\AEQ}'' := (\coqdocdefinition{eq\_aczel} \coqdocvar{EQ}).\coqdoceol
\end{coqdoccode}

\coqdocconstructor{Sup} \coqdocvar{X} \coqdocvar{f} $\equiv$ \coqdocconstructor{Sup} \coqdocvar{Y} \coqdocvar{g} $~\Leftrightarrow$ \coqdocvar{f} $\AEQ$ \coqdocvar{g}

\TODO{symmetry and transitivity proved by an easy induction on the first variable.}

\subsection{Set Membership}

\begin{coqdoccode}
  \coqdocnoindent
\coqdockw{Definition} \coqdocdefinition{IN} \coqdocvar{a} \coqdocvar{b} : \coqdockw{Prop} :=
\coqdockw{match} \coqdocvar{b} \coqdockw{with}\coqdoceol
\coqdocindent{1.00em}
\ensuremath{|} \coqdocconstructor{Sup} \coqdocvar{X} \coqdocvar{f} \ensuremath{\Rightarrow} \coqdoctac{\ensuremath{\exists}} \coqdocvar{x}, \coqdocvar{EQ} \coqdocvar{a} (\coqdocvar{f} \coqdocvar{x})\coqdoceol
\coqdocindent{0.00em}\coqdockw{end}.\coqdoceol
\coqdocnoindent
\coqdockw{Infix} ``X ∈ Y'' := (\coqdocdefinition{IN} \coqdocvar{X} \coqdocvar{Y}).\coqdoceol
\coqdocemptyline
\end{coqdoccode}


\begin{coqdoccode}
  \coqdocnoindent
\coqdockw{Definition} \coqdefRef{Internal.Common.in aczel}{in\_aczel}{\coqdocdefinition{in\_aczel}} \{\coqdocvar{X} \coqdocvar{Z}\} (\coqdocvar{R}: \coqdocvariable{Z} \coqexternalref{::type scope:x '->' x}{http://coq.inria.fr/distrib/V8.11.0/stdlib//Coq.Init.Logic}{\coqdocnotation{\ensuremath{\rightarrow}}} \coqdocvariable{Z} \coqexternalref{::type scope:x '->' x}{http://coq.inria.fr/distrib/V8.11.0/stdlib//Coq.Init.Logic}{\coqdocnotation{\ensuremath{\rightarrow}}} \coqdockw{Prop}) \coqdocvar{z} \coqdocvar{f} :=\coqdoceol
\coqdocindent{1.00em}
\coqexternalref{::type scope:'exists' x '..' x ',' x}{http://coq.inria.fr/distrib/V8.11.0/stdlib//Coq.Init.Logic}{\coqdocnotation{\ensuremath{\exists}}} \coqdocvar{x}: \coqdocvariable{X}\coqexternalref{::type scope:'exists' x '..' x ',' x}{http://coq.inria.fr/distrib/V8.11.0/stdlib//Coq.Init.Logic}{\coqdocnotation{,}} \coqdocvariable{R} (\coqdocvariable{f} \coqdocvariable{x}) \coqdocvariable{z}.\coqdoceol
\coqdocindent{0.00em}
\coqdockw{Infix} ``\AIN'' := (\coqdocdefinition{in\_aczel} \coqdocdefinition{IN}).\coqdoceol
\coqdocemptyline
\end{coqdoccode}

\section{\NFTWO}
%!TEX root = main.tex

\renewcommand\INX{\coqref{NF2.Model.IN}{{\IN}}~}
\renewcommand\EQX{\coqref{NF2.Model.EQ}{{\EQ}}~}

\NFTWO{} is the sub-theory of \NF{} characterized by the axiom of extensionality plus universe ($\Universe$), complement ($\Complement \Placeholder$), union ($\Union \Placeholder \Placeholder$), and singleton ($\Sing\Placeholder$) and their related axioms.

% $\Universe, \Complement \Placeholder, \Union \Placeholder \Placeholder, \Sing\Placeholder, \Essence\Placeholder$

\medskip

A first attempt at constructing a model for \NFTWO{} could be to take the \emph{free algebra} generated by these set constructors, basically identifying sets with free expressions of sets. Unfortunately, this approach is bound to fail:

\begin{itemize}
  \item 
  When defining the membership relation \coqdocdefinition{IN}, the singleton constructor $\Sing\Placeholder$ forces \coqdocdefinition{IN} to depend on \coqdocdefinition{EQ}, because \var s $\in \Sing{\text{\var t}}$ if and only if \var t $\equiv$ \var s.
  \item To define \coqdocdefinition{EQ} there is no other option than taking as definition exactly extensionality, \ie{} \\
  % 
  \centerline{
  \coqdocdefinition{EQ} \var s \var {s'} $\Leftrightarrow$ \coqdockw{\ensuremath{\forall}} \coqdocvar{t}, \coqdocvariable{t} \coqdocnotation{∈} \coqdocvariable{s} \coqexternalref{::type scope:x '<->' x}{http://coq.inria.fr/distrib/V8.11.0/stdlib//Coq.Init.Logic}{\coqdocnotation{\ensuremath{\leftrightarrow}}} \coqdocvariable{t} \coqdocnotation{∈} \coqdocvariable{s'}.}
  
  This makes \coqdocdefinition{EQ} depend on \coqdocdefinition{IN}, which makes both definitions invalid because no argument is decreasing.
\end{itemize}

The most promising way to construct a model is actually to first simplify set expressions, finding a ``normal form'' which makes extensional equality as close as possible to intensional equality.

Let us point out that the real difference between the set operators in \NFTWO{} and \ZF{} is the complement ($\Complement \Placeholder$), which is not definable in \ZF{}. A good bet may be that in \NFTWO{} there exist two kinds of sets, which we call \emph{low} and \emph{high} (following \cite{Forster2001}). A low set is similar to a \ZF{} set in that it is a positive collection of sets; a high set, instead, is the complement of a low set, and its extension includes all sets but the ones in a given collection --- in terms of \ZF{}, it actually consists of a proper class. This bet is correct: as we will see in the next sections, low and high sets are stable under boolean combinations. Moreover, they satisfy extensionality.

\subsection{The Type \texorpdfstring{\coqdocinductive{SET}}{SET}}

To define the type of \NFTWO{} sets, we extend Aczel's construction to accommodate high sets, thus introducing an additonal constructor:

\begin{coqdoccode}
  \coqdocnoindent
  \coqdockw{Inductive} \coqdefRef{NF2.Model.SET}{SET}{\coqdocinductive{SET}} :=\coqdoceol
  \coqdocindent{1.00em}
  \ensuremath{|} \coqdef{NF2.Model.Low}{Low}{\coqdocconstructor{Low}} : \coqdocnotation{\ensuremath{\forall}} \coqdocvar{X}, \coqexternalref{::type scope:x '->' x}{http://coq.inria.fr/distrib/V8.11.0/stdlib//Coq.Init.Logic}{\coqdocnotation{(}}\coqdocvariable{X} \coqexternalref{::type scope:x '->' x}{http://coq.inria.fr/distrib/V8.11.0/stdlib//Coq.Init.Logic}{\coqdocnotation{\ensuremath{\rightarrow}}} \coqref{NF2.Model.SET}{\coqdocinductive{SET}}\coqexternalref{::type scope:x '->' x}{http://coq.inria.fr/distrib/V8.11.0/stdlib//Coq.Init.Logic}{\coqdocnotation{)}} \coqexternalref{::type scope:x '->' x}{http://coq.inria.fr/distrib/V8.11.0/stdlib//Coq.Init.Logic}{\coqdocnotation{\ensuremath{\rightarrow}}} \coqref{NF2.Model.SET}{\coqdocinductive{SET}}\coqdoceol
  \coqdocindent{1.00em}
  \ensuremath{|} \coqdef{NF2.Model.High}{High}{\coqdocconstructor{High}} : \coqdocnotation{\ensuremath{\forall}} \coqdocvar{X}, \coqexternalref{::type scope:x '->' x}{http://coq.inria.fr/distrib/V8.11.0/stdlib//Coq.Init.Logic}{\coqdocnotation{(}}\coqdocvariable{X} \coqexternalref{::type scope:x '->' x}{http://coq.inria.fr/distrib/V8.11.0/stdlib//Coq.Init.Logic}{\coqdocnotation{\ensuremath{\rightarrow}}} \coqref{NF2.Model.SET}{\coqdocinductive{SET}}\coqexternalref{::type scope:x '->' x}{http://coq.inria.fr/distrib/V8.11.0/stdlib//Coq.Init.Logic}{\coqdocnotation{)}} \coqexternalref{::type scope:x '->' x}{http://coq.inria.fr/distrib/V8.11.0/stdlib//Coq.Init.Logic}{\coqdocnotation{\ensuremath{\rightarrow}}} \coqref{NF2.Model.SET}{\coqdocinductive{SET}}
  .\coqdoceol
  \coqdocemptyline
\end{coqdoccode}

Note that the indexing function \var f in the set \var s = \coqdef{NF2.Model.High}{High}{\coqdocconstructor{High}} \var X \var f does not actually index the sets in the extension of \var s, but the sets that are \emph{not} in the extension of \var s (because a high set is supposed to represent a complement).

\begin{example}[Empty set and Universe]\label{ex:EUNF2}
The empty set is defined as a low set with empty index:

  \begin{coqdoccode}
    \coqdocnoindent
    \coqdockw{Definition} \coqdocdefinition{Ø} := \coqdocconstructor{Low} \coqexternalref{False}{http://coq.inria.fr/distrib/V8.11.0/stdlib//Coq.Init.Logic}{\coqdocinductive{False}} (\coqdockw{fun} \coqdocvar{x} \ensuremath{\Rightarrow} \coqdockw{match} \coqdocvariable{x} \coqdockw{with} \coqdockw{end}).\coqdoceol
    \coqdocemptyline
  \end{coqdoccode}

  In a fully specular way, the universal set is defined as a \emph{high} set with empty index:

  \begin{coqdoccode}
    \coqdocnoindent
    \coqdockw{Definition} \coqdocdefinition{U} := \coqdocconstructor{High} \coqexternalref{False}{http://coq.inria.fr/distrib/V8.11.0/stdlib//Coq.Init.Logic}{\coqdocinductive{False}} (\coqdockw{fun} \coqdocvar{x} \ensuremath{\Rightarrow} \coqdockw{match} \coqdocvariable{x} \coqdockw{with} \coqdockw{end}).\coqdoceol
    \coqdocemptyline
  \end{coqdoccode}
\end{example}

\subsection{Set Equality}

To define equality of \NFO{} sets, we enforce our intuition that a low set cannot be equivalent to a high set (because high sets are much larger):

\begin{coqdoccode}
  \coqdocnoindent
\coqdockw{Fixpoint} \coqdefRef{NF2.Model.EQ}{EQ}{\coqdocdefinition{EQ}} \coqdocvar{s} \coqdocvar{t} := \coqdockw{match} \coqdocvariable{s},\coqdocvariable{t} \coqdockw{with}\coqdoceol
\coqdocindent{1.00em}
\ensuremath{|} \coqref{NF2.Model.Low}{\coqdocconstructor{Low}} \coqdocvar{\_} \coqdocvar{\_}, \coqref{NF2.Model.High}{\coqdocconstructor{High}} \coqdocvar{\_} \coqdocvar{\_} \ensuremath{\Rightarrow} \coqexternalref{False}{http://coq.inria.fr/distrib/V8.11.0/stdlib//Coq.Init.Logic}{\coqdocinductive{False}}\coqdoceol
\coqdocindent{1.00em}
\ensuremath{|} \coqref{NF2.Model.High}{\coqdocconstructor{High}} \coqdocvar{\_} \coqdocvar{\_}, \coqref{NF2.Model.Low}{\coqdocconstructor{Low}} \coqdocvar{\_} \coqdocvar{\_} \ensuremath{\Rightarrow} \coqexternalref{False}{http://coq.inria.fr/distrib/V8.11.0/stdlib//Coq.Init.Logic}{\coqdocinductive{False}}\coqdoceol
\coqdocindent{1.00em}
\ensuremath{|} \coqref{NF2.Model.Low}{\coqdocconstructor{Low}} \coqdocvar{\_} \coqdocvar{f}, \coqref{NF2.Model.Low}{\coqdocconstructor{Low}} \coqdocvar{\_} \coqdocvar{g} \ensuremath{\Rightarrow} \coqdocvar{f} $\AEQ$ \coqdocvar{g} \coqdoceol
\coqdocindent{1.00em}
\ensuremath{|} \coqref{NF2.Model.High}{\coqdocconstructor{High}} \coqdocvar{\_} \coqdocvar{f}, \coqref{NF2.Model.High}{\coqdocconstructor{High}} \coqdocvar{\_} \coqdocvar{g} \ensuremath{\Rightarrow} \coqdocvar{f} $\AEQ$ \coqdocvar{g} \coqdoceol
\coqdocnoindent
\coqdockw{end}.\coqdoceol

\end{coqdoccode}

We are a bit cheating here: in fact, the notation for \AEQ{} hides a dependency on \coqdocdefinition{EQ}, which would actually make the \coqdockw{Fixpoint} definition fail as it conceals what argument is decreasing. For the actual \Coq{} definition, we inlined \coqdocdefinition{eq\_aczel} so to make \Coq{} recognise the decreasing argument. However, we decided to just use the succint notation above, so to help the intuition.

\medskip

Proving that \coqdocdefinition{EQ} is an equivalence relation goes smoothly and just like in \ZF{}; the only difference here is the cases to consider are doubled because of the two constructors:

\begin{coqdoccode}
  \coqdocnoindent
  \coqdockw{Instance} \coqdefRef{NF2.Model.nf2 setoid}{nf2\_setoid}{\coqdocinstance{nf2\_setoid}} : \coqexternalref{Equivalence}{http://coq.inria.fr/distrib/V8.11.0/stdlib//Coq.Classes.RelationClasses}{\coqdocclass{Equivalence}} \coqref{NF2.Model.EQ}{\coqdocdefinition{EQ}}.
\end{coqdoccode}

\subsection{Set Membership}

To define membership of \NFTWO{} sets, 

\begin{coqdoccode}
  \coqdocnoindent
\coqdockw{Definition} \coqdefRef{NF2.Model.IN}{IN}{\coqdocdefinition{IN}} \coqdocvar{s} \coqdocvar{t} := \coqdockw{match} \coqdocvariable{t} \coqdockw{with}\coqdoceol
\coqdocindent{1.00em}
\ensuremath{|} \coqref{NF2.Model.Low}{\coqdocconstructor{Low}} \coqdocvar{\_} \coqdocvar{f} \ensuremath{\Rightarrow} \coqdocvariable{s} \coqref{Internal.Common.in aczel}{\AIN} \coqdocvariable{f} \coqdoceol
\coqdocindent{1.00em}
\ensuremath{|} \coqref{NF2.Model.High}{\coqdocconstructor{High}} \coqdocvar{\_} \coqdocvar{f} \ensuremath{\Rightarrow} \coqdocvariable{s} \coqref{Internal.Common.in aczel}{$\not\AIN$} \coqdocvariable{f} \coqdoceol
\coqdocnoindent
\coqdockw{end}.\coqdoceol
\end{coqdoccode}

\medskip

Proving that \coqdocdefinition{IN} is a morphism goes smoothly and just like in \ZF{}:

\begin{coqdoccode}
  \coqdocnoindent
\coqdockw{Add} \coqdockw{Morphism} \coqref{NF2.Model.IN}{\coqdocdefinition{IN}} \coqdockw{with} \coqdockw{signature} \coqref{NF2.Model.EQ}{\coqdocdefinition{EQ}} \coqexternalref{ProperNotations.::signature scope:x '==>' x}{http://coq.inria.fr/distrib/V8.11.0/stdlib//Coq.Classes.Morphisms}{\coqdocnotation{$\Rightarrow$}} \coqref{NF2.Model.EQ}{\coqdocdefinition{EQ}} \coqexternalref{ProperNotations.::signature scope:x '==>' x}{http://coq.inria.fr/distrib/V8.11.0/stdlib//Coq.Classes.Morphisms}{\coqdocnotation{$\Rightarrow$}} \coqexternalref{iff}{http://coq.inria.fr/distrib/V8.11.0/stdlib//Coq.Init.Logic}{\coqdocdefinition{iff}} \coqdockw{as} \coqdocdefinition{IN\_mor}.\coqdoceol
\coqdocnoindent
\end{coqdoccode}

\subsection{Set Operators}

Complementing a \NFTWO{} set is straighforward: the complement of a low set is a high set, and viceversa:

\begin{coqdoccode}
  \coqdockw{Definition} \coqdef{NF2.Sets.compl}{compl}{\coqdocdefinition{compl}} : \coqref{NF2.Model.SET}{\coqdocinductive{SET}} \coqexternalref{::type scope:x '->' x}{http://coq.inria.fr/distrib/V8.11.0/stdlib//Coq.Init.Logic}{\coqdocnotation{\ensuremath{\rightarrow}}} \coqref{NF2.Model.SET}{\coqdocinductive{SET}} :=\coqdoceol
\coqdocindent{1.00em}
\coqdockw{fun} \coqdocvar{s} \ensuremath{\Rightarrow} \coqdockw{match} \coqdocvariable{s} \coqdockw{with}\coqdoceol
\coqdocindent{1.00em}
\ensuremath{|} \coqref{NF2.Model.Low}{\coqdocconstructor{Low}} \coqdocvar{\_} \coqdocvar{f} \ensuremath{\Rightarrow} \coqref{NF2.Model.High}{\coqdocconstructor{High}} \coqdocvar{\_} \coqdocvar{f}\coqdoceol
\coqdocindent{1.00em}
\ensuremath{|} \coqref{NF2.Model.High}{\coqdocconstructor{High}} \coqdocvar{\_} \coqdocvar{f} \ensuremath{\Rightarrow} \coqref{NF2.Model.Low}{\coqdocconstructor{Low}} \coqdocvar{\_} \coqdocvar{f}\coqdoceol
\coqdocindent{1.00em}
\coqdockw{end}.\coqdoceol
\coqdocemptyline
\coqdocnoindent
\coqdockw{Lemma} \coqdef{NF2.Sets.compl ok}{compl\_ok}{\coqdoclemma{compl\_ok}} : \coqdockw{\ensuremath{\forall}} \coqdocvar{s} \coqdocvar{t}, \coqdocvariable{s} \coqref{NF2.Model.:::x 'xE2x88x88' x}{\coqdocnotation{∈}} \coqref{NF2.Sets.compl}{\coqdocdefinition{compl}} \coqdocvariable{t} \coqexternalref{::type scope:x '<->' x}{http://coq.inria.fr/distrib/V8.11.0/stdlib//Coq.Init.Logic}{\coqdocnotation{\ensuremath{\leftrightarrow}}} \coqexternalref{::type scope:x '<->' x}{http://coq.inria.fr/distrib/V8.11.0/stdlib//Coq.Init.Logic}{\coqdocnotation{(}}\coqdocvariable{s} \coqref{NF2.Model.:::x 'xE2x88x88' x}{\coqdocnotation{∈}} \coqdocvariable{t} \coqexternalref{::type scope:x '->' x}{http://coq.inria.fr/distrib/V8.11.0/stdlib//Coq.Init.Logic}{\coqdocnotation{\ensuremath{\rightarrow}}} \coqexternalref{False}{http://coq.inria.fr/distrib/V8.11.0/stdlib//Coq.Init.Logic}{\coqdocinductive{False}}\coqexternalref{::type scope:x '<->' x}{http://coq.inria.fr/distrib/V8.11.0/stdlib//Coq.Init.Logic}{\coqdocnotation{)}}.\coqdoceol
\end{coqdoccode}
\begin{proof}
  Just destruct \var t and apply few simplifications. Note: classical reasoning is required to turn a negated universal quantifier into an existential quantifier.
\end{proof}

A singleton is just a set whose indexing function is constant:

\begin{coqdoccode}
  \coqdocnoindent
\coqdockw{Definition} \coqdef{NF2.Sets.sing}{sing}{\coqdocdefinition{sing}} : \coqref{NF2.Model.SET}{\coqdocinductive{SET}} \coqexternalref{::type scope:x '->' x}{http://coq.inria.fr/distrib/V8.11.0/stdlib//Coq.Init.Logic}{\coqdocnotation{\ensuremath{\rightarrow}}} \coqref{NF2.Model.SET}{\coqdocinductive{SET}} :=\coqdoceol
\coqdocindent{1.00em}
\coqdockw{fun} \coqdocvar{s} \ensuremath{\Rightarrow} \coqref{NF2.Model.Low}{\coqdocconstructor{Low}} \coqexternalref{unit}{http://coq.inria.fr/distrib/V8.11.0/stdlib//Coq.Init.Datatypes}{\coqdocinductive{unit}} (\coqdockw{fun} \coqdocvar{\_} \ensuremath{\Rightarrow} \coqdocvariable{s}).\coqdoceol
\coqdocemptyline
\coqdocnoindent
\coqdockw{Definition} \coqdef{NF2.Sets.sing ok}{sing\_ok}{\coqdocdefinition{sing\_ok}} : \coqdockw{\ensuremath{\forall}} \coqdocvar{s} \coqdocvar{t}, \coqdocvariable{s} \coqref{NF2.Model.:::x 'xE2x88x88' x}{\coqdocnotation{∈}} \coqref{NF2.Sets.sing}{\coqdocdefinition{sing}} \coqdocvariable{t} \coqexternalref{::type scope:x '<->' x}{http://coq.inria.fr/distrib/V8.11.0/stdlib//Coq.Init.Logic}{\coqdocnotation{\ensuremath{\leftrightarrow}}} \coqdocvariable{t} \coqref{NF2.Model.:::x 'xE2x89xA1' x}{\coqdocnotation{≡}} \coqdocvariable{s}.\coqdoceol
\coqdocemptyline
\end{coqdoccode}

The definition of set union is slightly more involved, but straightforward. It relies on additional functions \coqdocdefinition{minus}, \coqdocdefinition{join} and \coqdocdefinition{meet} to perform respectively the set difference, intersection, and union of indexing functions. The various cases in the definition follow directly from the laws of boolean logic:

\begin{coqdoccode}
  \coqdocnoindent
  \coqdockw{Definition} \coqdef{NF2.Sets.cup}{cup}{\coqdocdefinition{cup}} \coqdocvar{s} \coqdocvar{s'} := \coqdockw{match} \coqdocvariable{s}, \coqdocvariable{s'} \coqdockw{with}\coqdoceol
\coqdocindent{1.00em}
\ensuremath{|} \coqref{NF2.Model.Low}{\coqdocconstructor{Low}} \coqdocvar{\_} \coqdocvar{f}, \coqref{NF2.Model.High}{\coqdocconstructor{High}} \coqdocvar{\_} \coqdocvar{g} \ensuremath{\Rightarrow} \coqref{NF2.Model.High}{\coqdocconstructor{High}} \coqdocvar{\_} (\coqref{NF2.Sets.minus}{\coqdocdefinition{minus}} \coqdocvar{g} \coqdocvar{f})\coqdoceol
\coqdocindent{1.00em}
\ensuremath{|} \coqref{NF2.Model.High}{\coqdocconstructor{High}} \coqdocvar{\_} \coqdocvar{f}, \coqref{NF2.Model.Low}{\coqdocconstructor{Low}} \coqdocvar{\_} \coqdocvar{g} \ensuremath{\Rightarrow} \coqref{NF2.Model.High}{\coqdocconstructor{High}} \coqdocvar{\_} (\coqref{NF2.Sets.minus}{\coqdocdefinition{minus}} \coqdocvar{f} \coqdocvar{g})\coqdoceol
\coqdocindent{1.00em}
\ensuremath{|} \coqref{NF2.Model.Low}{\coqdocconstructor{Low}} \coqdocvar{\_} \coqdocvar{f}, \coqref{NF2.Model.Low}{\coqdocconstructor{Low}} \coqdocvar{\_} \coqdocvar{g} \ensuremath{\Rightarrow} \coqref{NF2.Model.Low}{\coqdocconstructor{Low}} \coqdocvar{\_} (\coqref{NF2.Sets.join}{\coqdocdefinition{join}} \coqdocvar{f} \coqdocvar{g})\coqdoceol
\coqdocindent{1.00em}
\ensuremath{|} \coqref{NF2.Model.High}{\coqdocconstructor{High}} \coqdocvar{\_} \coqdocvar{f}, \coqref{NF2.Model.High}{\coqdocconstructor{High}} \coqdocvar{\_} \coqdocvar{g} \ensuremath{\Rightarrow} \coqref{NF2.Model.Low}{\coqdocconstructor{Low}} \coqdocvar{\_} (\coqref{NF2.Sets.meet}{\coqdocdefinition{meet}} \coqdocvar{f} \coqdocvar{g})\coqdoceol
\coqdocnoindent
\coqdockw{end}.\coqdoceol
\coqdocnoindent
\coqdockw{Notation} \coqdef{NF2.Sets.:::x 'xE2x88xAA' x}{"}{"}A ∪ B" := (\coqref{NF2.Sets.cup}{\coqdocdefinition{cup}} \coqdocvar{A} \coqdocvar{B}) (\coqdoctac{at} \coqdockw{level} 85).\coqdoceol
\coqdocemptyline
\end{coqdoccode}

As an example, we provide the definition of \coqdocdefinition{minus} (the others are similar):

\begin{coqdoccode}
  \coqdocnoindent
  \coqdockw{Definition} \coqdef{NF2.Sets.minus}{minus}{\coqdocdefinition{minus}} \{\coqdocvar{X} \coqdocvar{Y}\} \coqdocvar{f} \coqdocvar{g} \coqdoceol
  \coqdocindent{1.00em}
  : \coqexternalref{::type scope:'x7B' x ':' x 'x26' x 'x7D'}{http://coq.inria.fr/distrib/V8.11.0/stdlib//Coq.Init.Specif}{\coqdocnotation{\{}} \coqdocvar{x} \coqexternalref{::type scope:'x7B' x ':' x 'x26' x 'x7D'}{http://coq.inria.fr/distrib/V8.11.0/stdlib//Coq.Init.Specif}{\coqdocnotation{:}} \coqdocvariable{X} \coqexternalref{::type scope:'x7B' x ':' x 'x26' x 'x7D'}{http://coq.inria.fr/distrib/V8.11.0/stdlib//Coq.Init.Specif}{\coqdocnotation{\&}} \coqdockw{\ensuremath{\forall}} \coqdocvar{y} : \coqdocvariable{Y}, \coqexternalref{::type scope:'x7E' x}{http://coq.inria.fr/distrib/V8.11.0/stdlib//Coq.Init.Logic}{\coqdocnotation{\ensuremath{\lnot}}} \coqexternalref{::type scope:'x7E' x}{http://coq.inria.fr/distrib/V8.11.0/stdlib//Coq.Init.Logic}{\coqdocnotation{(}}\coqdocvariable{g} \coqdocvariable{y} \coqref{NF2.Model.:::x 'xE2x89xA1' x}{\coqdocnotation{≡}} \coqdocvariable{f} \coqdocvariable{x}\coqexternalref{::type scope:'x7E' x}{http://coq.inria.fr/distrib/V8.11.0/stdlib//Coq.Init.Logic}{\coqdocnotation{)}} \coqexternalref{::type scope:'x7B' x ':' x 'x26' x 'x7D'}{http://coq.inria.fr/distrib/V8.11.0/stdlib//Coq.Init.Specif}{\coqdocnotation{\}}} \coqexternalref{::type scope:x '->' x}{http://coq.inria.fr/distrib/V8.11.0/stdlib//Coq.Init.Logic}{\coqdocnotation{\ensuremath{\rightarrow}}} \coqref{NF2.Model.SET}{\coqdocinductive{SET}} \coqdoceol
\coqdocindent{1.00em}:=
\coqref{Internal.Misc.select}{\coqdocdefinition{select}} \coqdocvariable{f} (\coqdockw{fun} \coqdocvar{x} \ensuremath{\Rightarrow} \coqdockw{\ensuremath{\forall}} \coqdocvar{y}, \coqexternalref{::type scope:'x7E' x}{http://coq.inria.fr/distrib/V8.11.0/stdlib//Coq.Init.Logic}{\coqdocnotation{\ensuremath{\lnot}}} \coqexternalref{::type scope:'x7E' x}{http://coq.inria.fr/distrib/V8.11.0/stdlib//Coq.Init.Logic}{\coqdocnotation{(}}\coqdocvariable{g} \coqdocvariable{y} \coqref{NF2.Model.:::x 'xE2x89xA1' x}{\coqdocnotation{≡}} \coqdocvariable{f} \coqdocvariable{x}\coqexternalref{::type scope:'x7E' x}{http://coq.inria.fr/distrib/V8.11.0/stdlib//Coq.Init.Logic}{\coqdocnotation{)}}).\coqdoceol
\coqdocemptyline
\end{coqdoccode}

It requires an auxiliary function \coqdocdefinition{select} \var f \var P that restricts the domain of a function \var f according to a predicate \var P:

\begin{coqdoccode}
  \coqdocnoindent
  \coqdockw{Definition} \coqdef{Internal.Misc.select}{select}{\coqdocdefinition{select}} \{\coqdocvar{X} \coqdocvar{Y}\} (\coqdocvar{f}: \coqdocvariable{X} \coqexternalref{::type scope:x '->' x}{http://coq.inria.fr/distrib/V8.11.0/stdlib//Coq.Init.Logic}{\coqdocnotation{\ensuremath{\rightarrow}}} \coqdocvariable{Y}) (\coqdocvar{P}: \coqdocvariable{X} \coqexternalref{::type scope:x '->' x}{http://coq.inria.fr/distrib/V8.11.0/stdlib//Coq.Init.Logic}{\coqdocnotation{\ensuremath{\rightarrow}}} \coqdockw{Prop}) \coqdoceol
  \coqdocindent{1.00em} : \coqexternalref{::type scope:'x7B' x ':' x 'x26' x 'x7D'}{http://coq.inria.fr/distrib/V8.11.0/stdlib//Coq.Init.Specif}{\coqdocnotation{\{}}\coqdocvar{x}\coqexternalref{::type scope:'x7B' x ':' x 'x26' x 'x7D'}{http://coq.inria.fr/distrib/V8.11.0/stdlib//Coq.Init.Specif}{\coqdocnotation{:}} \coqdocvariable{X} \coqexternalref{::type scope:'x7B' x ':' x 'x26' x 'x7D'}{http://coq.inria.fr/distrib/V8.11.0/stdlib//Coq.Init.Specif}{\coqdocnotation{\&}} \coqdocvariable{P} \coqdocvariable{x}\coqexternalref{::type scope:'x7B' x ':' x 'x26' x 'x7D'}{http://coq.inria.fr/distrib/V8.11.0/stdlib//Coq.Init.Specif}{\coqdocnotation{\}}} \coqexternalref{::type scope:x '->' x}{http://coq.inria.fr/distrib/V8.11.0/stdlib//Coq.Init.Logic}{\coqdocnotation{\ensuremath{\rightarrow}}} \coqdocvariable{Y}\coqdoceol
  \coqdocindent{1.00em}
  := \coqdockw{fun} \coqdocvar{x} \ensuremath{\Rightarrow} \coqdocvariable{f} (\coqexternalref{projT1}{http://coq.inria.fr/distrib/V8.11.0/stdlib//Coq.Init.Specif}{\coqdocdefinition{projT1}} \coqdocvariable{x}).\coqdoceol
  \coqdocemptyline
\end{coqdoccode}

\begin{coqdoccode}
\coqdocnoindent
\coqdockw{Lemma} \coqdef{NF2.Sets.cup ok}{cup\_ok}{\coqdoclemma{cup\_ok}} : \coqdockw{\ensuremath{\forall}} \coqdocvar{s} \coqdocvar{s'} \coqdocvar{t}, \coqdocvariable{t} \coqref{NF2.Model.:::x 'xE2x88x88' x}{\coqdocnotation{∈}} \coqref{NF2.Model.:::x 'xE2x88x88' x}{\coqdocnotation{(}}\coqdocvariable{s} \coqref{NF2.Sets.:::x 'xE2x88xAA' x}{\coqdocnotation{∪}} \coqdocvariable{s'}\coqref{NF2.Model.:::x 'xE2x88x88' x}{\coqdocnotation{)}} \coqexternalref{::type scope:x '<->' x}{http://coq.inria.fr/distrib/V8.11.0/stdlib//Coq.Init.Logic}{\coqdocnotation{\ensuremath{\leftrightarrow}}} \coqexternalref{::type scope:x 'x5C/' x}{http://coq.inria.fr/distrib/V8.11.0/stdlib//Coq.Init.Logic}{\coqdocnotation{(}}\coqdocvariable{t} \coqref{NF2.Model.:::x 'xE2x88x88' x}{\coqdocnotation{∈}} \coqdocvariable{s}\coqexternalref{::type scope:x 'x5C/' x}{http://coq.inria.fr/distrib/V8.11.0/stdlib//Coq.Init.Logic}{\coqdocnotation{)}} \coqexternalref{::type scope:x 'x5C/' x}{http://coq.inria.fr/distrib/V8.11.0/stdlib//Coq.Init.Logic}{\coqdocnotation{\ensuremath{\lor}}} \coqexternalref{::type scope:x 'x5C/' x}{http://coq.inria.fr/distrib/V8.11.0/stdlib//Coq.Init.Logic}{\coqdocnotation{(}}\coqdocvariable{t} \coqref{NF2.Model.:::x 'xE2x88x88' x}{\coqdocnotation{∈}} \coqdocvariable{s'}\coqexternalref{::type scope:x 'x5C/' x}{http://coq.inria.fr/distrib/V8.11.0/stdlib//Coq.Init.Logic}{\coqdocnotation{)}}.\coqdoceol
\end{coqdoccode}

\subsection{Extensionality}
Proving extensionality in \NFTWO{} is slightly more involved than in \ZF{}. Like in \ZF{}, one direction of the double implication is easy and follows directly from \coqdocdefinition{in\_sound\_right}.

As for the other direction, \ie{} \\
\centerline{
  (\coqdockw{\ensuremath{\forall}} \coqdocvar{t}, \coqdocvariable{t} \INX \coqdocvariable{s} \coqexternalref{::type scope:x '<->' x}{http://coq.inria.fr/distrib/V8.11.0/stdlib//Coq.Init.Logic}{\coqdocnotation{\ensuremath{\leftrightarrow}}} \coqdocvariable{t} \INX \coqdocvariable{s'}) %
  \coqexternalref{::type scope:x '<->' x}{http://coq.inria.fr/distrib/V8.11.0/stdlib//Coq.Init.Logic}{\coqdocnotation{\ensuremath{\to}}} %
  \coqdocvariable{s} \EQX \coqdocvariable{s'}
}

\noindent
we proceed by cases over \var s and \var{s'}. When their constructor matches (cases \coqdocconstructor{Low}/\coqdocconstructor{Low} and \coqdocconstructor{High}/\coqdocconstructor{High}) the proof proceeds just like in \ZF{}. The new cases \coqdocconstructor{Low} \emph{vs} \coqdocconstructor{High} require some more work. In these cases, \coqdocvariable{s} \EQX \coqdocvariable{s'} does not hold because low sets and high sets are not equivalent by definition of \coqdocdefinition{EQ}. Therefore to conclude it suffices to assume \coqdockw{\ensuremath{\forall}} \coqdocvar{t}, \coqdocvariable{t} \INX \coqdocvariable{s} \coqexternalref{::type scope:x '<->' x}{http://coq.inria.fr/distrib/V8.11.0/stdlib//Coq.Init.Logic}{\coqdocnotation{\ensuremath{\leftrightarrow}}} \coqdocvariable{t} \INX \coqdocvariable{s'} and derive a contradiction.

Without loss of generality, let us say that \var s = \coqref{NF2.Model.Low}{\coqdocconstructor{Low}} \coqdocvar{X} \coqdocvar{f} and \var {s'} = \coqref{NF2.Model.High}{\coqdocconstructor{High}} \coqdocvar{Y} \coqdocvar{g} (the other case is specular). Basically, what we have to prove is that a low set and a high set cannot have the same extension (upcoming lemma \coqref{NF2.Ext.pos neg ext neq}{\coqdoclemma{pos\_neg\_ext\_neq}}), which is justified by our initial intuition that low sets are ``small'', while high sets correspond to proper classes.

\medskip

The crucial intuition is that when a low set has the same extension as a high set, it is actually possible to construct a universal set that is low (contrarily to the universal set defined in \Cref{ex:EUNF2}, which is correctly high). In fact, the images of \var f and \var g are complementary, and it suffices to take their disjoin union:

\begin{coqdoccode}
  \coqdocnoindent
  \coqdockw{Lemma} \coqdefRef{NF2.Ext.pos univ}{pos\_univ}{\coqdoclemma{pos\_univ}}: \coqdockw{\ensuremath{\forall}} \coqdocvar{X} \coqdocvar{f} \coqdocvar{Y} \coqdocvar{g},\coqdoceol
\coqdocindent{1.00em}
\coqexternalref{::type scope:x '->' x}{http://coq.inria.fr/distrib/V8.11.0/stdlib//Coq.Init.Logic}{\coqdocnotation{(}}\coqdockw{\ensuremath{\forall}} \coqdocvar{s}, \coqdocvariable{s} \INX \coqref{NF2.Model.Low}{\coqdocconstructor{Low}} \coqdocvariable{X} \coqdocvariable{f} \coqexternalref{::type scope:x '<->' x}{http://coq.inria.fr/distrib/V8.11.0/stdlib//Coq.Init.Logic}{\coqdocnotation{\ensuremath{\leftrightarrow}}} \coqdocvariable{s} \INX \coqref{NF2.Model.High}{\coqdocconstructor{High}} \coqdocvariable{Y} \coqdocvariable{g}\coqexternalref{::type scope:x '->' x}{http://coq.inria.fr/distrib/V8.11.0/stdlib//Coq.Init.Logic}{\coqdocnotation{)}}\coqdoceol
\coqdocindent{1.00em}
\coqexternalref{::type scope:x '->' x}{http://coq.inria.fr/distrib/V8.11.0/stdlib//Coq.Init.Logic}{\coqdocnotation{\ensuremath{\rightarrow}}} \coqdockw{\ensuremath{\forall}} \coqdocvar{s}, \coqdocvariable{s} \INX \coqref{NF2.Model.Low}{\coqdocconstructor{Low}} (\coqdocvariable{X} \coqexternalref{::type scope:x '+' x}{http://coq.inria.fr/distrib/V8.11.0/stdlib//Coq.Init.Datatypes}{\coqdocnotation{+}} \coqdocvariable{Y}) (\coqdocvariable{f} \coqref{Internal.Misc.:::x 'xE2xA8x81' x}{\coqdocnotation{⨁}} \coqdocvariable{g}).\coqdoceol
\end{coqdoccode}
\begin{proof}
  Note: classical logic is required to prove this lemma.
  Assume $\coqdocvariable{s} \colon \coqdocinductive{SET}$. To show that \coqdocvariable{s} \INX \coqref{NF2.Model.Low}{\coqdocconstructor{Low}} (\coqdocvariable{X} \coqexternalref{::type scope:x '+' x}{http://coq.inria.fr/distrib/V8.11.0/stdlib//Coq.Init.Datatypes}{\coqdocnotation{+}} \coqdocvariable{Y}) (\coqdocvariable{f} \coqref{Internal.Misc.:::x 'xE2xA8x81' x}{\coqdocnotation{⨁}} \coqdocvariable{g}), we need to supply an element of type \coqdocvariable{X} \coqexternalref{::type scope:x '+' x}{http://coq.inria.fr/distrib/V8.11.0/stdlib//Coq.Init.Datatypes}{\coqdocnotation{+}} \coqdocvariable{Y}. Choosing the left or right injection into \coqdocvariable{X} \coqexternalref{::type scope:x '+' x}{http://coq.inria.fr/distrib/V8.11.0/stdlib//Coq.Init.Datatypes}{\coqdocnotation{+}} \coqdocvariable{Y} is equivalent to deciding whether \coqdocvariable{s} \INX \coqref{NF2.Model.Low}{\coqdocconstructor{Low}} \coqdocvariable{X} \coqdocvariable{f} holds.
  % \TODO{}
\end{proof}

To conclude, we just need to show that a universal set cannot be low. We resort to a form of ``regularity'' of low sets. Recall that regularity is an axiom of \ZF{} set theory which fundamentally forces sets to be well-founded. Clearly regularity does not hold in \NFTWO{}, since set membership is clearly not well-founded --- for instance, $\Universe \in \Universe \in \ldots$. However, a weaker form of regularity holds for low sets, namely that a low set cannot contain itself:

\begin{coqdoccode}
  \coqdocnoindent
\coqdockw{Theorem} \coqdefRef{NF2.Ext.weak regularity}{weak\_regularity}{\coqdoclemma{weak\_regularity}}: \coqdockw{\ensuremath{\forall}} \coqdocvar{s}, \coqref{NF2.Model.low}{\coqdocdefinition{low}} \coqdocvariable{s} \coqexternalref{::type scope:x '->' x}{http://coq.inria.fr/distrib/V8.11.0/stdlib//Coq.Init.Logic}{\coqdocnotation{\ensuremath{\rightarrow}}} \coqdocvariable{s} \INX \coqdocvariable{s} \coqexternalref{::type scope:x '->' x}{http://coq.inria.fr/distrib/V8.11.0/stdlib//Coq.Init.Logic}{\coqdocnotation{\ensuremath{\rightarrow}}} \coqexternalref{False}{http://coq.inria.fr/distrib/V8.11.0/stdlib//Coq.Init.Logic}{\coqdocinductive{False}}.\coqdoceol
\end{coqdoccode}
\begin{proof}
  By structural induction on \coqdocvariable{s}, low by hypothesis. Let \coqdocvariable{s} = \coqref{NF2.Model.Low}{\coqdocconstructor{Low}} \coqdocvariable{X} \coqdocvariable{f}. From \coqdocvariable{s} \INX \coqdocvariable{s} it follows that there exists \coqdocvariable{x} such that \coqdocvariable{s} \EQX \coqdocvariable{f}~\coqdocvariable{x}. Note that \coqdocvariable{f}~\coqdocvariable{x} must be a low set by the definition of \coqdocdefinition{EQ}. Since \coqdocdefinition{IN} is a \coqdocdefinition{EQ}-morphism, it follows that \coqdocvariable{f}~\coqdocvariable{x} \INX \coqdocvariable{f}~\coqdocvariable{x}. We conclude by the inductive hypothesis.
\end{proof}

It directly follows:

\begin{coqdoccode}
  \coqdocnoindent
\coqdockw{Lemma} \coqdefRef{NF2.Ext.pos neg ext neq}{pos\_neg\_ext\_neq}{\coqdoclemma{pos\_neg\_ext\_neq}}: \coqdockw{\ensuremath{\forall}} \coqdocvar{X} \coqdocvar{f} \coqdocvar{Y} \coqdocvar{g},\coqdoceol
\coqdocindent{1.00em}
\coqexternalref{::type scope:x '->' x}{http://coq.inria.fr/distrib/V8.11.0/stdlib//Coq.Init.Logic}{\coqdocnotation{(}}\coqdockw{\ensuremath{\forall}} \coqdocvar{s}, \coqdocvariable{s} \INX \coqref{NF2.Model.Low}{\coqdocconstructor{Low}} \coqdocvariable{X} \coqdocvariable{f} \coqexternalref{::type scope:x '<->' x}{http://coq.inria.fr/distrib/V8.11.0/stdlib//Coq.Init.Logic}{\coqdocnotation{\ensuremath{\leftrightarrow}}} \coqdocvariable{s} \INX \coqref{NF2.Model.High}{\coqdocconstructor{High}} \coqdocvariable{Y} \coqdocvariable{g} \coqexternalref{::type scope:x '->' x}{http://coq.inria.fr/distrib/V8.11.0/stdlib//Coq.Init.Logic}{\coqdocnotation{)}} \coqexternalref{::type scope:x '->' x}{http://coq.inria.fr/distrib/V8.11.0/stdlib//Coq.Init.Logic}{\coqdocnotation{\ensuremath{\rightarrow}}} \coqexternalref{False}{http://coq.inria.fr/distrib/V8.11.0/stdlib//Coq.Init.Logic}{\coqdocinductive{False}}.\coqdoceol
\end{coqdoccode}
\begin{proof}
  By \coqref{NF2.Ext.pos univ}{\coqdoclemma{pos\_univ}} and \coqref{NF2.Ext.weak regularity}{\coqdoclemma{weak\_regularity}}.
\end{proof}

We are now ready to prove extensionality for \NFTWO:

\begin{coqdoccode}
  \coqdocnoindent
\coqdockw{Theorem} \coqdefRef{NF2.Ext.ext}{ext}{\coqdoclemma{ext}}: \coqdockw{\ensuremath{\forall}} \coqdocvar{s} \coqdocvar{s'}, \coqdocvariable{s} \EQX \coqdocvariable{s'} \coqexternalref{::type scope:x '<->' x}{http://coq.inria.fr/distrib/V8.11.0/stdlib//Coq.Init.Logic}{\coqdocnotation{\ensuremath{\leftrightarrow}}} \coqdockw{\ensuremath{\forall}} \coqdocvar{t}, \coqdocvariable{t} \INX \coqdocvariable{s} \coqexternalref{::type scope:x '<->' x}{http://coq.inria.fr/distrib/V8.11.0/stdlib//Coq.Init.Logic}{\coqdocnotation{\ensuremath{\leftrightarrow}}} \coqdocvariable{t} \INX \coqdocvariable{s'}.\coqdoceol
\end{coqdoccode}
\begin{proof}
  The direction $(\Rightarrow)$ of the double implication follows from \coqdocdefinition{IN} being an \coqdocdefinition{EQ}-morphism. As for the other direction, we proceed by cases on \coqdocvariable{s} and \coqdocvariable{s'}:
  \begin{itemize}
    \item If \coqdocvariable{s} and \coqdocvariable{s'} are both low sets or both high sets, then the proof carries on exactly as in \ZF.
    \item The cases when one is a low set and the other one is a high set are ruled out by lemma \coqref{NF2.Ext.pos neg ext neq}{\coqdoclemma{pos\_neg\_ext\_neq}}.
    \qedhere
  \end{itemize}
\end{proof}


\section{\NFO}
%!TEX root = main.tex

\renewcommand\INX{\coqref{NFO.In.IN}{{\IN}}~}

Normal form: $A \veebar B$, where $A$ is the Aczel-part and $B$ is the essence-part. More precisely, $B$ is a boolean expression having as atoms the essence things of the form $\Essence\Placeholder$

\subsection{Boolean expressions}

\begin{coqdoccode}
\coqdocemptyline
\end{coqdoccode}
We use the inductive type @\coqref{NFO.BoolExpr.BExpr}{\coqdocinductive{BExpr}} \coqdocvar{X} to encode a boolean expression whose atoms have type \coqdocvar{X}:

\begin{coqdoccode}
\coqdocnoindent
\coqdockw{Inductive} \coqdefRef{NFO.BoolExpr.BExpr}{BExpr}{\coqdocinductive{BExpr}} \{\coqdocvar{X}\} :=\coqdoceol
\coqdocindent{1.00em}
\ensuremath{|} \coqdefRef{NFO.BoolExpr.Bot}{Bot}{\coqdocconstructor{Bot}} : \coqref{NFO.BoolExpr.BExpr}{\coqdocinductive{BExpr}}\coqdoceol
\coqdocindent{1.00em}
\ensuremath{|} \coqdefRef{NFO.BoolExpr.Atom}{Atom}{\coqdocconstructor{Atom}} : \coqdocvariable{X} \coqexternalref{::type scope:x '->' x}{http://coq.inria.fr/distrib/V8.11.0/stdlib//Coq.Init.Logic}{\coqdocnotation{\ensuremath{\rightarrow}}} \coqref{NFO.BoolExpr.BExpr}{\coqdocinductive{BExpr}}\coqdoceol
\coqdocindent{1.00em}
\ensuremath{|} \coqdefRef{NFO.BoolExpr.Not}{Not}{\coqdocconstructor{Not}} : \coqref{NFO.BoolExpr.BExpr}{\coqdocinductive{BExpr}} \coqexternalref{::type scope:x '->' x}{http://coq.inria.fr/distrib/V8.11.0/stdlib//Coq.Init.Logic}{\coqdocnotation{\ensuremath{\rightarrow}}} \coqref{NFO.BoolExpr.BExpr}{\coqdocinductive{BExpr}}\coqdoceol
\coqdocindent{1.00em}
\ensuremath{|} \coqdefRef{NFO.BoolExpr.Or}{Or}{\coqdocconstructor{Or}} : \coqref{NFO.BoolExpr.BExpr}{\coqdocinductive{BExpr}} \coqexternalref{::type scope:x '->' x}{http://coq.inria.fr/distrib/V8.11.0/stdlib//Coq.Init.Logic}{\coqdocnotation{\ensuremath{\rightarrow}}} \coqref{NFO.BoolExpr.BExpr}{\coqdocinductive{BExpr}} \coqexternalref{::type scope:x '->' x}{http://coq.inria.fr/distrib/V8.11.0/stdlib//Coq.Init.Logic}{\coqdocnotation{\ensuremath{\rightarrow}}} \coqref{NFO.BoolExpr.BExpr}{\coqdocinductive{BExpr}}\coqdoceol
\coqdocnoindent
.\coqdoceol
\coqdocemptyline
\end{coqdoccode}

\coqref{NFO.BoolExpr.Bot}{\coqdocconstructor{Bot}} represents the constant false expression, \coqref{NFO.BoolExpr.Or}{\coqdocconstructor{Or}} and \coqref{NFO.BoolExpr.Not}{\coqdocconstructor{Not}} the usual boolean operators, \dots We do not ??? the other logical connectives, like conjunction or implication, because disjunction and complement are functionally complete for classical logic.

Map the atoms in a boolean expression:

\begin{coqdoccode}
\coqdocnoindent
\coqdockw{Fixpoint} \coqdefRef{NFO.BoolExpr.map}{map}{\coqdocdefinition{map}} \{\coqdocvar{X} \coqdocvar{Y}\} (\coqdocvar{f}: \coqdocvariable{X} \coqexternalref{::type scope:x '->' x}{http://coq.inria.fr/distrib/V8.11.0/stdlib//Coq.Init.Logic}{\coqdocnotation{\ensuremath{\rightarrow}}} \coqdocvariable{Y}) (\coqdocvar{e}: \coqref{NFO.BoolExpr.BExpr}{\coqdocinductive{BExpr}}) : \coqref{NFO.BoolExpr.BExpr}{\coqdocinductive{BExpr}} :=\coqdoceol
\coqdocnoindent
\coqdockw{match} \coqdocvariable{e} \coqdockw{with}\coqdoceol
\coqdocindent{1.00em}
\ensuremath{|} \coqref{NFO.BoolExpr.Atom}{\coqdocconstructor{Atom}} \coqdocvar{a} \ensuremath{\Rightarrow} \coqref{NFO.BoolExpr.Atom}{\coqdocconstructor{Atom}} (\coqdocvariable{f} \coqdocvar{a})\coqdoceol
\coqdocindent{1.00em}
\ensuremath{|} \coqref{NFO.BoolExpr.Bot}{\coqdocconstructor{Bot}} \ensuremath{\Rightarrow} \coqref{NFO.BoolExpr.Bot}{\coqdocconstructor{Bot}}\coqdoceol
\coqdocindent{1.00em}
\ensuremath{|} \coqref{NFO.BoolExpr.Not}{\coqdocconstructor{Not}} \coqdocvar{e} \ensuremath{\Rightarrow} \coqref{NFO.BoolExpr.Not}{\coqdocconstructor{Not}} (\coqref{NFO.BoolExpr.map}{\coqdocdefinition{map}} \coqdocvariable{f} \coqdocvariable{e})\coqdoceol
\coqdocindent{1.00em}
\ensuremath{|} \coqref{NFO.BoolExpr.Or}{\coqdocconstructor{Or}} \coqdocvar{e} \coqdocvar{e'} \ensuremath{\Rightarrow} \coqref{NFO.BoolExpr.Or}{\coqdocconstructor{Or}} (\coqref{NFO.BoolExpr.map}{\coqdocdefinition{map}} \coqdocvariable{f} \coqdocvariable{e}) (\coqref{NFO.BoolExpr.map}{\coqdocdefinition{map}} \coqdocvariable{f} \coqdocvar{e'})\coqdoceol
\coqdocnoindent
\coqdockw{end}.\coqdoceol
\coqdocemptyline
\end{coqdoccode}

Evaluate a boolean expression to a \coqdockw{Prop} 
when the atoms are of type \coqdockw{Prop}:

\begin{coqdoccode}
\coqdocnoindent
\coqdockw{Fixpoint} \coqdefRef{NFO.BoolExpr.eval}{eval}{\coqdocdefinition{eval}} (\coqdocvar{e}: \coqref{NFO.BoolExpr.BExpr}{\coqdocinductive{BExpr}}) := \coqdockw{match} \coqdocvariable{e} \coqdockw{with}\coqdoceol
\coqdocindent{1.00em}
\ensuremath{|} \coqref{NFO.BoolExpr.Atom}{\coqdocconstructor{Atom}} \coqdocvar{p} \ensuremath{\Rightarrow} \coqdocvar{p}\coqdoceol
\coqdocindent{1.00em}
\ensuremath{|} \coqref{NFO.BoolExpr.Bot}{\coqdocconstructor{Bot}} \ensuremath{\Rightarrow} \coqexternalref{False}{http://coq.inria.fr/distrib/V8.11.0/stdlib//Coq.Init.Logic}{\coqdocinductive{False}}\coqdoceol
\coqdocindent{1.00em}
\ensuremath{|} \coqref{NFO.BoolExpr.Not}{\coqdocconstructor{Not}} \coqdocvar{e} \ensuremath{\Rightarrow} \coqexternalref{::type scope:'x7E' x}{http://coq.inria.fr/distrib/V8.11.0/stdlib//Coq.Init.Logic}{\coqdocnotation{\ensuremath{\lnot}}} \coqref{NFO.BoolExpr.eval}{\coqdocdefinition{eval}} \coqdocvariable{e}\coqdoceol
\coqdocindent{1.00em}
\ensuremath{|} \coqref{NFO.BoolExpr.Or}{\coqdocconstructor{Or}} \coqdocvar{e} \coqdocvar{e'} \ensuremath{\Rightarrow} \coqref{NFO.BoolExpr.eval}{\coqdocdefinition{eval}} \coqdocvariable{e} \coqexternalref{::type scope:x 'x5C/' x}{http://coq.inria.fr/distrib/V8.11.0/stdlib//Coq.Init.Logic}{\coqdocnotation{\ensuremath{\lor}}} \coqref{NFO.BoolExpr.eval}{\coqdocdefinition{eval}} \coqdocvar{e'}\coqdoceol
\coqdocnoindent
\coqdockw{end}.\coqdoceol
\coqdocnoindent
\coqdockw{Notation} \coqdefRef{NFO.BoolExpr.:::'xE2x9FxA6' x 'xE2x9FxA7'}{"}{"}⟦ e ⟧" := (\coqref{NFO.BoolExpr.eval}{\coqdocdefinition{eval}} \coqdocvar{e}).\coqdoceol
\coqdocemptyline
\end{coqdoccode}


\subsubsection{Semantics}

As usual, two boolean expressions (with atoms of type \coqdocvariable{X}) are equivalent if and only if 
they evaluate to equivalent values with respect to every
truth assignment \coqdocvar{P}: \coqdocvariable{X} \coqdocnotation{\ensuremath{\rightarrow}} \coqdockw{Prop}. However, not every assignment is allowed here.
In our case, the set of atoms \coqdocvariable{X} is \TODO{ultimately meant to be SET, which is } equipped with a setoid structure (\coqdocvariable{SET}, \coqdocvar{EQ}), hence $\coqdocvar{P}$ must assign the same truth value to atoms that are equivalent according to $\coqdocvar{R}$. In general, $\coqdocvar{P}$ should be a $(\coqdocvariable{X}, \coqdocvar{R})$-morphism, \ie{} it should respect a binary relation $\coqdocvar{R}$.  \dots

As a consequence, we parametrize the $\coqdocdefinition{eq\_bexpr}$ on a binary relation $\coqdocvar{R}$ over $\coqdocvariable{X}$, and only quantify over assignments $\coqdocvar{P}$ such that $\coqref{Internal.FunExt.respects}{\coqdocdefinition{respects}}~\coqdocvariable{R}~\coqdocvariable{P}$\, holds:
    
\begin{coqdoccode}
\coqdocnoindent
\coqdockw{Definition} \coqdefRef{NFO.BoolExpr.eq\_bexpr}{eq\_bexpr}{\coqdocdefinition{eq\_bexpr}} \{\coqdocvar{X}\} (\coqdocvar{R}: \coqdocvariable{X} \coqexternalref{::type scope:x '->' x}{http://coq.inria.fr/distrib/V8.11.0/stdlib//Coq.Init.Logic}{\coqdocnotation{\ensuremath{\rightarrow}}} \coqdocvariable{X} \coqexternalref{::type scope:x '->' x}{http://coq.inria.fr/distrib/V8.11.0/stdlib//Coq.Init.Logic}{\coqdocnotation{\ensuremath{\rightarrow}}} \coqdockw{Prop}) \coqdocvar{e} \coqdocvar{e'} :=\coqdoceol
\coqdocindent{1.00em}
\coqdockw{\ensuremath{\forall}} \coqdocvar{P}, \coqref{Internal.FunExt.respects}{\coqdocdefinition{respects}} \coqdocvariable{R} \coqdocvariable{P} \coqexternalref{::type scope:x '->' x}{http://coq.inria.fr/distrib/V8.11.0/stdlib//Coq.Init.Logic}{\coqdocnotation{\ensuremath{\rightarrow}}} \coqref{NFO.BoolExpr.:::'xE2x9FxA6' x 'xE2x9FxA7'}{\coqdocnotation{⟦}}\coqref{NFO.BoolExpr.map}{\coqdocdefinition{map}} \coqdocvariable{P} \coqdocvariable{e}\coqref{NFO.BoolExpr.:::'xE2x9FxA6' x 'xE2x9FxA7'}{\coqdocnotation{⟧}} \coqexternalref{::type scope:x '<->' x}{http://coq.inria.fr/distrib/V8.11.0/stdlib//Coq.Init.Logic}{\coqdocnotation{\ensuremath{\leftrightarrow}}} \coqref{NFO.BoolExpr.:::'xE2x9FxA6' x 'xE2x9FxA7'}{\coqdocnotation{⟦}}\coqref{NFO.BoolExpr.map}{\coqdocdefinition{map}} \coqdocvariable{P} \coqdocvariable{e'}\coqref{NFO.BoolExpr.:::'xE2x9FxA6' x 'xE2x9FxA7'}{\coqdocnotation{⟧}}.\coqdoceol
\coqdocemptyline
\end{coqdoccode}

\subsection{The Type \texorpdfstring{\coqdocinductive{SET}}{SET}}

\begin{coqdoccode}
  \coqdocnoindent
\coqdockw{Inductive} \coqdefRef{NFO.Model.SET}{SET}{\coqdocinductive{SET}} :=\coqdoceol
\coqdocindent{1.00em}
\coqdef{NFO.Model.S}{S}{\coqdocconstructor{S}} : \coqdockw{\ensuremath{\forall}} \coqdocvar{X} \coqdocvar{Y} (\coqdocvar{f}: \coqdocvariable{X} \coqexternalref{::type scope:x '->' x}{http://coq.inria.fr/distrib/V8.11.0/stdlib//Coq.Init.Logic}{\coqdocnotation{\ensuremath{\rightarrow}}} \coqref{NFO.Model.SET}{\coqdocinductive{SET}}) (\coqdocvar{g}: \coqdocvariable{Y} \coqexternalref{::type scope:x '->' x}{http://coq.inria.fr/distrib/V8.11.0/stdlib//Coq.Init.Logic}{\coqdocnotation{\ensuremath{\rightarrow}}} \coqref{NFO.Model.SET}{\coqdocinductive{SET}}) (\coqdocvar{e}: @\coqref{NFO.BoolExpr.BExpr}{\coqdocinductive{BExpr}} \coqdocvariable{Y}), \coqref{NFO.Model.SET}{\coqdocinductive{SET}}.\coqdoceol
\coqdocemptyline
\coqdocnoindent
\end{coqdoccode}

\TODO{EXAMPLES + INJECTION}

\begin{tabular}{ccl}
  \coqdef{NF2.Model.High}{High}{\coqdocconstructor{High}} \coqdocvariable{X} \coqdocvariable{f} & $\hookrightarrow$ & \coqref{NFO.Model.S}{\coqdocconstructor{S}} \coqdocvariable{X} ${\_}_1$ \coqdocvariable{f} ${\_}_2$ \coqref{NFO.BoolExpr.Bot}{\coqdocconstructor{Bot}}
  \\
  \coqdef{NF2.Model.Low}{Low}{\coqdocconstructor{Low}} \coqdocvariable{X} \coqdocvariable{f} & $\hookrightarrow$ & \coqref{NFO.Model.S}{\coqdocconstructor{S}} \coqdocvariable{X} ${\_}_1$ \coqdocvariable{f} ${\_}_2$ (\coqref{NFO.BoolExpr.Not}{\coqdocconstructor{Not}} \coqref{NFO.BoolExpr.Bot}{\coqdocconstructor{Bot}})
\end{tabular}

where ${\_}_1$ and ${\_}_2$ can be anything (for simplicity, one can take respectively \coqexternalref{False}{http://coq.inria.fr/distrib/V8.11.0/stdlib//Coq.Init.Logic}{\coqdocinductive{False}} and \coqexternalref{False rect}{http://coq.inria.fr/distrib/V8.11.0/stdlib//Coq.Init.Logic}{\coqdocdefinition{False\_rect}} \coqdocvar{\_}) .

\subsection{Set Equality}

XXX

\begin{coqdoccode}
\coqdocnoindent
\sout{\coqdockw{Definition}} \coqdoclemma{EQ} \coqdocvar{s} \coqdocvar{s'} := \coqdockw{match} \coqdocvariable{s}, \coqdocvariable{s'} \coqdockw{with}\coqdoceol
\coqdocindent{2.00em}
\ensuremath{|} \coqexternalref{::core scope:'(' x ',' x ',' '..' ',' x ')'}{http://coq.inria.fr/distrib/V8.11.0/stdlib//Coq.Init.Datatypes}{\coqdocnotation{(}}\coqref{NFO.Model.S}{\coqdocconstructor{S}} \coqdocvar{X} \coqdocvar{Y} \coqdocvar{f} \coqdocvar{g} \coqdocvar{e}\coqexternalref{::core scope:'(' x ',' x ',' '..' ',' x ')'}{http://coq.inria.fr/distrib/V8.11.0/stdlib//Coq.Init.Datatypes}{\coqdocnotation{,}} \coqref{NFO.Model.S}{\coqdocconstructor{S}} \coqdocvar{X'} \coqdocvar{Y'} \coqdocvar{f'} \coqdocvar{g'} \coqdocvar{e'}\coqexternalref{::core scope:'(' x ',' x ',' '..' ',' x ')'}{http://coq.inria.fr/distrib/V8.11.0/stdlib//Coq.Init.Datatypes}{\coqdocnotation{)}} \ensuremath{\Rightarrow} \coqdoceol

\coqdocindent{4.00em} \coqdocvar{f} $\AEQ$ \coqdocvar{f'} % \coqdoceol
% 
% \coqdocindent{4.00em}
 \coqexternalref{::type scope:x '/x5C' x}{http://coq.inria.fr/distrib/V8.11.0/stdlib//Coq.Init.Logic}{\coqdocnotation{\ensuremath{\land}}} \coqref{NFO.BoolExpr.eq bexpr}{\coqdocdefinition{eq\_bexpr}} \coqref{NFO.Eq.EQ}{\coqdocdefinition{EQ}} (\coqref{NFO.BoolExpr.map}{\coqdocdefinition{map}} \coqdocvariable{g} \coqdocvariable{e}) (\coqref{NFO.BoolExpr.map}{\coqdocdefinition{map}} \coqdocvariable{g'} \coqdocvariable{e'}).\coqdoceol
\coqdocnoindent
\end{coqdoccode}

XXX

\begin{coqdoccode}
\coqdocnoindent
\coqdockw{Local} \coqdockw{Definition} \coqdefRef{NFO.Eq.EQ'}{EQ'}{\coqdocdefinition{EQ'}} : \coqref{NFO.Model.SET}{\coqdocinductive{SET}} \coqexternalref{::type scope:x '*' x}{http://coq.inria.fr/distrib/V8.11.0/stdlib//Coq.Init.Datatypes}{\coqdocnotation{\ensuremath{\times}}} \coqref{NFO.Model.SET}{\coqdocinductive{SET}} \coqexternalref{::type scope:x '->' x}{http://coq.inria.fr/distrib/V8.11.0/stdlib//Coq.Init.Logic}{\coqdocnotation{\ensuremath{\rightarrow}}} \coqdockw{Prop}.\coqdoceol
\coqdocnoindent
\coqdoctac{refine} ( \coqexternalref{Fix}{http://coq.inria.fr/distrib/V8.11.0/stdlib//Coq.Init.Wf}{\coqdocdefinition{Fix}} (\coqref{Internal.WfTuples.wf two}{\coqdoclemma{wf\_two}} \coqref{NFO.Model.wf lt}{\coqdoclemma{wf\_lt}}) (\coqdockw{fun} \coqdocvar{\_} \ensuremath{\Rightarrow} \coqdockw{Prop}) (\coqdoceol
\coqdocindent{1.00em}
\coqdockw{fun} \coqdocvar{p} \coqdocvar{rec} \ensuremath{\Rightarrow} (\coqdoceol
\coqdocindent{2.00em}
\coqdockw{match} \coqdocvariable{p} \coqdockw{as} \coqdocvar{p0} \coqdockw{return} (\coqdocvariable{p} \coqexternalref{::type scope:x '=' x}{http://coq.inria.fr/distrib/V8.11.0/stdlib//Coq.Init.Logic}{\coqdocnotation{=}} \coqdocvariable{p0} \coqexternalref{::type scope:x '->' x}{http://coq.inria.fr/distrib/V8.11.0/stdlib//Coq.Init.Logic}{\coqdocnotation{\ensuremath{\rightarrow}}} \coqdockw{Prop}) \coqdockw{with}\coqdoceol
\coqdocindent{2.00em}
\ensuremath{|} \coqexternalref{::core scope:'(' x ',' x ',' '..' ',' x ')'}{http://coq.inria.fr/distrib/V8.11.0/stdlib//Coq.Init.Datatypes}{\coqdocnotation{(}}\coqref{NFO.Model.S}{\coqdocconstructor{S}} \coqdocvar{X} \coqdocvar{Y} \coqdocvar{f} \coqdocvar{g} \coqdocvar{e}\coqexternalref{::core scope:'(' x ',' x ',' '..' ',' x ')'}{http://coq.inria.fr/distrib/V8.11.0/stdlib//Coq.Init.Datatypes}{\coqdocnotation{,}} \coqref{NFO.Model.S}{\coqdocconstructor{S}} \coqdocvar{X'} \coqdocvar{Y'} \coqdocvar{f'} \coqdocvar{g'} \coqdocvar{e'}\coqexternalref{::core scope:'(' x ',' x ',' '..' ',' x ')'}{http://coq.inria.fr/distrib/V8.11.0/stdlib//Coq.Init.Datatypes}{\coqdocnotation{)}} \ensuremath{\Rightarrow} \coqdockw{fun} \coqdocvar{eqx} \ensuremath{\Rightarrow}\coqdoceol
\coqdocindent{4.00em}
\coqexternalref{::type scope:x '/x5C' x}{http://coq.inria.fr/distrib/V8.11.0/stdlib//Coq.Init.Logic}{\coqdocnotation{((}}\coqdockw{\ensuremath{\forall}} \coqdocvar{x}, \coqexternalref{::type scope:'exists' x '..' x ',' x}{http://coq.inria.fr/distrib/V8.11.0/stdlib//Coq.Init.Logic}{\coqdocnotation{\ensuremath{\exists}}} \coqdocvar{x'}\coqexternalref{::type scope:'exists' x '..' x ',' x}{http://coq.inria.fr/distrib/V8.11.0/stdlib//Coq.Init.Logic}{\coqdocnotation{,}} \coqdocvariable{rec} \coqexternalref{::core scope:'(' x ',' x ',' '..' ',' x ')'}{http://coq.inria.fr/distrib/V8.11.0/stdlib//Coq.Init.Datatypes}{\coqdocnotation{(}}\coqdocvar{f} \coqdocvariable{x}\coqexternalref{::core scope:'(' x ',' x ',' '..' ',' x ')'}{http://coq.inria.fr/distrib/V8.11.0/stdlib//Coq.Init.Datatypes}{\coqdocnotation{,}} \coqdocvar{f'} \coqdocvariable{x'}\coqexternalref{::core scope:'(' x ',' x ',' '..' ',' x ')'}{http://coq.inria.fr/distrib/V8.11.0/stdlib//Coq.Init.Datatypes}{\coqdocnotation{)}} \coqdocvar{\_}\coqexternalref{::type scope:x '/x5C' x}{http://coq.inria.fr/distrib/V8.11.0/stdlib//Coq.Init.Logic}{\coqdocnotation{)}}\coqdoceol
\coqdocindent{4.00em}
\coqexternalref{::type scope:x '/x5C' x}{http://coq.inria.fr/distrib/V8.11.0/stdlib//Coq.Init.Logic}{\coqdocnotation{\ensuremath{\land}}} \coqexternalref{::type scope:x '/x5C' x}{http://coq.inria.fr/distrib/V8.11.0/stdlib//Coq.Init.Logic}{\coqdocnotation{(}}\coqdockw{\ensuremath{\forall}} \coqdocvar{x'}, \coqexternalref{::type scope:'exists' x '..' x ',' x}{http://coq.inria.fr/distrib/V8.11.0/stdlib//Coq.Init.Logic}{\coqdocnotation{\ensuremath{\exists}}} \coqdocvar{x}\coqexternalref{::type scope:'exists' x '..' x ',' x}{http://coq.inria.fr/distrib/V8.11.0/stdlib//Coq.Init.Logic}{\coqdocnotation{,}} \coqdocvariable{rec} \coqexternalref{::core scope:'(' x ',' x ',' '..' ',' x ')'}{http://coq.inria.fr/distrib/V8.11.0/stdlib//Coq.Init.Datatypes}{\coqdocnotation{(}}\coqdocvar{f} \coqdocvariable{x}\coqexternalref{::core scope:'(' x ',' x ',' '..' ',' x ')'}{http://coq.inria.fr/distrib/V8.11.0/stdlib//Coq.Init.Datatypes}{\coqdocnotation{,}} \coqdocvar{f'} \coqdocvariable{x'}\coqexternalref{::core scope:'(' x ',' x ',' '..' ',' x ')'}{http://coq.inria.fr/distrib/V8.11.0/stdlib//Coq.Init.Datatypes}{\coqdocnotation{)}} \coqdocvar{\_}\coqexternalref{::type scope:x '/x5C' x}{http://coq.inria.fr/distrib/V8.11.0/stdlib//Coq.Init.Logic}{\coqdocnotation{))}}\coqdoceol
\coqdocindent{4.00em}
\coqexternalref{::type scope:x '/x5C' x}{http://coq.inria.fr/distrib/V8.11.0/stdlib//Coq.Init.Logic}{\coqdocnotation{\ensuremath{\land}}} \coqdockw{let} \coqdocvar{R} (\coqdocvar{yy} \coqdocvar{yy'}: \coqdocvar{Y} \coqexternalref{::type scope:x '+' x}{http://coq.inria.fr/distrib/V8.11.0/stdlib//Coq.Init.Datatypes}{\coqdocnotation{+}} \coqdocvar{Y'}) := \coqdockw{match} \coqdocvariable{yy}, \coqdocvariable{yy'} \coqdockw{with}\coqdoceol
\coqdocindent{5.00em}
\ensuremath{|} \coqexternalref{inl}{http://coq.inria.fr/distrib/V8.11.0/stdlib//Coq.Init.Datatypes}{\coqdocconstructor{inl}} \coqdocvar{y}, \coqexternalref{inl}{http://coq.inria.fr/distrib/V8.11.0/stdlib//Coq.Init.Datatypes}{\coqdocconstructor{inl}} \coqdocvar{y'} \ensuremath{\Rightarrow} \coqdocvariable{rec} \coqexternalref{::core scope:'(' x ',' x ',' '..' ',' x ')'}{http://coq.inria.fr/distrib/V8.11.0/stdlib//Coq.Init.Datatypes}{\coqdocnotation{(}}\coqdocvar{g} \coqdocvar{y}\coqexternalref{::core scope:'(' x ',' x ',' '..' ',' x ')'}{http://coq.inria.fr/distrib/V8.11.0/stdlib//Coq.Init.Datatypes}{\coqdocnotation{,}} \coqdocvar{g} \coqdocvar{y'}\coqexternalref{::core scope:'(' x ',' x ',' '..' ',' x ')'}{http://coq.inria.fr/distrib/V8.11.0/stdlib//Coq.Init.Datatypes}{\coqdocnotation{)}} \coqdocvar{\_}\coqdoceol
\coqdocindent{5.00em}
\ensuremath{|} \coqexternalref{inl}{http://coq.inria.fr/distrib/V8.11.0/stdlib//Coq.Init.Datatypes}{\coqdocconstructor{inl}} \coqdocvar{y}, \coqexternalref{inr}{http://coq.inria.fr/distrib/V8.11.0/stdlib//Coq.Init.Datatypes}{\coqdocconstructor{inr}} \coqdocvar{y'} \ensuremath{\Rightarrow} \coqdocvariable{rec} \coqexternalref{::core scope:'(' x ',' x ',' '..' ',' x ')'}{http://coq.inria.fr/distrib/V8.11.0/stdlib//Coq.Init.Datatypes}{\coqdocnotation{(}}\coqdocvar{g} \coqdocvar{y}\coqexternalref{::core scope:'(' x ',' x ',' '..' ',' x ')'}{http://coq.inria.fr/distrib/V8.11.0/stdlib//Coq.Init.Datatypes}{\coqdocnotation{,}} \coqdocvar{g'} \coqdocvar{y'}\coqexternalref{::core scope:'(' x ',' x ',' '..' ',' x ')'}{http://coq.inria.fr/distrib/V8.11.0/stdlib//Coq.Init.Datatypes}{\coqdocnotation{)}} \coqdocvar{\_}\coqdoceol
\coqdocindent{5.00em}
\ensuremath{|} \coqexternalref{inr}{http://coq.inria.fr/distrib/V8.11.0/stdlib//Coq.Init.Datatypes}{\coqdocconstructor{inr}} \coqdocvar{y}, \coqexternalref{inl}{http://coq.inria.fr/distrib/V8.11.0/stdlib//Coq.Init.Datatypes}{\coqdocconstructor{inl}} \coqdocvar{y'} \ensuremath{\Rightarrow} \coqdocvariable{rec} \coqexternalref{::core scope:'(' x ',' x ',' '..' ',' x ')'}{http://coq.inria.fr/distrib/V8.11.0/stdlib//Coq.Init.Datatypes}{\coqdocnotation{(}}\coqdocvar{g'} \coqdocvar{y}\coqexternalref{::core scope:'(' x ',' x ',' '..' ',' x ')'}{http://coq.inria.fr/distrib/V8.11.0/stdlib//Coq.Init.Datatypes}{\coqdocnotation{,}} \coqdocvar{g} \coqdocvar{y'}\coqexternalref{::core scope:'(' x ',' x ',' '..' ',' x ')'}{http://coq.inria.fr/distrib/V8.11.0/stdlib//Coq.Init.Datatypes}{\coqdocnotation{)}} \coqdocvar{\_}\coqdoceol
\coqdocindent{5.00em}
\ensuremath{|} \coqexternalref{inr}{http://coq.inria.fr/distrib/V8.11.0/stdlib//Coq.Init.Datatypes}{\coqdocconstructor{inr}} \coqdocvar{y}, \coqexternalref{inr}{http://coq.inria.fr/distrib/V8.11.0/stdlib//Coq.Init.Datatypes}{\coqdocconstructor{inr}} \coqdocvar{y'} \ensuremath{\Rightarrow} \coqdocvariable{rec} \coqexternalref{::core scope:'(' x ',' x ',' '..' ',' x ')'}{http://coq.inria.fr/distrib/V8.11.0/stdlib//Coq.Init.Datatypes}{\coqdocnotation{(}}\coqdocvar{g'} \coqdocvar{y}\coqexternalref{::core scope:'(' x ',' x ',' '..' ',' x ')'}{http://coq.inria.fr/distrib/V8.11.0/stdlib//Coq.Init.Datatypes}{\coqdocnotation{,}} \coqdocvar{g'} \coqdocvar{y'}\coqexternalref{::core scope:'(' x ',' x ',' '..' ',' x ')'}{http://coq.inria.fr/distrib/V8.11.0/stdlib//Coq.Init.Datatypes}{\coqdocnotation{)}} \coqdocvar{\_}\coqdoceol
\coqdocindent{5.00em}
\coqdockw{end} \coqdoctac{in} \coqdoceol
\coqdocindent{5.00em}
\coqref{NFO.BoolExpr.eq bexpr}{\coqdocdefinition{eq\_bexpr}} \coqdocvariable{R} (\coqref{NFO.BoolExpr.map}{\coqdocdefinition{map}} \coqexternalref{inl}{http://coq.inria.fr/distrib/V8.11.0/stdlib//Coq.Init.Datatypes}{\coqdocconstructor{inl}} \coqdocvar{e}) (\coqref{NFO.BoolExpr.map}{\coqdocdefinition{map}} \coqexternalref{inr}{http://coq.inria.fr/distrib/V8.11.0/stdlib//Coq.Init.Datatypes}{\coqdocconstructor{inr}} \coqdocvar{e'})\coqdoceol
\coqdocindent{2.00em}
\coqdockw{end}) \coqexternalref{eq refl}{http://coq.inria.fr/distrib/V8.11.0/stdlib//Coq.Init.Logic}{\coqdocconstructor{eq\_refl}}\coqdoceol
\coqdocindent{0.50em}
))\coqdoceol
\coqdocindent{0.50em}
; \coqdoctac{rewrite} \coqdocvar{eqx}; \coqdoctac{eauto} \coqdockw{with} \coqdocvar{Wff}.\coqdoceol
\coqdocnoindent
\coqdockw{Defined}.\coqdoceol
\coqdocnoindent
\coqdockw{Definition} \coqdefRef{NFO.Eq.EQ}{EQ}{\coqdocdefinition{EQ}} \coqdocvar{s} \coqdocvar{s'} := \coqref{NFO.Eq.EQ'}{\coqdocdefinition{EQ'}} \coqexternalref{::core scope:'(' x ',' x ',' '..' ',' x ')'}{http://coq.inria.fr/distrib/V8.11.0/stdlib//Coq.Init.Datatypes}{\coqdocnotation{(}}\coqdocvariable{s}\coqexternalref{::core scope:'(' x ',' x ',' '..' ',' x ')'}{http://coq.inria.fr/distrib/V8.11.0/stdlib//Coq.Init.Datatypes}{\coqdocnotation{,}} \coqdocvariable{s'}\coqexternalref{::core scope:'(' x ',' x ',' '..' ',' x ')'}{http://coq.inria.fr/distrib/V8.11.0/stdlib//Coq.Init.Datatypes}{\coqdocnotation{)}}.\coqdoceol
\end{coqdoccode}

XXX

\TODO{copied here property of EQ boolean:}

This lemma is quite important: it will show that our initial
definition of NFO equality is correct.
TODO: Because: we wanted to write the rhs, but we had to write
the lhs in order to be able to prove termination.

\begin{coqdoccode}
\coqdocnoindent
\coqdockw{Lemma} \coqdefRef{NFO.BoolExpr.eq bexpr simpl}{eq\_bexpr\_simpl}{\coqdoclemma{eq\_bexpr\_simpl}}:\coqdoceol
\coqdocindent{1.00em}
\coqdockw{\ensuremath{\forall}} \{\coqdocvar{Y} \coqdocvar{Y'} \coqdocvar{Z} \coqdocvar{R}\} \{\coqdocvar{g}: \coqdocvariable{Y} \coqexternalref{::type scope:x '->' x}{http://coq.inria.fr/distrib/V8.11.0/stdlib//Coq.Init.Logic}{\coqdocnotation{\ensuremath{\rightarrow}}} \coqdocvariable{Z}\} \{\coqdocvar{g'}: \coqdocvariable{Y'} \coqexternalref{::type scope:x '->' x}{http://coq.inria.fr/distrib/V8.11.0/stdlib//Coq.Init.Logic}{\coqdocnotation{\ensuremath{\rightarrow}}} \coqdocvariable{Z}\} \{\coqdocvar{e} \coqdocvar{e'}\},\coqdoceol
\coqdocindent{2.00em}
\coqexternalref{Equivalence}{http://coq.inria.fr/distrib/V8.11.0/stdlib//Coq.Classes.RelationClasses}{\coqdocclass{Equivalence}} \coqdocvariable{R} \coqexternalref{::type scope:x '->' x}{http://coq.inria.fr/distrib/V8.11.0/stdlib//Coq.Init.Logic}{\coqdocnotation{\ensuremath{\rightarrow}}}\coqdoceol
\coqdocindent{3.00em}
\coqref{NFO.BoolExpr.eq bexpr}{\coqdocdefinition{eq\_bexpr}} (\coqdocvariable{R} \coqref{Internal.Misc.:::x 'xE2xA8x80' x}{\coqdocnotation{⨀}} \coqref{Internal.Misc.:::x 'xE2xA8x80' x}{\coqdocnotation{(}}\coqdocvariable{g} \coqref{Internal.Misc.:::x 'xE2xA8x81' x}{\coqdocnotation{⨁}} \coqdocvariable{g'}\coqref{Internal.Misc.:::x 'xE2xA8x80' x}{\coqdocnotation{)}}) (\coqref{NFO.BoolExpr.map}{\coqdocdefinition{map}} \coqexternalref{inl}{http://coq.inria.fr/distrib/V8.11.0/stdlib//Coq.Init.Datatypes}{\coqdocconstructor{inl}} \coqdocvariable{e}) (\coqref{NFO.BoolExpr.map}{\coqdocdefinition{map}} \coqexternalref{inr}{http://coq.inria.fr/distrib/V8.11.0/stdlib//Coq.Init.Datatypes}{\coqdocconstructor{inr}} \coqdocvariable{e'})\coqdoceol
\coqdocindent{4.00em}
\coqexternalref{::type scope:x '<->' x}{http://coq.inria.fr/distrib/V8.11.0/stdlib//Coq.Init.Logic}{\coqdocnotation{\ensuremath{\leftrightarrow}}} \coqref{NFO.BoolExpr.eq bexpr}{\coqdocdefinition{eq\_bexpr}} \coqdocvariable{R} (\coqref{NFO.BoolExpr.map}{\coqdocdefinition{map}} \coqdocvariable{g} \coqdocvariable{e}) (\coqref{NFO.BoolExpr.map}{\coqdocdefinition{map}} \coqdocvariable{g'} \coqdocvariable{e'}).\coqdoceol
\coqdocemptyline
\end{coqdoccode}

\subsubsection{Detour: Well-founded Orders}


\subsubsection{Back to Equality}




\subsubsection{Equivalence TODO copied here eq bexpr}

In this section, we prove that eq\_bexpr is an equivalence relation.
Actually, we prove variations of the usual reflexivity, symmetry and
trasitivity. These variants are exactly what is needed to prove
that equality of NFO sets is an equivalence relation (see NFO.Eeq).
\begin{coqdoccode}
\coqdocemptyline
\coqdocnoindent
\coqdockw{Lemma} \coqdefRef{NFO.BoolExpr.eq bexpr refl}{eq\_bexpr\_refl}{\coqdoclemma{eq\_bexpr\_refl}} \{\coqdocvar{X} \coqdocvar{Y}\} \{\coqdocvar{f}: \coqdocvariable{X} \coqexternalref{::type scope:x '->' x}{http://coq.inria.fr/distrib/V8.11.0/stdlib//Coq.Init.Logic}{\coqdocnotation{\ensuremath{\rightarrow}}} \coqdocvariable{Y}\} \{\coqdocvar{R}: \coqdocvariable{Y} \coqexternalref{::type scope:x '->' x}{http://coq.inria.fr/distrib/V8.11.0/stdlib//Coq.Init.Logic}{\coqdocnotation{\ensuremath{\rightarrow}}} \coqdocvariable{Y} \coqexternalref{::type scope:x '->' x}{http://coq.inria.fr/distrib/V8.11.0/stdlib//Coq.Init.Logic}{\coqdocnotation{\ensuremath{\rightarrow}}} \coqdockw{Prop}\} \{\coqdocvar{e}\}:\coqdoceol
\coqdocindent{2.00em}
\coqexternalref{::type scope:x '->' x}{http://coq.inria.fr/distrib/V8.11.0/stdlib//Coq.Init.Logic}{\coqdocnotation{(}}\coqdockw{\ensuremath{\forall}} \coqdocvar{x}, \coqdocvariable{R} (\coqdocvariable{f} \coqdocvariable{x}) (\coqdocvariable{f} \coqdocvariable{x})\coqexternalref{::type scope:x '->' x}{http://coq.inria.fr/distrib/V8.11.0/stdlib//Coq.Init.Logic}{\coqdocnotation{)}}\coqdoceol
\coqdocindent{3.00em}
\coqexternalref{::type scope:x '->' x}{http://coq.inria.fr/distrib/V8.11.0/stdlib//Coq.Init.Logic}{\coqdocnotation{\ensuremath{\rightarrow}}} \coqref{NFO.BoolExpr.eq bexpr}{\coqdocdefinition{eq\_bexpr}} (\coqdocvariable{R} \coqref{Internal.Misc.:::x 'xE2xA8x80' x}{\coqdocnotation{⨀}} \coqref{Internal.Misc.:::x 'xE2xA8x80' x}{\coqdocnotation{(}}\coqdocvariable{f} \coqref{Internal.Misc.:::x 'xE2xA8x81' x}{\coqdocnotation{⨁}} \coqdocvariable{f}\coqref{Internal.Misc.:::x 'xE2xA8x80' x}{\coqdocnotation{)}}) (\coqref{NFO.BoolExpr.map}{\coqdocdefinition{map}} \coqexternalref{inl}{http://coq.inria.fr/distrib/V8.11.0/stdlib//Coq.Init.Datatypes}{\coqdocconstructor{inl}} \coqdocvariable{e}) (\coqref{NFO.BoolExpr.map}{\coqdocdefinition{map}} \coqexternalref{inr}{http://coq.inria.fr/distrib/V8.11.0/stdlib//Coq.Init.Datatypes}{\coqdocconstructor{inr}} \coqdocvariable{e}).\coqdoceol
\coqdocemptyline
\coqdocnoindent
\coqdockw{Lemma} \coqdefRef{NFO.BoolExpr.eq bexpr sym}{eq\_bexpr\_sym}{\coqdoclemma{eq\_bexpr\_sym}} \{\coqdocvar{X} \coqdocvar{Y} \coqdocvar{Z} \coqdocvar{R}\} \{\coqdocvar{f}: \coqdocvariable{X} \coqexternalref{::type scope:x '->' x}{http://coq.inria.fr/distrib/V8.11.0/stdlib//Coq.Init.Logic}{\coqdocnotation{\ensuremath{\rightarrow}}} \coqdocvariable{Z}\} \{\coqdocvar{g}: \coqdocvariable{Y} \coqexternalref{::type scope:x '->' x}{http://coq.inria.fr/distrib/V8.11.0/stdlib//Coq.Init.Logic}{\coqdocnotation{\ensuremath{\rightarrow}}} \coqdocvariable{Z}\} \{\coqdocvar{e} \coqdocvar{e'}\} :\coqdoceol
\coqdocindent{1.00em}
\coqref{NFO.BoolExpr.eq bexpr}{\coqdocdefinition{eq\_bexpr}} (\coqdocvariable{R} \coqref{Internal.Misc.:::x 'xE2xA8x80' x}{\coqdocnotation{⨀}} \coqref{Internal.Misc.:::x 'xE2xA8x80' x}{\coqdocnotation{(}}\coqdocvariable{f} \coqref{Internal.Misc.:::x 'xE2xA8x81' x}{\coqdocnotation{⨁}} \coqdocvariable{g}\coqref{Internal.Misc.:::x 'xE2xA8x80' x}{\coqdocnotation{)}}) (\coqref{NFO.BoolExpr.map}{\coqdocdefinition{map}} \coqexternalref{inl}{http://coq.inria.fr/distrib/V8.11.0/stdlib//Coq.Init.Datatypes}{\coqdocconstructor{inl}} \coqdocvariable{e}) (\coqref{NFO.BoolExpr.map}{\coqdocdefinition{map}} \coqexternalref{inr}{http://coq.inria.fr/distrib/V8.11.0/stdlib//Coq.Init.Datatypes}{\coqdocconstructor{inr}} \coqdocvariable{e'})\coqdoceol
\coqdocindent{2.00em}
\coqexternalref{::type scope:x '->' x}{http://coq.inria.fr/distrib/V8.11.0/stdlib//Coq.Init.Logic}{\coqdocnotation{\ensuremath{\rightarrow}}} \coqref{NFO.BoolExpr.eq bexpr}{\coqdocdefinition{eq\_bexpr}} (\coqdocvariable{R} \coqref{Internal.Misc.:::x 'xE2xA8x80' x}{\coqdocnotation{⨀}} \coqref{Internal.Misc.:::x 'xE2xA8x80' x}{\coqdocnotation{(}}\coqdocvariable{g} \coqref{Internal.Misc.:::x 'xE2xA8x81' x}{\coqdocnotation{⨁}} \coqdocvariable{f}\coqref{Internal.Misc.:::x 'xE2xA8x80' x}{\coqdocnotation{)}}) (\coqref{NFO.BoolExpr.map}{\coqdocdefinition{map}} \coqexternalref{inl}{http://coq.inria.fr/distrib/V8.11.0/stdlib//Coq.Init.Datatypes}{\coqdocconstructor{inl}} \coqdocvariable{e'}) (\coqref{NFO.BoolExpr.map}{\coqdocdefinition{map}} \coqexternalref{inr}{http://coq.inria.fr/distrib/V8.11.0/stdlib//Coq.Init.Datatypes}{\coqdocconstructor{inr}} \coqdocvariable{e}).\coqdoceol
\coqdocemptyline
\end{coqdoccode}
TODO: This needs some love... \begin{coqdoccode}
\coqdocnoindent
\coqdockw{Lemma} \coqdefRef{NFO.BoolExpr.eq bexpr trans}{eq\_bexpr\_trans}{\coqdoclemma{eq\_bexpr\_trans}} \{\coqdocvar{X} \coqdocvar{Y} \coqdocvar{Z} \coqdocvar{W}\} \{\coqdocvar{R} : \coqdocvariable{W} \coqexternalref{::type scope:x '->' x}{http://coq.inria.fr/distrib/V8.11.0/stdlib//Coq.Init.Logic}{\coqdocnotation{\ensuremath{\rightarrow}}} \coqdocvariable{W} \coqexternalref{::type scope:x '->' x}{http://coq.inria.fr/distrib/V8.11.0/stdlib//Coq.Init.Logic}{\coqdocnotation{\ensuremath{\rightarrow}}} \coqdockw{Prop}\} \{\coqdocvar{f1} \coqdocvar{f2} \coqdocvar{f3}\}\coqdoceol
\coqdocindent{1.00em}
\{\coqdocvar{e1} : @\coqref{NFO.BoolExpr.BExpr}{\coqdocinductive{BExpr}} \coqdocvariable{X}\} \{\coqdocvar{e2} : @\coqref{NFO.BoolExpr.BExpr}{\coqdocinductive{BExpr}} \coqdocvariable{Y}\} \{\coqdocvar{e3} : @\coqref{NFO.BoolExpr.BExpr}{\coqdocinductive{BExpr}} \coqdocvariable{Z}\}\coqdoceol
\coqdocindent{1.00em}
:  \coqexternalref{::type scope:x '->' x}{http://coq.inria.fr/distrib/V8.11.0/stdlib//Coq.Init.Logic}{\coqdocnotation{(}}\coqdockw{\ensuremath{\forall}} \coqdocvar{a} \coqdocvar{b}, \coqdocvariable{R} \coqdocvariable{a} \coqdocvariable{b} \coqexternalref{::type scope:x '->' x}{http://coq.inria.fr/distrib/V8.11.0/stdlib//Coq.Init.Logic}{\coqdocnotation{\ensuremath{\rightarrow}}} \coqdocvariable{R} \coqdocvariable{b} \coqdocvariable{a}\coqexternalref{::type scope:x '->' x}{http://coq.inria.fr/distrib/V8.11.0/stdlib//Coq.Init.Logic}{\coqdocnotation{)}}\coqdoceol
\coqdocindent{1.00em}
\coqexternalref{::type scope:x '->' x}{http://coq.inria.fr/distrib/V8.11.0/stdlib//Coq.Init.Logic}{\coqdocnotation{\ensuremath{\rightarrow}}} \coqexternalref{::type scope:x '->' x}{http://coq.inria.fr/distrib/V8.11.0/stdlib//Coq.Init.Logic}{\coqdocnotation{(}}\coqdockw{\ensuremath{\forall}} \coqdocvar{a} \coqdocvar{b} \coqdocvar{c}, \coqref{Internal.Misc.inv3}{\coqdocdefinition{inv3}} \coqdocvariable{f1} \coqdocvariable{f2} \coqdocvariable{f3} \coqdocvariable{a} \coqexternalref{::type scope:x '->' x}{http://coq.inria.fr/distrib/V8.11.0/stdlib//Coq.Init.Logic}{\coqdocnotation{\ensuremath{\rightarrow}}} \coqref{Internal.Misc.inv3}{\coqdocdefinition{inv3}} \coqdocvariable{f1} \coqdocvariable{f2} \coqdocvariable{f3} \coqdocvariable{b} \coqexternalref{::type scope:x '->' x}{http://coq.inria.fr/distrib/V8.11.0/stdlib//Coq.Init.Logic}{\coqdocnotation{\ensuremath{\rightarrow}}} \coqref{Internal.Misc.inv3}{\coqdocdefinition{inv3}} \coqdocvariable{f1} \coqdocvariable{f2} \coqdocvariable{f3} \coqdocvariable{c} \coqexternalref{::type scope:x '->' x}{http://coq.inria.fr/distrib/V8.11.0/stdlib//Coq.Init.Logic}{\coqdocnotation{\ensuremath{\rightarrow}}} \coqdocvariable{R} \coqdocvariable{a} \coqdocvariable{b} \coqexternalref{::type scope:x '->' x}{http://coq.inria.fr/distrib/V8.11.0/stdlib//Coq.Init.Logic}{\coqdocnotation{\ensuremath{\rightarrow}}} \coqdocvariable{R} \coqdocvariable{b} \coqdocvariable{c} \coqexternalref{::type scope:x '->' x}{http://coq.inria.fr/distrib/V8.11.0/stdlib//Coq.Init.Logic}{\coqdocnotation{\ensuremath{\rightarrow}}} \coqdocvariable{R} \coqdocvariable{a} \coqdocvariable{c}\coqexternalref{::type scope:x '->' x}{http://coq.inria.fr/distrib/V8.11.0/stdlib//Coq.Init.Logic}{\coqdocnotation{)}}\coqdoceol
\coqdocindent{1.00em}
\coqexternalref{::type scope:x '->' x}{http://coq.inria.fr/distrib/V8.11.0/stdlib//Coq.Init.Logic}{\coqdocnotation{\ensuremath{\rightarrow}}} \coqref{NFO.BoolExpr.eq bexpr}{\coqdocdefinition{eq\_bexpr}} (\coqdocvariable{R} \coqref{Internal.Misc.:::x 'xE2xA8x80' x}{\coqdocnotation{⨀}} \coqref{Internal.Misc.:::x 'xE2xA8x80' x}{\coqdocnotation{(}}\coqdocvariable{f1} \coqref{Internal.Misc.:::x 'xE2xA8x81' x}{\coqdocnotation{⨁}} \coqdocvariable{f2}\coqref{Internal.Misc.:::x 'xE2xA8x80' x}{\coqdocnotation{)}}) (\coqref{NFO.BoolExpr.map}{\coqdocdefinition{map}} \coqexternalref{inl}{http://coq.inria.fr/distrib/V8.11.0/stdlib//Coq.Init.Datatypes}{\coqdocconstructor{inl}} \coqdocvariable{e1}) (\coqref{NFO.BoolExpr.map}{\coqdocdefinition{map}} \coqexternalref{inr}{http://coq.inria.fr/distrib/V8.11.0/stdlib//Coq.Init.Datatypes}{\coqdocconstructor{inr}} \coqdocvariable{e2})\coqdoceol
\coqdocindent{1.00em}
\coqexternalref{::type scope:x '->' x}{http://coq.inria.fr/distrib/V8.11.0/stdlib//Coq.Init.Logic}{\coqdocnotation{\ensuremath{\rightarrow}}} \coqref{NFO.BoolExpr.eq bexpr}{\coqdocdefinition{eq\_bexpr}} (\coqdocvariable{R} \coqref{Internal.Misc.:::x 'xE2xA8x80' x}{\coqdocnotation{⨀}} \coqref{Internal.Misc.:::x 'xE2xA8x80' x}{\coqdocnotation{(}}\coqdocvariable{f2} \coqref{Internal.Misc.:::x 'xE2xA8x81' x}{\coqdocnotation{⨁}} \coqdocvariable{f3}\coqref{Internal.Misc.:::x 'xE2xA8x80' x}{\coqdocnotation{)}}) (\coqref{NFO.BoolExpr.map}{\coqdocdefinition{map}} \coqexternalref{inl}{http://coq.inria.fr/distrib/V8.11.0/stdlib//Coq.Init.Datatypes}{\coqdocconstructor{inl}} \coqdocvariable{e2}) (\coqref{NFO.BoolExpr.map}{\coqdocdefinition{map}} \coqexternalref{inr}{http://coq.inria.fr/distrib/V8.11.0/stdlib//Coq.Init.Datatypes}{\coqdocconstructor{inr}} \coqdocvariable{e3})\coqdoceol
\coqdocindent{1.00em}
\coqexternalref{::type scope:x '->' x}{http://coq.inria.fr/distrib/V8.11.0/stdlib//Coq.Init.Logic}{\coqdocnotation{\ensuremath{\rightarrow}}} \coqref{NFO.BoolExpr.eq bexpr}{\coqdocdefinition{eq\_bexpr}} (\coqdocvariable{R} \coqref{Internal.Misc.:::x 'xE2xA8x80' x}{\coqdocnotation{⨀}} \coqref{Internal.Misc.:::x 'xE2xA8x80' x}{\coqdocnotation{(}}\coqdocvariable{f1} \coqref{Internal.Misc.:::x 'xE2xA8x81' x}{\coqdocnotation{⨁}} \coqdocvariable{f3}\coqref{Internal.Misc.:::x 'xE2xA8x80' x}{\coqdocnotation{)}}) (\coqref{NFO.BoolExpr.map}{\coqdocdefinition{map}} \coqexternalref{inl}{http://coq.inria.fr/distrib/V8.11.0/stdlib//Coq.Init.Datatypes}{\coqdocconstructor{inl}} \coqdocvariable{e1}) (\coqref{NFO.BoolExpr.map}{\coqdocdefinition{map}} \coqexternalref{inr}{http://coq.inria.fr/distrib/V8.11.0/stdlib//Coq.Init.Datatypes}{\coqdocconstructor{inr}} \coqdocvariable{e3}).\coqdoceol
\end{coqdoccode}

\TODO{Missing how to lift to EQ, and then SETOID definition}

\subsection{Set Membership}

\TODO{is true if and only if precisely one of A and B is true}

\begin{coqdoccode}
  \coqdocemptyline
  \coqdocnoindent
  \coqdockw{Definition} \coqdefRef{NFO.Xor.xor}{xor}{\coqdocdefinition{xor}} \coqdocvar{p} \coqdocvar{q} := \coqexternalref{::type scope:x 'x5C/' x}{http://coq.inria.fr/distrib/V8.11.0/stdlib//Coq.Init.Logic}{\coqdocnotation{(}}\coqdocvariable{p} \coqexternalref{::type scope:x '/x5C' x}{http://coq.inria.fr/distrib/V8.11.0/stdlib//Coq.Init.Logic}{\coqdocnotation{\ensuremath{\land}}} \coqexternalref{::type scope:'x7E' x}{http://coq.inria.fr/distrib/V8.11.0/stdlib//Coq.Init.Logic}{\coqdocnotation{\ensuremath{\lnot}}}\coqdocvariable{q}\coqexternalref{::type scope:x 'x5C/' x}{http://coq.inria.fr/distrib/V8.11.0/stdlib//Coq.Init.Logic}{\coqdocnotation{)}} \coqexternalref{::type scope:x 'x5C/' x}{http://coq.inria.fr/distrib/V8.11.0/stdlib//Coq.Init.Logic}{\coqdocnotation{\ensuremath{\lor}}} \coqexternalref{::type scope:x 'x5C/' x}{http://coq.inria.fr/distrib/V8.11.0/stdlib//Coq.Init.Logic}{\coqdocnotation{(}}\coqexternalref{::type scope:'x7E' x}{http://coq.inria.fr/distrib/V8.11.0/stdlib//Coq.Init.Logic}{\coqdocnotation{\ensuremath{\lnot}}}\coqdocvariable{p} \coqexternalref{::type scope:x '/x5C' x}{http://coq.inria.fr/distrib/V8.11.0/stdlib//Coq.Init.Logic}{\coqdocnotation{\ensuremath{\land}}} \coqdocvariable{q}\coqexternalref{::type scope:x 'x5C/' x}{http://coq.inria.fr/distrib/V8.11.0/stdlib//Coq.Init.Logic}{\coqdocnotation{)}}.\coqdoceol
  \coqdocnoindent
  \coqdockw{Infix} \coqdefRef{NFO.Xor.:::x 'xE2x8AxBB' x}{"}{"}⊻" := \coqref{NFO.Xor.xor}{\coqdocdefinition{xor}} (\coqdoctac{at} \coqdockw{level} 80, \coqdoctac{right} \coqdockw{associativity}).\coqdoceol
  \coqdocemptyline
\end{coqdoccode}

\begin{coqdoccode}
  \coqdocnoindent
  \coqdockw{Fixpoint} \coqdefRef{NFO.In.IN bexpr}{IN\_bexpr}{\coqdocdefinition{IN\_bexpr}} \{\coqdocvar{Z}\} (\coqdocvar{R}: \coqdocvariable{Z} \coqexternalref{::type scope:x '->' x}{http://coq.inria.fr/distrib/V8.11.0/stdlib//Coq.Init.Logic}{\coqdocnotation{\ensuremath{\rightarrow}}} \coqdocvariable{Z} \coqexternalref{::type scope:x '->' x}{http://coq.inria.fr/distrib/V8.11.0/stdlib//Coq.Init.Logic}{\coqdocnotation{\ensuremath{\rightarrow}}} \coqdockw{Prop}) \coqdocvar{y} (\coqdocvar{p}: \coqref{NFO.BoolExpr.BExpr}{\coqdocinductive{BExpr}}) := \coqdockw{match} \coqdocvariable{p} \coqdockw{with}\coqdoceol
  \coqdocindent{1.00em}
  \ensuremath{|} \coqref{NFO.BoolExpr.Bot}{\coqdocconstructor{Bot}} \ensuremath{\Rightarrow} \coqexternalref{False}{http://coq.inria.fr/distrib/V8.11.0/stdlib//Coq.Init.Logic}{\coqdocinductive{False}}\coqdoceol
  \coqdocindent{1.00em}
  \ensuremath{|} \coqref{NFO.BoolExpr.Atom}{\coqdocconstructor{Atom}} \coqdocvar{x} \ensuremath{\Rightarrow} \coqdocvariable{R} \coqdocvar{x} \coqdocvariable{y}\coqdoceol
  \coqdocindent{1.00em}
  \ensuremath{|} \coqref{NFO.BoolExpr.Not}{\coqdocconstructor{Not}} \coqdocvar{p'} \ensuremath{\Rightarrow} \coqexternalref{::type scope:'x7E' x}{http://coq.inria.fr/distrib/V8.11.0/stdlib//Coq.Init.Logic}{\coqdocnotation{\ensuremath{\lnot}}} \coqref{NFO.In.IN bexpr}{\coqdocdefinition{IN\_bexpr}} \coqdocvariable{R} \coqdocvariable{y} \coqdocvar{p'}\coqdoceol
  \coqdocindent{1.00em}
  \ensuremath{|} \coqref{NFO.BoolExpr.Or}{\coqdocconstructor{Or}} \coqdocvar{p1} \coqdocvar{p2} \ensuremath{\Rightarrow} \coqref{NFO.In.IN bexpr}{\coqdocdefinition{IN\_bexpr}} \coqdocvariable{R} \coqdocvariable{y} \coqdocvar{p1} \coqexternalref{::type scope:x 'x5C/' x}{http://coq.inria.fr/distrib/V8.11.0/stdlib//Coq.Init.Logic}{\coqdocnotation{\ensuremath{\lor}}} \coqref{NFO.In.IN bexpr}{\coqdocdefinition{IN\_bexpr}} \coqdocvariable{R} \coqdocvariable{y} \coqdocvar{p2}\coqdoceol
  \coqdocnoindent
  \coqdockw{end}.\coqdoceol
\end{coqdoccode}

aaa

\begin{coqdoccode}
  \coqdocnoindent
  \sout{\coqdockw{Definition}} \coqdoclemma{IN} \coqdocvar{s} \coqdocvar{s'} := \coqdockw{match} \coqdocvariable{s'} \coqdockw{with}\coqdoceol
  \coqdocindent{2.00em}
  \ensuremath{|} \coqref{NFO.Model.S}{\coqdocconstructor{S}} \coqdocvar{X} \coqdocvar{Y} \coqdocvar{f} \coqdocvar{g} \coqdocvar{e} \ensuremath{\Rightarrow} \coqdocvar{s} \coqref{NFO.In.AIN}{\AIN} \coqdocvar{f}
  \XOR \coqref{NFO.In.IN bexpr}{\coqdocdefinition{IN\_bexpr}} \coqref{NFO.In.IN}{\coqdocdefinition{IN}} \coqdocvar{s} (\coqref{NFO.BoolExpr.map}{\coqdocdefinition{map}} \coqdocvariable{g} \coqdocvariable{e})\coqdoceol
  \coqdocnoindent \coqdockw{end}.
\end{coqdoccode}

bbb

\begin{coqdoccode}
  \coqdocnoindent
  \coqdockw{Lemma} \coqdefRef{NFO.In.IN bexpr map2}{IN\_bexpr\_map2}{\coqdoclemma{IN\_bexpr\_map2}} \{\coqdocvar{X} \coqdocvar{Z}\} (\coqdocvar{R}: \coqdocvariable{Z} \coqexternalref{::type scope:x '->' x}{http://coq.inria.fr/distrib/V8.11.0/stdlib//Coq.Init.Logic}{\coqdocnotation{\ensuremath{\rightarrow}}} \coqdocvariable{Z} \coqexternalref{::type scope:x '->' x}{http://coq.inria.fr/distrib/V8.11.0/stdlib//Coq.Init.Logic}{\coqdocnotation{\ensuremath{\rightarrow}}} \coqdockw{Prop}) \coqdocvar{f} \coqdocvar{z} (\coqdocvar{e}: @\coqref{NFO.BoolExpr.BExpr}{\coqdocinductive{BExpr}} \coqdocvariable{X}):\coqdoceol
  \coqdocindent{1.00em}
  \coqref{NFO.BoolExpr.:::'xE2x9FxA6' x 'xE2x9FxA7'}{\coqdocnotation{⟦}} \coqref{NFO.BoolExpr.map}{\coqdocdefinition{map}} (\coqdockw{fun} \coqdocvar{x} \ensuremath{\Rightarrow} \coqdocvariable{R} (\coqdocvariable{f} \coqdocvariable{x}) \coqdocvariable{z}) \coqdocvariable{e} \coqref{NFO.BoolExpr.:::'xE2x9FxA6' x 'xE2x9FxA7'}{\coqdocnotation{⟧}} \coqexternalref{::type scope:x '<->' x}{http://coq.inria.fr/distrib/V8.11.0/stdlib//Coq.Init.Logic}{\coqdocnotation{\ensuremath{\leftrightarrow}}} \coqref{NFO.In.IN bexpr}{\coqdocdefinition{IN\_bexpr}} \coqdocvariable{R} \coqdocvariable{z} (\coqref{NFO.BoolExpr.map}{\coqdocdefinition{map}} \coqdocvariable{f} \coqdocvariable{e}).\coqdoceol
\end{coqdoccode}

ccc

\begin{coqdoccode}
  \coqdocemptyline
  \coqdocnoindent
  \coqdockw{Local} \coqdockw{Definition} \coqdefRef{NFO.In.IN'}{IN'}{\coqdocdefinition{IN'}} : \coqref{NFO.Model.SET}{\coqdocinductive{SET}} \coqexternalref{::type scope:x '*' x}{http://coq.inria.fr/distrib/V8.11.0/stdlib//Coq.Init.Datatypes}{\coqdocnotation{\ensuremath{\times}}} \coqref{NFO.Model.SET}{\coqdocinductive{SET}} \coqexternalref{::type scope:x '->' x}{http://coq.inria.fr/distrib/V8.11.0/stdlib//Coq.Init.Logic}{\coqdocnotation{\ensuremath{\rightarrow}}} \coqdockw{Prop}.\coqdoceol
  \coqdocindent{0.50em}
  \coqdoctac{refine} ( \coqexternalref{Fix}{http://coq.inria.fr/distrib/V8.11.0/stdlib//Coq.Init.Wf}{\coqdocdefinition{Fix}} (\coqexternalref{wf swapprod}{http://coq.inria.fr/distrib/V8.11.0/stdlib//Coq.Wellfounded.Lexicographic\_Product}{\coqdoclemma{wf\_swapprod}} \coqdocvar{\_} \coqref{NFO.Wf.lt}{\coqdocinductive{lt}} \coqref{NFO.Wf.wf lt}{\coqdoclemma{wf\_lt}}) (\coqdockw{fun} \coqdocvar{\_} \ensuremath{\Rightarrow} \coqdockw{Prop}) \coqdoceol
  \coqdocindent{1.00em}
  (\coqdockw{fun} \coqdocvar{i} \coqdockw{rec} \ensuremath{\Rightarrow} \coqdoceol
  \coqdocindent{2.00em}
  (\coqdockw{match} \coqdocvariable{i} \coqdockw{as} \coqdocvar{i0} \coqdockw{return} (\coqdocvariable{i} \coqexternalref{::type scope:x '=' x}{http://coq.inria.fr/distrib/V8.11.0/stdlib//Coq.Init.Logic}{\coqdocnotation{=}} \coqdocvariable{i0} \coqexternalref{::type scope:x '->' x}{http://coq.inria.fr/distrib/V8.11.0/stdlib//Coq.Init.Logic}{\coqdocnotation{\ensuremath{\rightarrow}}} \coqdockw{Prop}) \coqdockw{with} \coqdoceol
  \coqdocindent{3.00em}
  \coqexternalref{::core scope:'(' x ',' x ',' '..' ',' x ')'}{http://coq.inria.fr/distrib/V8.11.0/stdlib//Coq.Init.Datatypes}{\coqdocnotation{(}}\coqdocvar{z}\coqexternalref{::core scope:'(' x ',' x ',' '..' ',' x ')'}{http://coq.inria.fr/distrib/V8.11.0/stdlib//Coq.Init.Datatypes}{\coqdocnotation{,}} \coqref{NFO.Model.S}{\coqdocconstructor{S}} \coqdocvar{X} \coqdocvar{Y} \coqdocvar{f} \coqdocvar{g} \coqdocvar{e}\coqexternalref{::core scope:'(' x ',' x ',' '..' ',' x ')'}{http://coq.inria.fr/distrib/V8.11.0/stdlib//Coq.Init.Datatypes}{\coqdocnotation{)}}
  % \coqdoceol
  % 
  \ensuremath{\Rightarrow} \coqdockw{fun} \coqdocvar{eqx}   \ensuremath{\Rightarrow} \coqdoceol
  \coqdocindent{4.00em}
  \coqdocvar{z} \AIN ~\coqdocvar{f} \coqref{NFO.Xor.:::x 'xE2x8AxBB' x}{\coqdocnotation{⊻}} \coqref{NFO.BoolExpr.:::'xE2x9FxA6' x 'xE2x9FxA7'}{\coqdocnotation{⟦}} \coqref{NFO.BoolExpr.map}{\coqdocdefinition{map}} (\coqdockw{fun} \coqdocvar{y} \ensuremath{\Rightarrow} \coqdocvariable{rec} \coqexternalref{::core scope:'(' x ',' x ',' '..' ',' x ')'}{http://coq.inria.fr/distrib/V8.11.0/stdlib//Coq.Init.Datatypes}{\coqdocnotation{(}}\coqdocvar{g} \coqdocvariable{y}\coqexternalref{::core scope:'(' x ',' x ',' '..' ',' x ')'}{http://coq.inria.fr/distrib/V8.11.0/stdlib//Coq.Init.Datatypes}{\coqdocnotation{,}} \coqdocvar{z}\coqexternalref{::core scope:'(' x ',' x ',' '..' ',' x ')'}{http://coq.inria.fr/distrib/V8.11.0/stdlib//Coq.Init.Datatypes}{\coqdocnotation{)}} \coqdocvar{\_}) \coqdocvar{e} \coqref{NFO.BoolExpr.:::'xE2x9FxA6' x 'xE2x9FxA7'}{\coqdocnotation{⟧}} \coqdoceol
  \coqdocindent{2.00em}
  \coqdockw{end}) \coqexternalref{eq refl}{http://coq.inria.fr/distrib/V8.11.0/stdlib//Coq.Init.Logic}{\coqdocconstructor{eq\_refl}} 
  )).\coqdoceol
  \coqdocnoindent
  [...] \coqdockw{Defined}.\coqdoceol
  \coqdocnoindent
  \coqdockw{Definition} \coqdefRef{NFO.In.IN}{IN}{\coqdocdefinition{IN}} \coqdocvar{s} \coqdocvar{s'} := \coqref{NFO.In.IN'}{\coqdocdefinition{IN'}} \coqexternalref{::core scope:'(' x ',' x ',' '..' ',' x ')'}{http://coq.inria.fr/distrib/V8.11.0/stdlib//Coq.Init.Datatypes}{\coqdocnotation{(}}\coqdocvariable{s}\coqexternalref{::core scope:'(' x ',' x ',' '..' ',' x ')'}{http://coq.inria.fr/distrib/V8.11.0/stdlib//Coq.Init.Datatypes}{\coqdocnotation{,}} \coqdocvariable{s'}\coqexternalref{::core scope:'(' x ',' x ',' '..' ',' x ')'}{http://coq.inria.fr/distrib/V8.11.0/stdlib//Coq.Init.Datatypes}{\coqdocnotation{)}}.\coqdoceol
\coqdocemptyline
\end{coqdoccode}

\TODO{Spiegare il well-order}

\begin{coqdoccode}
  \coqdocnoindent
  \coqdockw{Infix} ``\BIN'' := (\coqdocdefinition{in\_bexpr} \coqdocdefinition{IN}).\coqdoceol
  \coqdocnoindent
  \coqdockw{Lemma} \coqdefRef{NFO.In.IN unfold}{IN\_unfold}{\coqdoclemma{IN\_unfold}} \{\coqdocvar{X} \coqdocvar{Y} \coqdocvar{f} \coqdocvar{g} \coqdocvar{e} \coqdocvar{s}\} :\coqdoceol
  \coqdocindent{1.00em}
  \coqdocvariable{s} \INX \coqref{NFO.Model.S}{\coqdocconstructor{S}} \coqdocvariable{X} \coqdocvariable{Y} \coqdocvariable{f} \coqdocvariable{g} \coqdocvariable{e} \coqexternalref{::type scope:x '<->' x}{http://coq.inria.fr/distrib/V8.11.0/stdlib//Coq.Init.Logic}{\coqdocnotation{\ensuremath{\leftrightarrow}}} \coqdocvariable{s} \coqref{NFO.In.AIN}{\AIN} \coqdocvariable{f} \coqref{NFO.Xor.:::x 'xE2x8AxBB' x}{\coqdocnotation{⊻}} \coqdocvariable{s} \coqref{NFO.In.BIN}{\BIN} (\coqref{NFO.BoolExpr.map}{\coqdocdefinition{map}} \coqdocvariable{g} \coqdocvariable{e}).\coqdoceol
\end{coqdoccode}
\begin{proof}
  By \coqref{NFO.In.IN bexpr map2}{\coqdoclemma{IN\_bexpr\_map2}}.
\end{proof}

\TODO{Missing: morphism, respects (depends on Ain and Bin morphisms, etc.)}

\subsection{Set Operators}
\TODO{universo gia' definito in ex?, and correctness is trivial.}
\TODO{Dire che precedenti (a parte unione) sono facili.}

\begin{coqdoccode}
  \coqdocnoindent
  \coqdockw{Definition} \coqdef{NFO.Sets.compl}{compl}{\coqdocdefinition{compl}} \coqdocvar{s} := \coqdockw{match} \coqdocvariable{s} \coqdockw{with}\coqdoceol
  \coqdocindent{1.00em}
  \coqref{NFO.Model.S}{\coqdocconstructor{S}} \coqdocvar{X} \coqdocvar{Y} \coqdocvar{f} \coqdocvar{g} \coqdocvar{e} \ensuremath{\Rightarrow} \coqref{NFO.Model.S}{\coqdocconstructor{S}} \coqdocvar{X} \coqdocvar{Y} \coqdocvar{f} \coqdocvar{g} (\coqref{NFO.BoolExpr.Not}{\coqdocconstructor{Not}} \coqdocvar{e})\coqdoceol
  \coqdocnoindent
  \coqdockw{end}.\coqdoceol
  \coqdocemptyline
  \coqdocnoindent
  \coqdockw{Theorem} \coqdef{NFO.Sets.compl ok}{compl\_ok}{\coqdoclemma{compl\_ok}}: \coqdockw{\ensuremath{\forall}} \coqdocvar{s} \coqdocvar{t},\coqdoceol
  \coqdocindent{1.00em}
   \coqdocvariable{s} \INX (\coqref{NFO.Sets.compl}{\coqdocdefinition{compl}} \coqdocvariable{t}) \coqexternalref{::type scope:x '<->' x}{http://coq.inria.fr/distrib/V8.11.0/stdlib//Coq.Init.Logic}{\coqdocnotation{\ensuremath{\leftrightarrow}}} \coqexternalref{::type scope:x '<->' x}{http://coq.inria.fr/distrib/V8.11.0/stdlib//Coq.Init.Logic}{\coqdocnotation{(}}\coqdocvariable{s} \INX \coqdocvariable{t} \coqexternalref{::type scope:x '->' x}{http://coq.inria.fr/distrib/V8.11.0/stdlib//Coq.Init.Logic}{\coqdocnotation{\ensuremath{\rightarrow}}} \coqexternalref{False}{http://coq.inria.fr/distrib/V8.11.0/stdlib//Coq.Init.Logic}{\coqdocinductive{False}}\coqexternalref{::type scope:x '<->' x}{http://coq.inria.fr/distrib/V8.11.0/stdlib//Coq.Init.Logic}{\coqdocnotation{)}}.\coqdoceol
\end{coqdoccode}
\begin{proof}
  Let \coqdocvar{t} = \coqref{NFO.Model.S}{\coqdocconstructor{S}} \coqdocvar{X} \coqdocvar{Y} \coqdocvar{f} \coqdocvar{g} \coqdocvar{e}.
  After basic simplifications, the statement becomes:
  \begin{center}
    \coqdockw{\ensuremath{\forall}} \coqdocvar{s} \coqdocvar{t}, \coqdocvar{s} \AIN \coqdocvar{f} \XOR $\neg$ (\coqdocvar{s} \BIN \coqref{NFO.BoolExpr.map}{\coqdocdefinition{map}} \coqdocvar{g} \coqdocvar{e}) \coqexternalref{::type scope:x '<->' x}{http://coq.inria.fr/distrib/V8.11.0/stdlib//Coq.Init.Logic}{\coqdocnotation{\ensuremath{\leftrightarrow}}} $\neg$ (\coqdocvar{s} \AIN \coqdocvar{f}\XOR \coqdocvar{s} \BIN \coqref{NFO.BoolExpr.map}{\coqdocdefinition{map}} \coqdocvar{g} \coqdocvar{e})
  \end{center}
  which holds classically because negation commutes with xor.
\end{proof}

AAA

\begin{coqdoccode}
  \coqdocnoindent
  \coqdockw{Definition} \coqdef{NFO.Sets.sin}{sin}{\coqdocdefinition{sin}} \coqdocvar{x} := \coqdoceol
  \coqdocindent{1.00em}
  \coqref{NFO.Model.S}{\coqdocconstructor{S}} \coqexternalref{unit}{http://coq.inria.fr/distrib/V8.11.0/stdlib//Coq.Init.Datatypes}{\coqdocinductive{unit}} \coqexternalref{False}{http://coq.inria.fr/distrib/V8.11.0/stdlib//Coq.Init.Logic}{\coqdocinductive{False}} (\coqdockw{fun} \coqdocvar{\_} \ensuremath{\Rightarrow} \coqdocvariable{x}) (\coqexternalref{False rect}{http://coq.inria.fr/distrib/V8.11.0/stdlib//Coq.Init.Logic}{\coqdocdefinition{False\_rect}} \coqdocvar{\_}) \coqref{NFO.BoolExpr.Bot}{\coqdocconstructor{Bot}}.\coqdoceol
  \coqdocemptyline
  \coqdocnoindent
  \coqdockw{Theorem} \coqdef{NFO.Sets.sin ok}{sin\_ok}{\coqdoclemma{sin\_ok}}: \coqdockw{\ensuremath{\forall}} \coqdocvar{x} \coqdocvar{y}, \coqdocvariable{x} \INX (\coqref{NFO.Sets.sin}{\coqdocdefinition{sin}} \coqdocvariable{y}) \coqexternalref{::type scope:x '<->' x}{http://coq.inria.fr/distrib/V8.11.0/stdlib//Coq.Init.Logic}{\coqdocnotation{\ensuremath{\leftrightarrow}}} \coqdocvariable{y} \EQX \coqdocvariable{x}.\coqdoceol
\end{coqdoccode}
\begin{proof}
  After basic simplifications, the statement becomes
    \coqdockw{\ensuremath{\forall}} \coqdocvar{s} \coqdocvar{t}, \coqdocvar{t} \EQX \coqdocvar{s} \XOR \coqexternalref{False}{http://coq.inria.fr/distrib/V8.11.0/stdlib//Coq.Init.Logic}{\coqdocinductive{False}} \coqexternalref{::type scope:x '<->' x}{http://coq.inria.fr/distrib/V8.11.0/stdlib//Coq.Init.Logic}{\coqdocnotation{\ensuremath{\leftrightarrow}}} \coqdocvar{t} \EQX \coqdocvar{s},
  which holds classically because \coqexternalref{False}{http://coq.inria.fr/distrib/V8.11.0/stdlib//Coq.Init.Logic}{\coqdocinductive{False}} is the identity element for xor.
\end{proof}

AAA

\begin{coqdoccode}
  \coqdocnoindent
  \coqdockw{Definition} \coqdef{NFO.Sets.cosin}{cosin}{\coqdocdefinition{cosin}} \coqdocvar{s} := \coqdoceol
  \coqdocindent{1.00em}\coqref{NFO.Model.S}{\coqdocconstructor{S}} \coqexternalref{False}{http://coq.inria.fr/distrib/V8.11.0/stdlib//Coq.Init.Logic}{\coqdocinductive{False}} \coqexternalref{unit}{http://coq.inria.fr/distrib/V8.11.0/stdlib//Coq.Init.Datatypes}{\coqdocinductive{unit}} (\coqexternalref{False rect}{http://coq.inria.fr/distrib/V8.11.0/stdlib//Coq.Init.Logic}{\coqdocdefinition{False\_rect}} \coqdocvar{\_}) (\coqdockw{fun} \coqdocvar{\_} \ensuremath{\Rightarrow} \coqdocvariable{s}) (\coqref{NFO.BoolExpr.Atom}{\coqdocconstructor{Atom}} \coqexternalref{tt}{http://coq.inria.fr/distrib/V8.11.0/stdlib//Coq.Init.Datatypes}{\coqdocconstructor{tt}}).\coqdoceol
  \coqdocemptyline
  \coqdocnoindent
  \coqdockw{Theorem} \coqdef{NFO.Sets.cosin ok}{cosin\_ok}{\coqdoclemma{cosin\_ok}}: \coqdockw{\ensuremath{\forall}} \coqdocvar{s} \coqdocvar{t}, \coqdocvariable{s} \INX (\coqref{NFO.Sets.cosin}{\coqdocdefinition{cosin}} \coqdocvariable{t}) \coqexternalref{::type scope:x '<->' x}{http://coq.inria.fr/distrib/V8.11.0/stdlib//Coq.Init.Logic}{\coqdocnotation{\ensuremath{\leftrightarrow}}} \coqdocvariable{t} \INX \coqdocvariable{s}.\coqdoceol
\end{coqdoccode}
\begin{proof}
  After basic simplifications, the statement becomes \coqdockw{\ensuremath{\forall}} \coqdocvar{s} \coqdocvar{t}, \coqexternalref{False}{http://coq.inria.fr/distrib/V8.11.0/stdlib//Coq.Init.Logic}{\coqdocinductive{False}} \XOR (\coqdocvariable{t} \BIN \coqdocvariable{s}) \coqexternalref{::type scope:x '<->' x}{http://coq.inria.fr/distrib/V8.11.0/stdlib//Coq.Init.Logic}{\coqdocnotation{\ensuremath{\leftrightarrow}}} (\coqdocvariable{t} \BIN \coqdocvariable{s}), which holds classically because \coqexternalref{False}{http://coq.inria.fr/distrib/V8.11.0/stdlib//Coq.Init.Logic}{\coqdocinductive{False}} is the identity element for xor.
\end{proof}


\paragraph{Set union.}

Informally: $(a_1 \veebar b_1) \vee (a_2 \veebar b_2) \approx a \veebar (b_1 \lor b_2)$ where 
\[ \begin{array}{rl}
  a := & \hspace{0.8em}(a_1 \land (b_2 \to b_1) \land ( b_1 \to ( a_1 \leftrightarrow b_2 ) )) \\
         & \lor ~(a_2 \land (b_1 \to b_2) \land ( b_2 \to ( a_2 \leftrightarrow b_1 ) )) \\
\end{array} \]

% (* (a -> b) /\ (b -> (c <-> a)) *)


The union of NFO sets is defined as follows:

\begin{itemize}
\item  The resulting B-part is simply the union of the input B-parts.

\item  The resulting A-part is a more complex combination of the input A-parts and B-parts.
      TODO:

\end{itemize}


 TODO: clean up and explain
 
\begin{coqdoccode}
\coqdocnoindent
\coqdockw{Definition} \coqdef{NFO.Union.half}{half}{\coqdocdefinition{half}} \{\coqdocvar{X}\} (\coqdocvar{f}: \coqdocvariable{X} \coqexternalref{::type scope:x '->' x}{http://coq.inria.fr/distrib/V8.11.0/stdlib//Coq.Init.Logic}{\coqdocnotation{\ensuremath{\rightarrow}}} \coqref{NFO.Model.SET}{\coqdocinductive{SET}}) \coqdocvar{e} \coqdocvar{e'} \coqdocvar{s} := \coqdoceol
\coqdocindent{1.00em}
\coqexternalref{::type scope:x '/x5C' x}{http://coq.inria.fr/distrib/V8.11.0/stdlib//Coq.Init.Logic}{\coqdocnotation{(}}\coqdocvariable{s} \BIN \coqdocvariable{e'} \coqexternalref{::type scope:x '->' x}{http://coq.inria.fr/distrib/V8.11.0/stdlib//Coq.Init.Logic}{\coqdocnotation{\ensuremath{\rightarrow}}} \coqdocvariable{s} \BIN \coqdocvariable{e}\coqexternalref{::type scope:x '/x5C' x}{http://coq.inria.fr/distrib/V8.11.0/stdlib//Coq.Init.Logic}{\coqdocnotation{)}} \coqexternalref{::type scope:x '/x5C' x}{http://coq.inria.fr/distrib/V8.11.0/stdlib//Coq.Init.Logic}{\coqdocnotation{\ensuremath{\land}}} \coqexternalref{::type scope:x '/x5C' x}{http://coq.inria.fr/distrib/V8.11.0/stdlib//Coq.Init.Logic}{\coqdocnotation{(}}\coqdocvariable{s} \BIN \coqdocvariable{e} \coqexternalref{::type scope:x '->' x}{http://coq.inria.fr/distrib/V8.11.0/stdlib//Coq.Init.Logic}{\coqdocnotation{\ensuremath{\rightarrow}}}  \coqexternalref{::type scope:x '/x5C' x}{http://coq.inria.fr/distrib/V8.11.0/stdlib//Coq.Init.Logic}{\coqdocnotation{(}}\coqdocvariable{s} \AIN \coqdocvariable{f} \coqexternalref{::type scope:x '<->' x}{http://coq.inria.fr/distrib/V8.11.0/stdlib//Coq.Init.Logic}{\coqdocnotation{\ensuremath{\leftrightarrow}}} \coqdocvariable{s} \BIN \coqdocvariable{e'}\coqexternalref{::type scope:x '->' x}{http://coq.inria.fr/distrib/V8.11.0/stdlib//Coq.Init.Logic}{\coqdocnotation{)}}\coqexternalref{::type scope:x '/x5C' x}{http://coq.inria.fr/distrib/V8.11.0/stdlib//Coq.Init.Logic}{\coqdocnotation{)}}.\coqdoceol
\coqdocemptyline
\coqdocemptyline
\coqdocnoindent
\coqdockw{Lemma} \coqdef{NFO.Union.half respects}{half\_respects}{\coqdoclemma{half\_respects}} \{\coqdocvar{X}\} (\coqdocvar{f}: \coqdocvariable{X} \coqexternalref{::type scope:x '->' x}{http://coq.inria.fr/distrib/V8.11.0/stdlib//Coq.Init.Logic}{\coqdocnotation{\ensuremath{\rightarrow}}} \coqref{NFO.Model.SET}{\coqdocinductive{SET}}) \coqdocvar{e} \coqdocvar{e'}:\coqdoceol
\coqdocindent{1.00em}
\coqref{Internal.FunExt.respects}{\coqdocdefinition{respects}} \coqref{NFO.Eq.EQ}{\coqdocdefinition{EQ}} (\coqref{NFO.Union.half}{\coqdocdefinition{half}} \coqdocvariable{f} \coqdocvariable{e} \coqdocvariable{e'}).\coqdoceol
\end{coqdoccode}
\begin{proof}
  It follows directly from \coqref{NFO.Union.half}{\coqdocdefinition{half}} being defined in terms of \coqdocdefinition{AIN} and \coqdocdefinition{BIN}, which are \coqdocdefinition{EQ}-morphisms.
\end{proof}

'restrictC f P' restricts the domain of a function f according to a predicate P that restricts the codomain:

\begin{coqdoccode}
\coqdocnoindent
\coqdockw{Definition} \coqdef{NFO.Union.restrictC}{restrictC}{\coqdocdefinition{restrictC}} \{\coqdocvar{X} \coqdocvar{Y}\} (\coqdocvar{f}: \coqdocvariable{X} \coqexternalref{::type scope:x '->' x}{http://coq.inria.fr/distrib/V8.11.0/stdlib//Coq.Init.Logic}{\coqdocnotation{\ensuremath{\rightarrow}}} \coqdocvariable{Y}) (\coqdocvar{P}: \coqdocvariable{Y} \coqexternalref{::type scope:x '->' x}{http://coq.inria.fr/distrib/V8.11.0/stdlib//Coq.Init.Logic}{\coqdocnotation{\ensuremath{\rightarrow}}} \coqdockw{Prop})\coqdoceol
\coqdocindent{1.00em}
: \coqexternalref{::type scope:'x7B' x ':' x 'x26' x 'x7D'}{http://coq.inria.fr/distrib/V8.11.0/stdlib//Coq.Init.Specif}{\coqdocnotation{\{}} \coqdocvar{x}\coqexternalref{::type scope:'x7B' x ':' x 'x26' x 'x7D'}{http://coq.inria.fr/distrib/V8.11.0/stdlib//Coq.Init.Specif}{\coqdocnotation{:}} \coqdocvariable{X} \coqexternalref{::type scope:'x7B' x ':' x 'x26' x 'x7D'}{http://coq.inria.fr/distrib/V8.11.0/stdlib//Coq.Init.Specif}{\coqdocnotation{\&}} \coqdocvariable{P} (\coqdocvariable{f} \coqdocvariable{x}) \coqexternalref{::type scope:'x7B' x ':' x 'x26' x 'x7D'}{http://coq.inria.fr/distrib/V8.11.0/stdlib//Coq.Init.Specif}{\coqdocnotation{\}}} \coqexternalref{::type scope:x '->' x}{http://coq.inria.fr/distrib/V8.11.0/stdlib//Coq.Init.Logic}{\coqdocnotation{\ensuremath{\rightarrow}}} \coqdocvariable{Y}\coqdoceol
\coqdocindent{1.00em}
:= \coqdockw{fun} \coqdocvar{x} \ensuremath{\Rightarrow} \coqdocvariable{f} (\coqexternalref{projT1}{http://coq.inria.fr/distrib/V8.11.0/stdlib//Coq.Init.Specif}{\coqdocdefinition{projT1}} \coqdocvariable{x}).\coqdoceol
\coqdocemptyline
\coqdocnoindent
\coqdockw{Definition} \coqdef{NFO.Union.cup}{cup}{\coqdocdefinition{cup}} \coqdocvar{s} \coqdocvar{t} := \coqdoceol
\coqdocindent{1.00em}
\coqdockw{match} \coqdocvariable{s}, \coqdocvariable{t} \coqdockw{with} \coqref{NFO.Model.S}{\coqdocconstructor{S}} \coqdocvar{X} \coqdocvar{Y} \coqdocvar{f} \coqdocvar{g} \coqdocvar{e}, \coqref{NFO.Model.S}{\coqdocconstructor{S}} \coqdocvar{X'} \coqdocvar{Y'} \coqdocvar{f'} \coqdocvar{g'} \coqdocvar{e'} \ensuremath{\Rightarrow}\coqdoceol
\coqdocindent{2.00em}
\coqref{NFO.Model.S}{\coqdocconstructor{S}} \coqdocvar{\_} \coqdocvar{\_}\coqdoceol
\coqdocindent{3.00em}
(   \coqref{NFO.Union.restrictC}{\coqdocdefinition{restrictC}} \coqdocvar{f}  (\coqref{NFO.Union.half}{\coqdocdefinition{half}} \coqdocvar{f'} (\coqref{NFO.BoolExpr.map}{\coqdocdefinition{map}} \coqdocvar{g}  \coqdocvar{e} ) (\coqref{NFO.BoolExpr.map}{\coqdocdefinition{map}} \coqdocvar{g'} \coqdocvar{e'}))\coqdoceol
\coqdocindent{3.50em}
\coqref{Internal.Misc.:::x 'xE2xA8x81' x}{\coqdocnotation{⨁}} \coqref{NFO.Union.restrictC}{\coqdocdefinition{restrictC}} \coqdocvar{f'} (\coqref{NFO.Union.half}{\coqdocdefinition{half}} \coqdocvar{f}  (\coqref{NFO.BoolExpr.map}{\coqdocdefinition{map}} \coqdocvar{g'} \coqdocvar{e'}) (\coqref{NFO.BoolExpr.map}{\coqdocdefinition{map}} \coqdocvar{g}  \coqdocvar{e} )))\coqdoceol
\coqdocindent{3.00em}
(\coqdocvar{g} \coqref{Internal.Misc.:::x 'xE2xA8x81' x}{\coqdocnotation{⨁}} \coqdocvar{g'})\coqdoceol
\coqdocindent{3.00em}
(\coqref{NFO.BoolExpr.Or}{\coqdocconstructor{Or}} (\coqref{NFO.BoolExpr.map}{\coqdocdefinition{map}} \coqexternalref{inl}{http://coq.inria.fr/distrib/V8.11.0/stdlib//Coq.Init.Datatypes}{\coqdocconstructor{inl}} \coqdocvar{e}) (\coqref{NFO.BoolExpr.map}{\coqdocdefinition{map}} \coqexternalref{inr}{http://coq.inria.fr/distrib/V8.11.0/stdlib//Coq.Init.Datatypes}{\coqdocconstructor{inr}} \coqdocvar{e'}))\coqdoceol
\coqdocindent{1.00em}
\coqdockw{end}.\coqdoceol
\coqdocemptyline
\coqdocnoindent
\coqdockw{Theorem} \coqdefRef{NFO.Union.cup ok}{cup\_ok}{\coqdoclemma{cup\_ok}} \coqdocvar{s} \coqdocvar{s'} \coqdocvar{t}: \coqdocvariable{t} \INX (\coqref{NFO.Union.cup}{\coqdocdefinition{cup}} \coqdocvariable{s} \coqdocvariable{s'}) \coqexternalref{::type scope:x '<->' x}{http://coq.inria.fr/distrib/V8.11.0/stdlib//Coq.Init.Logic}{\coqdocnotation{\ensuremath{\leftrightarrow}}} \coqdocvariable{t} \INX \coqdocvariable{s} \coqexternalref{::type scope:x 'x5C/' x}{http://coq.inria.fr/distrib/V8.11.0/stdlib//Coq.Init.Logic}{\coqdocnotation{\ensuremath{\lor}}} \coqdocvariable{t} \INX \coqdocvariable{s'}.\coqdoceol
\end{coqdoccode}
\begin{proof}
  \TODO{}
\end{proof}

\paragraph{Symmetric difference}

Informally: $(a_1 \veebar b_1) \veebar (a_2 \veebar b_2) \approx (a_1 \veebar a_2) \veebar (b_1 \veebar b_2)$ since xor is associative and commutative.
a

\begin{coqdoccode}
    \coqdocemptyline
    \coqdocnoindent
    \coqdockw{Definition} \coqdef{NFO.Sets.AXor}{AXor}{\coqdocdefinition{Axor}} \{\coqdocvar{X} \coqdocvar{Y}\} (\coqdocvar{f}: \coqdocvariable{X} \coqexternalref{::type scope:x '->' x}{http://coq.inria.fr/distrib/V8.11.0/stdlib//Coq.Init.Logic}{\coqdocnotation{\ensuremath{\rightarrow}}} \coqref{NFO.Model.SET}{\coqdocinductive{SET}}) (\coqdocvar{g}: \coqdocvariable{Y} \coqexternalref{::type scope:x '->' x}{http://coq.inria.fr/distrib/V8.11.0/stdlib//Coq.Init.Logic}{\coqdocnotation{\ensuremath{\rightarrow}}} \coqref{NFO.Model.SET}{\coqdocinductive{SET}})\coqdoceol
    \coqdocindent{1.00em}
    : \coqexternalref{::type scope:'x7B' x 'x26' x 'x7D'}{http://coq.inria.fr/distrib/V8.11.0/stdlib//Coq.Init.Specif}{\coqdocnotation{\{}}\coqdocvar{x} \coqexternalref{::type scope:'x7B' x 'x26' x 'x7D'}{http://coq.inria.fr/distrib/V8.11.0/stdlib//Coq.Init.Specif}{\coqdocnotation{\&}} \coqexternalref{::type scope:'x7E' x}{http://coq.inria.fr/distrib/V8.11.0/stdlib//Coq.Init.Logic}{\coqdocnotation{\ensuremath{\lnot}}} \coqexternalref{::type scope:'exists' x '..' x ',' x}{http://coq.inria.fr/distrib/V8.11.0/stdlib//Coq.Init.Logic}{\coqdocnotation{\ensuremath{\exists}}} \coqdocvar{y}\coqexternalref{::type scope:'exists' x '..' x ',' x}{http://coq.inria.fr/distrib/V8.11.0/stdlib//Coq.Init.Logic}{\coqdocnotation{,}} (\coqdocvariable{g} \coqdocvariable{y}) \EQX (\coqdocvariable{f} \coqdocvariable{x})\coqexternalref{::type scope:'x7B' x 'x26' x 'x7D'}{http://coq.inria.fr/distrib/V8.11.0/stdlib//Coq.Init.Specif}{\coqdocnotation{\}}} \coqdoceol
    \coqdocindent{2.00em}
    \coqexternalref{sum}{http://coq.inria.fr/distrib/V8.11.0/stdlib//Coq.Init.Datatypes}{\coqdocinductive{+}} \coqexternalref{::type scope:'x7B' x 'x26' x 'x7D'}{http://coq.inria.fr/distrib/V8.11.0/stdlib//Coq.Init.Specif}{\coqdocnotation{\{}}\coqdocvar{y} \coqexternalref{::type scope:'x7B' x 'x26' x 'x7D'}{http://coq.inria.fr/distrib/V8.11.0/stdlib//Coq.Init.Specif}{\coqdocnotation{\&}} \coqexternalref{::type scope:'x7E' x}{http://coq.inria.fr/distrib/V8.11.0/stdlib//Coq.Init.Logic}{\coqdocnotation{\ensuremath{\lnot}}} \coqexternalref{::type scope:'exists' x '..' x ',' x}{http://coq.inria.fr/distrib/V8.11.0/stdlib//Coq.Init.Logic}{\coqdocnotation{\ensuremath{\exists}}} \coqdocvar{x}\coqexternalref{::type scope:'exists' x '..' x ',' x}{http://coq.inria.fr/distrib/V8.11.0/stdlib//Coq.Init.Logic}{\coqdocnotation{,}} (\coqdocvariable{f} \coqdocvariable{x}) \EQX (\coqdocvariable{g} \coqdocvariable{y})\coqexternalref{::type scope:'x7B' x 'x26' x 'x7D'}{http://coq.inria.fr/distrib/V8.11.0/stdlib//Coq.Init.Specif}{\coqdocnotation{\}}} \coqdoceol
    \coqdocindent{3.00em}
    \coqexternalref{::type scope:x '->' x}{http://coq.inria.fr/distrib/V8.11.0/stdlib//Coq.Init.Logic}{\coqdocnotation{\ensuremath{\rightarrow}}} \coqref{NFO.Model.SET}{\coqdocinductive{SET}}\coqdoceol
    \coqdocindent{1.00em}
    := \coqdockw{fun} \coqdocvar{s} \ensuremath{\Rightarrow} \coqdockw{match} \coqdocvariable{s} \coqdockw{with}\coqdoceol
    \coqdocindent{3.00em}
    \ensuremath{|} \coqexternalref{inl}{http://coq.inria.fr/distrib/V8.11.0/stdlib//Coq.Init.Datatypes}{\coqdocconstructor{inl}} \coqdocvar{x} \ensuremath{\Rightarrow} \coqdocvariable{f} (\coqexternalref{projT1}{http://coq.inria.fr/distrib/V8.11.0/stdlib//Coq.Init.Specif}{\coqdocdefinition{projT1}} \coqdocvar{x})\coqdoceol
    \coqdocindent{3.00em}
    \ensuremath{|} \coqexternalref{inr}{http://coq.inria.fr/distrib/V8.11.0/stdlib//Coq.Init.Datatypes}{\coqdocconstructor{inr}} \coqdocvar{y} \ensuremath{\Rightarrow} \coqdocvariable{g} (\coqexternalref{projT1}{http://coq.inria.fr/distrib/V8.11.0/stdlib//Coq.Init.Specif}{\coqdocdefinition{projT1}} \coqdocvar{y}) \coqdoceol
    \coqdocindent{2.00em}
    \coqdockw{end}.\coqdoceol
    \coqdocnoindent
    \coqdockw{Infix} \coqdef{NFO.Sets.:::x 'x5EAx5E' x}{"}{"}\AXOR" := \coqref{NFO.Sets.AXor}{\coqdocdefinition{Axor}} (\coqdoctac{at} \coqdockw{level} 50).\coqdoceol
    \coqdocemptyline
\end{coqdoccode}

\begin{coqdoccode}
  \coqdocnoindent
\coqdockw{Definition} \coqdef{NFO.Sets.bexpr xor}{bexpr\_xor}{\coqdocdefinition{bexpr\_xor}} \{\coqdocvar{Y}\} (\coqdocvar{e} \coqdocvar{e'}: @\coqref{NFO.BoolExpr.BExpr}{\coqdocinductive{BExpr}} \coqdocvariable{Y}) :=\coqdoceol
\coqdocindent{1.00em}
\coqref{NFO.BoolExpr.Or}{\coqdocconstructor{Or}} (\coqref{NFO.BoolExpr.Not}{\coqdocconstructor{Not}} (\coqref{NFO.BoolExpr.Or}{\coqdocconstructor{Or}} \coqdocvariable{e} (\coqref{NFO.BoolExpr.Not}{\coqdocconstructor{Not}} \coqdocvariable{e'}))) (\coqref{NFO.BoolExpr.Not}{\coqdocconstructor{Not}} (\coqref{NFO.BoolExpr.Or}{\coqdocconstructor{Or}} (\coqref{NFO.BoolExpr.Not}{\coqdocconstructor{Not}} \coqdocvariable{e}) \coqdocvariable{e'})).\coqdoceol
\coqdocemptyline

\coqdocnoindent
\coqdockw{Definition} \coqdefRef{NFO.Sets.QXor}{QXor}{\coqdocdefinition{XOR}} \coqdocvar{s} \coqdocvar{s'} := \coqdoceol
\coqdocindent{1.00em}
\coqdockw{match} \coqdocvariable{s}, \coqdocvariable{s'} \coqdockw{with} \coqref{NFO.Model.S}{\coqdocconstructor{S}} \coqdocvar{X} \coqdocvar{Y} \coqdocvar{f} \coqdocvar{g} \coqdocvar{e}, \coqref{NFO.Model.S}{\coqdocconstructor{S}} \coqdocvar{X'} \coqdocvar{Y'} \coqdocvar{f'} \coqdocvar{g'} \coqdocvar{e'} \ensuremath{\Rightarrow}\coqdoceol
\coqdocindent{2.00em}
\coqref{NFO.Model.S}{\coqdocconstructor{S}} \coqdocvar{\_} \coqdocvar{\_}\coqdoceol
\coqdocindent{3.00em}
(\coqdocvar{f} \AXOR \coqdocvar{f'})\coqdoceol
\coqdocindent{4.00em}
(\coqdocvar{g} \coqref{Internal.Misc.:::x 'xE2xA8x81' x}{\coqdocnotation{⨁}} \coqdocvar{g'}) (\coqref{NFO.Sets.bexpr xor}{\coqdocdefinition{bexpr\_xor}} (\coqref{NFO.BoolExpr.map}{\coqdocdefinition{map}} \coqexternalref{inl}{http://coq.inria.fr/distrib/V8.11.0/stdlib//Coq.Init.Datatypes}{\coqdocconstructor{inl}} \coqdocvar{e}) (\coqref{NFO.BoolExpr.map}{\coqdocdefinition{map}} \coqexternalref{inr}{http://coq.inria.fr/distrib/V8.11.0/stdlib//Coq.Init.Datatypes}{\coqdocconstructor{inr}} \coqdocvar{e'}))\coqdoceol
\coqdocindent{1.00em}
\coqdockw{end}.\coqdoceol
\coqdocnoindent
\coqdockw{Infix} \coqdef{NFO.Sets.:::x 'x5Ex5Ex5E' x}{"}{"}\SXOR" := \coqref{NFO.Sets.QXor}{\coqdocdefinition{XOR}} (\coqdoctac{at} \coqdockw{level} 50).\coqdoceol
\coqdocemptyline
\end{coqdoccode}

\begin{coqdoccode}
  \coqdocnoindent
\coqdockw{Lemma} \coqdef{NFO.Sets.AXor ok}{Axor\_ok}{\coqdoclemma{Axor\_ok}} \{\coqdocvar{X} \coqdocvar{X'}\} \{\coqdocvar{f}: \coqdocvariable{X} \coqexternalref{::type scope:x '->' x}{http://coq.inria.fr/distrib/V8.11.0/stdlib//Coq.Init.Logic}{\coqdocnotation{\ensuremath{\rightarrow}}} \coqref{NFO.Model.SET}{\coqdocinductive{SET}}\} \{\coqdocvar{f'}: \coqdocvariable{X'} \coqexternalref{::type scope:x '->' x}{http://coq.inria.fr/distrib/V8.11.0/stdlib//Coq.Init.Logic}{\coqdocnotation{\ensuremath{\rightarrow}}} \coqref{NFO.Model.SET}{\coqdocinductive{SET}}\} \{\coqdocvar{x}\}:\coqdoceol
\coqdocindent{1.00em}
\coqdocvariable{x} \AIN (\coqdocvariable{f} \AXOR \coqdocvariable{f'}) \coqexternalref{::type scope:x '<->' x}{http://coq.inria.fr/distrib/V8.11.0/stdlib//Coq.Init.Logic}{\coqdocnotation{\ensuremath{\leftrightarrow}}} \coqdocvariable{x} \AIN \coqdocvariable{f} \coqref{NFO.Xor.:::x 'xE2x8AxBB' x}{\coqdocnotation{⊻}} \coqdocvariable{x} \AIN \coqdocvariable{f'}.\coqdoceol
\end{coqdoccode}
\begin{coqdoccode}
  \coqdocnoindent
  \coqdockw{Theorem} \coqdefRef{NFO.Sets.xor ok}{xor\_ok}{\coqdoclemma{XOR\_ok}}: \coqdockw{\ensuremath{\forall}} \coqdocvar{s} \coqdocvar{s'} \coqdocvar{t},\coqdoceol
\coqdocindent{1.00em}
\coqdocvariable{t} \INX (\coqdocvariable{s} \SXOR \coqdocvariable{s'}) \coqexternalref{::type scope:x '<->' x}{http://coq.inria.fr/distrib/V8.11.0/stdlib//Coq.Init.Logic}{\coqdocnotation{\ensuremath{\leftrightarrow}}} \coqdocvariable{t} \INX \coqdocvariable{s} \coqref{NFO.Xor.:::x 'xE2x8AxBB' x}{\coqdocnotation{⊻}} \coqdocvariable{t} \INX \coqdocvariable{s'}.\coqdoceol
\end{coqdoccode}

\TODO{define xor of sets! needed in next subsection.}

\subsection{Extensionality}
%!TEX root = main.tex

To prove extensionality for \NFO{} we follow a strategy similar to the one used for \NFTWO{}, but quite more involved. As usual, one of the directions of the double implication follows from \coqdocdefinition{in\_sound\_right}; the other direction is non-trivial. The idea is similar to the \NFTWO{} case: obtain a contradiction when a low set has the same extension of a high set. However here there is no clear syntactical difference between low sets and high sets, thus a more refined argument is required.

A fundamental lemma is again\coqdoclemma{weak\_regularity}:

\begin{coqdoccode}
  \coqdocnoindent
  \coqdockw{Lemma} \coqdefRef{NFO.Ext.weak regularity}{weak\_regularity}{\coqdoclemma{weak\_regularity}} \coqdocvar{s} :\coqdoceol
  \coqdocindent{1.00em}
  \coqdockw{match} \coqdocvariable{s} \coqdockw{with} \coqref{NFO.Model.S}{\coqdocconstructor{S}} \coqdocvar{\_} \coqdocvar{\_} \coqdocvar{f} \coqdocvar{\_} \coqdocvar{\_} \ensuremath{\Rightarrow} \coqdocvariable{s} \AIN \coqdocvar{f} \coqexternalref{::type scope:x '->' x}{http://coq.inria.fr/distrib/V8.11.0/stdlib//Coq.Init.Logic}{\coqdocnotation{\ensuremath{\rightarrow}}} \coqexternalref{False}{http://coq.inria.fr/distrib/V8.11.0/stdlib//Coq.Init.Logic}{\coqdocinductive{False}} \coqdockw{end}.\coqdoceol
\end{coqdoccode}
\begin{proof}
  By induction on the structure of \coqdocvar{s}.
\end{proof}

% \TODO{The problem: let $A \XOR B$ and $A' \XOR B'$ two \NFO{} sets. } It is not too difficult to prove extensionality separately for A-parts and B-parts of sets:

\begin{coqdoccode}
  \coqdocnoindent
  \coqdockw{Theorem} \coqdefRef{NFO.Morphism.Aext}{Aext}{\coqdoclemma{Aext}} \coqdocvar{X} \coqdocvar{Y} (\coqdocvar{f}: \coqdocvariable{X} \coqexternalref{::type scope:x '->' x}{http://coq.inria.fr/distrib/V8.11.0/stdlib//Coq.Init.Logic}{\coqdocnotation{\ensuremath{\rightarrow}}} \coqdocvar{\_}) (\coqdocvar{f'}: \coqdocvariable{Y} \coqexternalref{::type scope:x '->' x}{http://coq.inria.fr/distrib/V8.11.0/stdlib//Coq.Init.Logic}{\coqdocnotation{\ensuremath{\rightarrow}}} \coqdocvar{\_}) :\coqdoceol
\coqdocindent{1.00em} \coqdocvariable{f} \AEQ \coqdocvariable{f'} \coqexternalref{::type scope:x '<->' x}{http://coq.inria.fr/distrib/V8.11.0/stdlib//Coq.Init.Logic}{\coqdocnotation{\ensuremath{\leftrightarrow}}} \coqdockw{\ensuremath{\forall}} \coqdocvar{s}, \coqdocvariable{s} \AIN \coqdocvariable{f} \coqexternalref{::type scope:x '<->' x}{http://coq.inria.fr/distrib/V8.11.0/stdlib//Coq.Init.Logic}{\coqdocnotation{\ensuremath{\leftrightarrow}}} \coqdocvariable{s} \AIN \coqdocvariable{f'}.\coqdoceol
\end{coqdoccode}


\begin{coqdoccode}
  \coqdocnoindent
  \coqdockw{Theorem} \coqdefRef{NFO.Morphism.Bext}{Bext}{\coqdoclemma{Bext}} \coqdocvar{X} \coqdocvar{Y} (\coqdocvar{g}: \coqdocvariable{X} \coqexternalref{::type scope:x '->' x}{http://coq.inria.fr/distrib/V8.11.0/stdlib//Coq.Init.Logic}{\coqdocnotation{\ensuremath{\rightarrow}}} \coqdocvar{\_}) (\coqdocvar{g'}: \coqdocvariable{Y} \coqexternalref{::type scope:x '->' x}{http://coq.inria.fr/distrib/V8.11.0/stdlib//Coq.Init.Logic}{\coqdocnotation{\ensuremath{\rightarrow}}} \coqdocvar{\_}) \coqdocvar{e} \coqdocvar{e'} :\coqdoceol
\coqdocindent{1.00em}
\coqref{NFO.BoolExpr.map}{\coqdocdefinition{map}} \coqdocvariable{g} \coqdocvariable{e} \BEQ \coqref{NFO.BoolExpr.map}{\coqdocdefinition{map}} \coqdocvariable{g'} \coqdocvariable{e'}\coqdoceol
\coqdocindent{1.00em}
\coqexternalref{::type scope:x '<->' x}{http://coq.inria.fr/distrib/V8.11.0/stdlib//Coq.Init.Logic}{\coqdocnotation{\ensuremath{\leftrightarrow}}} \coqdockw{\ensuremath{\forall}} \coqdocvar{s}, \coqdocvariable{s} \BIN \coqref{NFO.BoolExpr.map}{\coqdocdefinition{map}} \coqdocvariable{g} \coqdocvariable{e} \coqexternalref{::type scope:x '<->' x}{http://coq.inria.fr/distrib/V8.11.0/stdlib//Coq.Init.Logic}{\coqdocnotation{\ensuremath{\leftrightarrow}}} \coqdocvariable{s} \BIN \coqref{NFO.BoolExpr.map}{\coqdocdefinition{map}} \coqdocvariable{g'} \coqdocvariable{e'}.\coqdoceol
\end{coqdoccode}

 Set with empty extension:
 
 \begin{coqdoccode}
  \coqdocnoindent
  \coqdockw{Definition} \coqdef{NFO.Ext.ext empty}{ext\_empty}{\coqdocdefinition{ext\_empty}} \coqdocvar{s} := \coqdockw{\ensuremath{\forall}} \coqdocvar{t}, \coqexternalref{::type scope:'x7E' x}{http://coq.inria.fr/distrib/V8.11.0/stdlib//Coq.Init.Logic}{\coqdocnotation{\ensuremath{\lnot}}} \coqref{NFO.In.IN}{\coqdocdefinition{IN}} \coqdocvariable{t} \coqdocvariable{s}.\coqdoceol
  \coqdocemptyline
\end{coqdoccode}

  Two sets have the same extension iff their simmetric difference has empty extension:

  \begin{coqdoccode}
  \coqdocnoindent
  \coqdockw{Lemma} \coqdefRef{NFO.Ext.xor ext}{xor\_ext}{\coqdoclemma{xor\_ext}}: \coqdockw{\ensuremath{\forall}} \{\coqdocvar{s} \coqdocvar{s'}\},\coqdoceol
  \coqdocindent{1.00em}
  \coqexternalref{::type scope:x '<->' x}{http://coq.inria.fr/distrib/V8.11.0/stdlib//Coq.Init.Logic}{\coqdocnotation{(}}\coqdockw{\ensuremath{\forall}} \coqdocvar{t}, \coqdocvariable{t} \INX \coqdocvariable{s} \coqexternalref{::type scope:x '<->' x}{http://coq.inria.fr/distrib/V8.11.0/stdlib//Coq.Init.Logic}{\coqdocnotation{\ensuremath{\leftrightarrow}}} \coqdocvariable{t} \INX \coqdocvariable{s'}\coqexternalref{::type scope:x '<->' x}{http://coq.inria.fr/distrib/V8.11.0/stdlib//Coq.Init.Logic}{\coqdocnotation{)}} \coqexternalref{::type scope:x '<->' x}{http://coq.inria.fr/distrib/V8.11.0/stdlib//Coq.Init.Logic}{\coqdocnotation{\ensuremath{\leftrightarrow}}} \coqref{NFO.Ext.ext empty}{\coqdocdefinition{ext\_empty}} (\coqdocvariable{s} \SXOR \coqdocvariable{s'}).\coqdoceol
  \end{coqdoccode}

Then, the symmetric difference of two sets is the empty set iff the two sets are equivalent:

\begin{coqdoccode}
  \coqdocnoindent
  \coqdockw{Theorem} \coqdefRef{NFO.Sets.xor empty}{xor\_empty}{\coqdoclemma{xor\_empty}}: \coqdockw{\ensuremath{\forall}} \coqdocvar{s} \coqdocvar{s'}, \coqref{NFO.Eq.::type scope:x '==' x}{\coqdocnotation{(}}\coqdocvariable{s}  \SXOR \coqdocvariable{s'}\coqref{NFO.Eq.::type scope:x '==' x}{\coqdocnotation{)}} \EQX \coqref{NFO.Sets.emptyset}{\coqdocdefinition{emptyset}} \coqexternalref{::type scope:x '->' x}{http://coq.inria.fr/distrib/V8.11.0/stdlib//Coq.Init.Logic}{\coqdocnotation{\ensuremath{\rightarrow}}} \coqdocvariable{s} \EQX \coqdocvariable{s'}.\coqdoceol
\end{coqdoccode}
\begin{proof}
  Follows from the definition of \XORS{} and by lemmas \coqref{NFO.Morphism.Aext}{\coqdoclemma{Aext}} and \coqref{NFO.Morphism.Bext}{\coqdoclemma{Bext}}
\end{proof}

It remains to prove is that \coqref{NFO.Ext.ext empty}{\coqdocdefinition{ext\_empty}} \coqdocvariable{s} implies \coqdocvariable{s} \EQX \coqref{NFO.Sets.emptyset}{\coqdocdefinition{emptyset}} (which will only be proved in \coqref{NFO.Ext.no urelements}{\coqdoclemma{no\_urelements}} at the end of the section).
%  The difficulty is the following:

  % TODO: define ``urelement'' in the article (a set where the A-part has the same extension as the B-part, i.e. it's extensionally empty, but it's different from the empty set, i.e. has non-empty (and with same extension) parts )


  

\paragraph{Sloppy membership.}
B-sets have a peculiar property which makes us call them \emph{sloppy}: unlike A-sets, a B-set cannot be required to contain exactly some precise, specified elements. By construction, the boolean expression defining a B-set states whether the elements of the B-set should or should not contain the sets in the codomain of the indexing g; a bexpr however does not mention any set outside of the codomain of g.

We call informally \coqdocvar{g}-\emph{signature} of a set \coqdocvar{s} (with respect to an indexing \coqdocvar{g}) the sets in the image of \coqdocvar{g} which are also elements of \coqdocvar{s}. The signature is a fundamental concept concerning B-sets: a B-set defined by an indexing \coqdocvar{g} and a bexrp \var{e} cannot distinguish between sets having the same \coqdocvar{g}-\emph{signature}. In other words, when a B-set contains a set of a given signature, then it must contain all sets having the same signature.

This property is formalized in the following lemma:
   
\begin{coqdoccode}
  \coqdocnoindent
  \coqdockw{Lemma} \coqdefRef{NFO.Ext.sloppy Bext}{sloppy\_Bext}{\coqdoclemma{sloppy\_Bext}} \{\coqdocvar{Y}\} (\coqdocvar{g}: \coqdocvariable{Y} \coqexternalref{::type scope:x '->' x}{http://coq.inria.fr/distrib/V8.11.0/stdlib//Coq.Init.Logic}{\coqdocnotation{\ensuremath{\rightarrow}}} \coqref{NFO.Model.SET}{\coqdocinductive{SET}}) \coqdocvar{s} \coqdocvar{s'} \coqdocvar{e}:\coqdoceol
  \coqdocindent{1.00em}
  \coqexternalref{::type scope:x '->' x}{http://coq.inria.fr/distrib/V8.11.0/stdlib//Coq.Init.Logic}{\coqdocnotation{(}}\coqdockw{\ensuremath{\forall}} \coqdocvar{y}, (\coqdocvariable{g} \coqdocvariable{y}) \INX \coqdocvariable{s} \coqexternalref{::type scope:x '<->' x}{http://coq.inria.fr/distrib/V8.11.0/stdlib//Coq.Init.Logic}{\coqdocnotation{\ensuremath{\leftrightarrow}}}  (\coqdocvariable{g} \coqdocvariable{y}) \INX \coqdocvariable{s'}\coqexternalref{::type scope:x '->' x}{http://coq.inria.fr/distrib/V8.11.0/stdlib//Coq.Init.Logic}{\coqdocnotation{)}}\coqdoceol
  \coqdocindent{2.00em}
  \coqexternalref{::type scope:x '->' x}{http://coq.inria.fr/distrib/V8.11.0/stdlib//Coq.Init.Logic}{\coqdocnotation{\ensuremath{\rightarrow}}}  \coqdocvariable{s} \BIN(\coqref{NFO.BoolExpr.map}{\coqdocdefinition{map}} \coqdocvariable{g} \coqdocvariable{e}) \coqexternalref{::type scope:x '->' x}{http://coq.inria.fr/distrib/V8.11.0/stdlib//Coq.Init.Logic}{\coqdocnotation{\ensuremath{\rightarrow}}} \coqdocvariable{s'} \BIN (\coqref{NFO.BoolExpr.map}{\coqdocdefinition{map}} \coqdocvariable{g} \coqdocvariable{e}).\coqdoceol
\end{coqdoccode}
\begin{proof}
  By induction on the structure of \coqdocvariable{e}.
\end{proof}

% \TODO{If we assuming the existence of urelements, then \dots and we are in the case where the A-part has the same extension of the B-part. Which is contradictory. Let us assume by contradiction. }

In case a A-set has the same extension of a B-set, the A-set satisfies a property corresponding to \coqref{NFO.Ext.sloppy Bext}{\coqdoclemma{sloppy\_Bext}}: it cannot distinguish between sets having the same \coqdocvar{g}-signature.

\begin{coqdoccode}
  \coqdocnoindent
  \coqdockw{Lemma} \coqdefRef{NFO.Ext.sloppy Aext}{sloppy\_Aext}{\coqdoclemma{sloppy\_Aext}} \{\coqdocvar{X} \coqdocvar{Y}\} (\coqdocvar{f}: \coqdocvariable{X} \coqexternalref{::type scope:x '->' x}{http://coq.inria.fr/distrib/V8.11.0/stdlib//Coq.Init.Logic}{\coqdocnotation{\ensuremath{\rightarrow}}} \coqref{NFO.Model.SET}{\coqdocinductive{SET}}) (\coqdocvar{g}: \coqdocvariable{Y} \coqexternalref{::type scope:x '->' x}{http://coq.inria.fr/distrib/V8.11.0/stdlib//Coq.Init.Logic}{\coqdocnotation{\ensuremath{\rightarrow}}} \coqref{NFO.Model.SET}{\coqdocinductive{SET}}) \coqdocvar{s} \coqdocvar{s'} \coqdocvar{e}:\coqdoceol
  \coqdocindent{1.00em}
  \coqexternalref{::type scope:x '->' x}{http://coq.inria.fr/distrib/V8.11.0/stdlib//Coq.Init.Logic}{\coqdocnotation{(}}\coqdockw{\ensuremath{\forall}} \coqdocvar{t}, \coqdocvariable{t} \AIN \coqdocvariable{f} \coqexternalref{::type scope:x '<->' x}{http://coq.inria.fr/distrib/V8.11.0/stdlib//Coq.Init.Logic}{\coqdocnotation{\ensuremath{\leftrightarrow}}} \coqdocvariable{t} \BIN (\coqref{NFO.BoolExpr.map}{\coqdocdefinition{map}} \coqdocvariable{g} \coqdocvariable{e})\coqexternalref{::type scope:x '->' x}{http://coq.inria.fr/distrib/V8.11.0/stdlib//Coq.Init.Logic}{\coqdocnotation{)}}\coqdoceol
  \coqdocindent{1.00em}
  \coqexternalref{::type scope:x '->' x}{http://coq.inria.fr/distrib/V8.11.0/stdlib//Coq.Init.Logic}{\coqdocnotation{\ensuremath{\rightarrow}}} \coqexternalref{::type scope:x '->' x}{http://coq.inria.fr/distrib/V8.11.0/stdlib//Coq.Init.Logic}{\coqdocnotation{(}}\coqdockw{\ensuremath{\forall}} \coqdocvar{y}, (\coqdocvariable{g} \coqdocvariable{y}) \INX \coqdocvariable{s} \coqexternalref{::type scope:x '<->' x}{http://coq.inria.fr/distrib/V8.11.0/stdlib//Coq.Init.Logic}{\coqdocnotation{\ensuremath{\leftrightarrow}}} (\coqdocvariable{g} \coqdocvariable{y}) \INX \coqdocvariable{s'}\coqexternalref{::type scope:x '->' x}{http://coq.inria.fr/distrib/V8.11.0/stdlib//Coq.Init.Logic}{\coqdocnotation{)}}\coqdoceol
  \coqdocindent{1.00em}
  \coqexternalref{::type scope:x '->' x}{http://coq.inria.fr/distrib/V8.11.0/stdlib//Coq.Init.Logic}{\coqdocnotation{\ensuremath{\rightarrow}}}  \coqdocvariable{s} \AIN \coqdocvariable{f} \coqexternalref{::type scope:x '->' x}{http://coq.inria.fr/distrib/V8.11.0/stdlib//Coq.Init.Logic}{\coqdocnotation{\ensuremath{\rightarrow}}} \coqdocvariable{s'} \AIN \coqdocvariable{f}.\coqdoceol
\end{coqdoccode}
\begin{proof}
  By \coqref{NFO.Ext.sloppy Bext}{\coqdoclemma{sloppy\_Bext}}.
\end{proof}

Like for \NFTWO, we can construct a universal low set, \ie{} a universal \A-set in this case:

\begin{coqdoccode}
  \coqdocnoindent
  \coqdockw{Definition} \coqdefRef{NFO.Ext.X univ}{X\_univ}{\coqdocdefinition{X\_univ}} \{\coqdocvar{Y}\} \coqdocvar{X} (\coqdocvar{g}: \coqdocvariable{Y} \coqexternalref{::type scope:x '->' x}{http://coq.inria.fr/distrib/V8.11.0/stdlib//Coq.Init.Logic}{\coqdocnotation{\ensuremath{\rightarrow}}} \coqref{NFO.Model.SET}{\coqdocinductive{SET}}) :=\coqdoceol
  \coqdocindent{1.00em}
  \coqexternalref{prod}{http://coq.inria.fr/distrib/V8.11.0/stdlib//Coq.Init.Datatypes}{\coqdocinductive{prod}} \coqdocvariable{X} \coqexternalref{::type scope:'x7B' x ':' x 'x26' x 'x7D'}{http://coq.inria.fr/distrib/V8.11.0/stdlib//Coq.Init.Specif}{\coqdocnotation{\{}} \sig{} \coqexternalref{::type scope:'x7B' x ':' x 'x26' x 'x7D'}{http://coq.inria.fr/distrib/V8.11.0/stdlib//Coq.Init.Specif}{\coqdocnotation{:}} \coqdocvariable{Y} \coqexternalref{::type scope:x '->' x}{http://coq.inria.fr/distrib/V8.11.0/stdlib//Coq.Init.Logic}{\coqdocnotation{\ensuremath{\rightarrow}}} \coqdockw{Prop} \coqexternalref{::type scope:'x7B' x ':' x 'x26' x 'x7D'}{http://coq.inria.fr/distrib/V8.11.0/stdlib//Coq.Init.Specif}{\coqdocnotation{\&}} \coqref{Internal.FunExt.respects}{\coqdocdefinition{respects}} (\coqref{NFO.Eq.EQ}{\coqdocdefinition{EQ}} \coqref{Internal.Misc.:::x 'xE2xA8x80' x}{\coqdocnotation{⨀}} \coqdocvariable{g}) \sig{} \coqexternalref{::type scope:'x7B' x ':' x 'x26' x 'x7D'}{http://coq.inria.fr/distrib/V8.11.0/stdlib//Coq.Init.Specif}{\coqdocnotation{\}}}.\coqdoceol
  \coqdocemptyline
  \coqdocnoindent
  \coqdockw{Definition} \coqdefRef{NFO.Ext.f univ}{f\_univ}{\coqdocdefinition{f\_univ}} \{\coqdocvar{X} \coqdocvar{Y}\} (\coqdocvar{f}: \coqdocvariable{X} \coqexternalref{::type scope:x '->' x}{http://coq.inria.fr/distrib/V8.11.0/stdlib//Coq.Init.Logic}{\coqdocnotation{\ensuremath{\rightarrow}}} \coqref{NFO.Model.SET}{\coqdocinductive{SET}}) (\coqdocvar{g}: \coqdocvariable{Y} \coqexternalref{::type scope:x '->' x}{http://coq.inria.fr/distrib/V8.11.0/stdlib//Coq.Init.Logic}{\coqdocnotation{\ensuremath{\rightarrow}}} \coqref{NFO.Model.SET}{\coqdocinductive{SET}})\coqdoceol
  \coqdocindent{1.00em}
  : \coqref{NFO.Ext.X univ}{\coqdocdefinition{X\_univ}} \coqdocvariable{X} \coqdocvariable{g} \coqexternalref{::type scope:x '->' x}{http://coq.inria.fr/distrib/V8.11.0/stdlib//Coq.Init.Logic}{\coqdocnotation{\ensuremath{\rightarrow}}} \coqref{NFO.Model.SET}{\coqdocinductive{SET}} :=\coqdoceol
  \coqdocindent{1.00em}
  \coqdockw{fun} \coqdocvar{xu} \ensuremath{\Rightarrow} \coqdockw{match} \coqdocvariable{xu} \coqdockw{with} \coqexternalref{::core scope:'(' x ',' x ',' '..' ',' x ')'}{http://coq.inria.fr/distrib/V8.11.0/stdlib//Coq.Init.Datatypes}{\coqdocnotation{(}}\coqdocvar{x}\coqexternalref{::core scope:'(' x ',' x ',' '..' ',' x ')'}{http://coq.inria.fr/distrib/V8.11.0/stdlib//Coq.Init.Datatypes}{\coqdocnotation{,}} \coqexternalref{existT}{http://coq.inria.fr/distrib/V8.11.0/stdlib//Coq.Init.Specif}{\coqdocconstructor{existT}} \coqdocvar{\_} \sig{} \coqdocvar{\_}\coqexternalref{::core scope:'(' x ',' x ',' '..' ',' x ')'}{http://coq.inria.fr/distrib/V8.11.0/stdlib//Coq.Init.Datatypes}{\coqdocnotation{)}} \ensuremath{\Rightarrow}\coqdoceol
  \coqdocindent{2.00em}
  \coqdockw{match} \coqdocvariable{f} \coqdocvar{x} \coqdockw{with}\coqdoceol
  \coqdocindent{2.00em}
  \ensuremath{|} \coqref{NFO.Model.S}{\coqdocconstructor{S}} \coqdocvar{\_} \coqdocvar{\_} \coqdocvar{f'} \coqdocvar{g'} \coqdocvar{e'} \ensuremath{\Rightarrow}  \coqref{NFO.Model.S}{\coqdocconstructor{S}} \coqdocvar{\_} \coqdocvar{\_} (\coqdocvar{f'} \AXOR \coqref{Internal.Misc.select}{\coqdocdefinition{select}} \coqdocvariable{g} \sig{}) \coqdocvar{g'} \coqdocvar{e'}\coqdoceol
  \coqdocindent{2.00em}
  \coqdockw{end}\coqdoceol
  \coqdocindent{1.00em}
  \coqdockw{end}.\coqdoceol
  \coqdocemptyline
\end{coqdoccode}

The fundamental result is:

\begin{coqdoccode}
  \coqdocnoindent
  \coqdockw{Lemma} \coqdefRef{NFO.Ext.universal low}{universal\_low}{\coqdoclemma{universal\_low}} \{\coqdocvar{X} \coqdocvar{Y}\} \{\coqdocvar{f}: \coqdocvariable{X} \coqexternalref{::type scope:x '->' x}{http://coq.inria.fr/distrib/V8.11.0/stdlib//Coq.Init.Logic}{\coqdocnotation{\ensuremath{\rightarrow}}} \coqref{NFO.Model.SET}{\coqdocinductive{SET}}\} (\coqdocvar{g}: \coqdocvariable{Y} \coqexternalref{::type scope:x '->' x}{http://coq.inria.fr/distrib/V8.11.0/stdlib//Coq.Init.Logic}{\coqdocnotation{\ensuremath{\rightarrow}}} \coqref{NFO.Model.SET}{\coqdocinductive{SET}}) \coqdocvar{e}:\coqdoceol
  \coqdocindent{1.00em}
  \coqexternalref{::type scope:x '->' x}{http://coq.inria.fr/distrib/V8.11.0/stdlib//Coq.Init.Logic}{\coqdocnotation{(}}\coqdockw{\ensuremath{\forall}} \coqdocvar{s}, \coqdocvariable{s} \AIN \coqdocvariable{f} \coqexternalref{::type scope:x '<->' x}{http://coq.inria.fr/distrib/V8.11.0/stdlib//Coq.Init.Logic}{\coqdocnotation{\ensuremath{\leftrightarrow}}} \coqdocvariable{s} \BIN (\coqref{NFO.BoolExpr.map}{\coqdocdefinition{map}} \coqdocvariable{g} \coqdocvariable{e})\coqexternalref{::type scope:x '->' x}{http://coq.inria.fr/distrib/V8.11.0/stdlib//Coq.Init.Logic}{\coqdocnotation{)}}\coqdoceol
  \coqdocindent{1.00em}
  \coqexternalref{::type scope:x '->' x}{http://coq.inria.fr/distrib/V8.11.0/stdlib//Coq.Init.Logic}{\coqdocnotation{\ensuremath{\rightarrow}}} \coqexternalref{::type scope:x '->' x}{http://coq.inria.fr/distrib/V8.11.0/stdlib//Coq.Init.Logic}{\coqdocnotation{(}}\coqexternalref{::type scope:'exists' x '..' x ',' x}{http://coq.inria.fr/distrib/V8.11.0/stdlib//Coq.Init.Logic}{\coqdocnotation{\ensuremath{\exists}}} \coqdocvar{s}\coqexternalref{::type scope:'exists' x '..' x ',' x}{http://coq.inria.fr/distrib/V8.11.0/stdlib//Coq.Init.Logic}{\coqdocnotation{,}} \coqdocvariable{s} \AIN \coqdocvariable{f}\coqexternalref{::type scope:x '->' x}{http://coq.inria.fr/distrib/V8.11.0/stdlib//Coq.Init.Logic}{\coqdocnotation{)}}\coqdoceol
  \coqdocindent{1.00em}
  \coqexternalref{::type scope:x '->' x}{http://coq.inria.fr/distrib/V8.11.0/stdlib//Coq.Init.Logic}{\coqdocnotation{\ensuremath{\rightarrow}}} \coqdockw{\ensuremath{\forall}} \coqdocvar{s}, \coqdocvariable{s} \AIN (\coqref{NFO.Ext.f univ}{\coqdocdefinition{f\_univ}} \coqdocvariable{f} \coqdocvariable{g}).\coqdoceol
\end{coqdoccode}
\begin{proof}
  % \TODO{ IMPORTANT! }
% 
  Let us assume a set \var t such that \var t \AIN \coqdocvariable{f}, and let \var s = \coqref{NFO.Model.S}{\coqdocconstructor{S}} \coqdocvar{\_} \coqdocvar{\_} \coqdocvar{f'} \coqdocvar{g'} \coqdocvar{e'} be any set.
% 
  % By the first hypothesis, it follows that \var t \BIN \coqref{NFO.BoolExpr.map}{\coqdocdefinition{map}} \coqdocvariable{g} \coqdocvariable{e}.

  
  To show that \coqdocvariable{s} \AIN (\coqref{NFO.Ext.f univ}{\coqdocdefinition{f\_univ}} \coqdocvariable{f} \coqdocvariable{g}), we need to provide an index \var u: \coqref{NFO.Ext.X univ}{\coqdocdefinition{X\_univ}} \var X \var Y such that \var s \EQX \coqref{NFO.Ext.f univ}{\coqdocdefinition{f\_univ}} \var f \var g \var u. The index \var u should be a \var x: \var X together with a \var g-signature $\sigma$: then, it must hold \coqref{NFO.Model.S}{\coqdocconstructor{S}} \coqdocvar{\_} \coqdocvar{\_} \coqdocvar{f'} \coqdocvar{g'} \coqdocvar{e'} \EQX \coqref{NFO.Model.S}{\coqdocconstructor{S}} \coqdocvar{\_} \coqdocvar{\_} (\coqdocvar{f'} \AXOR \coqref{Internal.Misc.select}{\coqdocdefinition{select}} \coqdocvariable{g} \sig{}) \coqdocvar{g'} \coqdocvar{e'}

  \newcommand\stilde{\var{$\tilde s$}}
  Let \sig s be the \var g-signature of \var s and \sig t be the \var g-signature of \var t.

  The idea is to alter the signature of \var s, obtaining a variant \stilde{} of \var s that belongs to the image of \var f. We obtain \stilde{} by simply applying to \var s the signature \sig s \XORP \sig t, the xor of the two signatures: in this way, \stilde{} has the same \var g-signature of \var t, and thus \AIN \var f (by \coqref{NFO.Ext.sloppy Aext}{\coqdoclemma{sloppy\_Aext}}). 

  % pose (S _ _ (AXor f0 (select g sig_xor)) g0 e0) as s_signed.
  % cut (forall y, IN (g y) t <-> IN (g y) s_signed). intro H1.
\end{proof}

% \TODO{define \XORP}

% a

\begin{coqdoccode}
  \coqdocnoindent
  \coqdockw{Lemma} \coqdefRef{NFO.Ext.not IN not AIN}{not\_IN\_not\_AIN}{\coqdoclemma{not\_IN\_not\_AIN}} \{\coqdocvar{X} \coqdocvar{Y} \coqdocvar{f} \coqdocvar{g} \coqdocvar{e}\}:\coqdoceol
  \coqdocindent{1.00em}
  \coqref{NFO.Ext.ext empty}{\coqdocdefinition{ext\_empty}} (\coqref{NFO.Model.S}{\coqdocconstructor{S}} \coqdocvariable{X} \coqdocvariable{Y} \coqdocvariable{f} \coqdocvariable{g} \coqdocvariable{e}) \coqexternalref{::type scope:x '->' x}{http://coq.inria.fr/distrib/V8.11.0/stdlib//Coq.Init.Logic}{\coqdocnotation{\ensuremath{\rightarrow}}} \coqdockw{\ensuremath{\forall}} \coqdocvar{s}, \coqexternalref{::type scope:'x7E' x}{http://coq.inria.fr/distrib/V8.11.0/stdlib//Coq.Init.Logic}{\coqdocnotation{\ensuremath{\lnot}}}  \coqdocvariable{s} \AIN \coqdocvariable{f}.\coqdoceol
\end{coqdoccode}
\begin{proof}
  By \coqref{NFO.Ext.weak regularity}{\coqdoclemma{weak\_regularity}} and \coqref{NFO.Ext.universal low}{\coqdoclemma{universal\_low}}.
\end{proof}

\begin{coqdoccode}
  \coqdocnoindent
  \coqdockw{Lemma} \coqdefRef{NFO.Ext.no urelements}{no\_urelements}{\coqdoclemma{no\_urelements}}: \coqdockw{\ensuremath{\forall}} \coqdocvar{s}, \coqref{NFO.Ext.ext empty}{\coqdocdefinition{ext\_empty}} \coqdocvariable{s} \coqexternalref{::type scope:x '->' x}{http://coq.inria.fr/distrib/V8.11.0/stdlib//Coq.Init.Logic}{\coqdocnotation{\ensuremath{\rightarrow}}} \coqdocvariable{s} \EQX \coqref{NFO.Sets.emptyset}{\coqdocdefinition{emptyset}}.\coqdoceol
\end{coqdoccode}
\begin{proof}
  By \coqref{NFO.Ext.not IN not AIN}{\coqdoclemma{not\_IN\_not\_AIN}} it follows that the A-part of \var s has empty extension; this, combined with the hypothesis that \var s has empty extension, easily implies that also the B-part of \var s must have empty extension. To conclude, just use \coqref{NFO.Morphism.Aext}{\coqdoclemma{Aext}} and \coqref{NFO.Morphism.Bext}{\coqdoclemma{Bext}}.
\end{proof}

  We can finally prove extensionality of NFO sets:

  \begin{coqdoccode}
  \coqdocnoindent
  \coqdockw{Theorem} \coqdefRef{NFO.Ext.ext}{ext}{\coqdoclemma{ext}}:\coqdoceol
  \coqdocindent{1.00em}
  \coqdockw{\ensuremath{\forall}} \coqdocvar{s} \coqdocvar{s'}, \coqdocvariable{s} \EQX \coqdocvariable{s'} \coqexternalref{::type scope:x '<->' x}{http://coq.inria.fr/distrib/V8.11.0/stdlib//Coq.Init.Logic}{\coqdocnotation{\ensuremath{\leftrightarrow}}} \coqdockw{\ensuremath{\forall}} \coqdocvar{t}, \coqdocvariable{t} \INX \coqdocvariable{s} \coqexternalref{::type scope:x '<->' x}{http://coq.inria.fr/distrib/V8.11.0/stdlib//Coq.Init.Logic}{\coqdocnotation{\ensuremath{\leftrightarrow}}} \coqdocvariable{t} \INX \coqdocvariable{s'}.\coqdoceol
\end{coqdoccode}
\begin{proof}
  Follows directly from lemmas 
  \coqref{NFO.Ext.xor ext}{\coqdoclemma{xor\_ext}},
  \coqref{NFO.Ext.no urelements}{\coqdoclemma{no\_urelements}}, and
  \coqref{NFO.Sets.xor empty}{\coqdoclemma{xor\_empty}}.
\end{proof}



\section{Discussion}
\subsection{Related Work}
\subsection{Future Work}



%%
%% The acknowledgments section is defined using the "acks" environment
%% (and NOT an unnumbered section). This ensures the proper
%% identification of the section in the article metadata, and the
%% consistent spelling of the heading.
% \begin{acks}
% To Robert, for the bagels and explaining CMYK and color spaces.
% \end{acks}

%%
%% The next two lines define the bibliography style to be used, and
%% the bibliography file.
\bibliographystyle{ACM-Reference-Format}
\bibliography{main}

%%
%% If your work has an appendix, this is the place to put it.
\appendix


\end{document}
\endinput
