\documentclass[sigplan,10pt,anonymous,review]{acmart}\settopmatter{printfolios=true,printccs=false,printacmref=false}

% !TEX root = main.tex

\usepackage[color]{coqdoc}
\usepackage{amsmath,amssymb}
\usepackage{bbold}
\usepackage{hyperref}
\usepackage{marginnote}
\usepackage{xcolor}
\usepackage{stmaryrd}
\usepackage{graphicx}
\usepackage{ulem}
\normalem
% \usepackage{fancyvrb}

\newcommand\margincoq[1]{\marginpar{\rotatebox{270}{\footnotesize $\star$ [\coqid{#1}]}}}
\newcommand{\coqdefRef}[3]{\margincoq{#1}\coqdef{#1}{#2}{#3}}
\newcommand{\coqid}[1]{\texttt{#1}}

\newcommand\AEQ{\equiv_A}

% AUX
\newcommand\TODO[1]{\textcolor{red}{{\bf TODO} #1}}

\newcommand\Placeholder{\textcolor{gray}{\bullet}}

\DeclareUnicodeCharacter{1D4E5}{\ensuremath{\mathcal{V}}}
\DeclareUnicodeCharacter{2261}{\ensuremath{\equiv}}
\DeclareUnicodeCharacter{27E6}{\ensuremath{\llbracket}}
\DeclareUnicodeCharacter{27E7}{\ensuremath{\rrbracket}}
\DeclareUnicodeCharacter{2218}{\ensuremath{\circ}}
\DeclareUnicodeCharacter{2A01}{\ensuremath{\oplus}}
\DeclareUnicodeCharacter{2A00}{\ensuremath{\odot}}

% LOGIC
\newcommand\grameq{::=}
\newcommand\defeq{:=}
\newcommand\ie{\textit{i.e.}}
\newcommand\fv[1]{\mathrm{fv}(#1)}

% NF
\newcommand\ZF{{\sf ZF}}
\newcommand\NF{{\sf NF}}
\newcommand\NFTWO{{\tt NF2}}
\newcommand\NFO{{\tt NFO}}
\newcommand\Coq{{\tt Coq}}
\newcommand\SetCompr{{\textsc{SetCompr}}}
\newcommand\SetExt{{\textsc{SetExt}}}

\newcommand\sctag[1]{\tag{\textsc{#1}}}

\newcommand\Essence[1]{\Verb{Ess}\,#1}
% \newcommand\Sing[1]{\Verb{\{}#1\Verb{\}}}
% \newcommand\Sing[1]{\Verb{Sing}\,#1}
\newcommand\Sing[1]{\Verb{\{} #1 \Verb{\}}}
\newcommand\Complement[1]{\overline{#1}}
% \newcommand\Complement[1]{\Verb{Compl}\,#1}
\newcommand\Char[1]{\Verb{Char}_{#1}} % \mathbb{1}
\newcommand\Compr[2]{\Verb{\{} #1 ~\Verb{|}~ #2 \Verb{\}}}
\newcommand\Universe{U}
\newcommand\Verb[1]{\text{\texttt{#1}}}
% \newcommand\Union[2]{#1 \mathrel{\Verb{U}} #2}
\newcommand\Union[2]{#1 \cup #2}

% COQ DOC
\newcommand\Definition{\coqdockw{Definition}}
\newcommand\Math[1]{\coqdocnotation{\ensuremath{#1}}}
\newcommand\Var\coqdocvariable


%%
%% \BibTeX command to typeset BibTeX logo in the docs
\AtBeginDocument{%
  \providecommand\BibTeX{{%
    \normalfont B\kern-0.5em{\scshape i\kern-0.25em b}\kern-0.8em\TeX}}}

%% Rights management information.  This information is sent to you
%% when you complete the rights form.  These commands have SAMPLE
%% values in them; it is your responsibility as an author to replace
%% the commands and values with those provided to you when you
%% complete the rights form.
\setcopyright{acmcopyright}
\copyrightyear{2018}
\acmYear{2018}
\acmDOI{10.1145/1122445.1122456}

%% These commands are for a PROCEEDINGS abstract or paper.
\acmConference[Woodstock '18]{Woodstock '18: ACM Symposium on Neural
  Gaze Detection}{June 03--05, 2018}{Woodstock, NY}
\acmBooktitle{Woodstock '18: ACM Symposium on Neural Gaze Detection,
  June 03--05, 2018, Woodstock, NY}
\acmPrice{15.00}
\acmISBN{978-1-4503-XXXX-X/18/06}


%%
%% Submission ID.
%% Use this when submitting an article to a sponsored event. You'll
%% receive a unique submission ID from the organizers
%% of the event, and this ID should be used as the parameter to this command.
%%\acmSubmissionID{123-A56-BU3}

%%
%% The majority of ACM publications use numbered citations and
%% references.  The command \citestyle{authoryear} switches to the
%% "author year" style.
%%
%% If you are preparing content for an event
%% sponsored by ACM SIGGRAPH, you must use the "author year" style of
%% citations and references.
%% Uncommenting
%% the next command will enable that style.
%%\citestyle{acmauthoryear}

%%
%% end of the preamble, start of the body of the document source.
\begin{document}

%%
%% The "title" command has an optional parameter,
%% allowing the author to define a "short title" to be used in page headers.
\title{Universal Sets in Coq}

%%
%% The "author" command and its associated commands are used to define
%% the authors and their affiliations.
%% Of note is the shared affiliation of the first two authors, and the
%% "authornote" and "authornotemark" commands
%% used to denote shared contribution to the research.
\author{Andrea Condoluci}
\email{andreacondoluci@gmail.com}
\orcid{1234-5678-9012}
\affiliation{%
  \institution{Institute for Clarity in Documentation}
  \streetaddress{P.O. Box 1212}
  \city{Dublin}
  \state{Ohio}
  \postcode{43017-6221}
}

%%
%% By default, the full list of authors will be used in the page
%% headers. Often, this list is too long, and will overlap
%% other information printed in the page headers. This command allows
%% the author to define a more concise list
%% of authors' names for this purpose.
% \renewcommand{\shortauthors}{Trovato and Tobin, et al.}

%%
%% The abstract is a short summary of the work to be presented in the
%% article.
\begin{abstract}
  New Foundations (\NF) is an exotic alternative to mainstream set theories like Zermelo–Fraenkel (\ZF). The distinguishing feature of \NF{} is permitting the construction of a \emph{universal set}, \ie{} a set that contains all sets (even itself). \TODO{\NF{} is perfectly cabable to caryy out usual mathematics; however, it did not take off because, more than 80 years after its introduction, its consistency is still unclear.}
  
  In this paper, we consider two weak sub-theories of \NF{} whose consistency has already been established: \NFTWO{} and \NFO{}.
  \dots
\end{abstract}

% %%
% %% The code below is generated by the tool at http://dl.acm.org/ccs.cfm.
% %% Please copy and paste the code instead of the example below.
% %%
% \begin{CCSXML}
% <ccs2012>
%  <concept>
%   <concept_id>10010520.10010553.10010562</concept_id>
%   <concept_desc>Computer systems organization~Embedded systems</concept_desc>
%   <concept_significance>500</concept_significance>
%  </concept>
%  <concept>
%   <concept_id>10010520.10010575.10010755</concept_id>
%   <concept_desc>Computer systems organization~Redundancy</concept_desc>
%   <concept_significance>300</concept_significance>
%  </concept>
%  <concept>
%   <concept_id>10010520.10010553.10010554</concept_id>
%   <concept_desc>Computer systems organization~Robotics</concept_desc>
%   <concept_significance>100</concept_significance>
%  </concept>
%  <concept>
%   <concept_id>10003033.10003083.10003095</concept_id>
%   <concept_desc>Networks~Network reliability</concept_desc>
%   <concept_significance>100</concept_significance>
%  </concept>
% </ccs2012>
% \end{CCSXML}

% \ccsdesc[500]{Computer systems organization~Embedded systems}
% \ccsdesc[300]{Computer systems organization~Redundancy}
% \ccsdesc{Computer systems organization~Robotics}
% \ccsdesc[100]{Networks~Network reliability}

%%
%% Keywords. The author(s) should pick words that accurately describe
%% the work being presented. Separate the keywords with commas.
\keywords{set theory, new foundations, universal set}

%%
%% This command processes the author and affiliation and title
%% information and builds the first part of the formatted document.
\maketitle

\section{Introduction}

\TODO{The history of the foundations of mathematics has been \dots}

\paragraph{Set theories.}
A \emph{set theory} is a first-order theory having the usual logical connectives ($\neg$, $\land$, $\lor$, $\to$, $\leftrightarrow$, $\exists$, $\forall$) plus two relation symbols: \emph{equality} ($=$) and \emph{membership} ($\in$). Variables (like $x,y,z,\ldots$) range over sets, and a set is meant to be a collection of sets itself: $x \in y$ means that $x$ is a \emph{member} or an \emph{element} of $y$.

%Formulas of set theory are thus described by the following grammar: 

% \[\begin{array}{rrl}
%   \varphi & \grameq & \bot \mid \neg \varphi \mid \varphi \land \varphi \mid \varphi \lor \varphi \\
%   & & \mid \forall x.\, \varphi \mid \exists x.\, \varphi \\
%   & & \mid x = y \mid x \in y .
% % \TODO{Oh no, we need $\leftrightarrow$!!!}
% \end{array}\]

Sets do not posses any internal structure other than their \emph{extension}, \ie{} the elements they contain. The \emph{axiom of extensionality} formalizes this fact by requiring that sets having the same extension are equal:

\[\forall x y.\, x = y \leftrightarrow \forall z. \, z \in x \leftrightarrow z \in y \quad \tag{\SetExt} \]

A natural way to express a collection of sets is by taking all sets that satisfy a given predicate; this collection is expressed in symbols by $\Compr x \varphi$, reading ``the collection of all $x$'s such that $\varphi$ holds''. \TODO{for instance, ordinals, \dots} A very desirable ??? would be that collections specified in this way are sets themselves, as the \emph{axiom of comprehension} requires:

\[ \exists y.\, \forall x.\, x \in y \leftrightarrow \varphi  \quad \tag{\SetCompr} \]

\TODO{another point about extension is the identification between sets (citizens) and unary predicates over sets:}

Where $y$ does not occur in $\varphi$. \TODO{denote with $\Compr x \varphi$ the set which witnesses the existential $\exists y\ldots$, which is then unique by extensionality \dots }
Set comprehension is actually an axiom schema: one adds an axiom for every $\varphi$. \TODO{but which $\varphi$s are allowed? NOT ALL!}
When all $\varphi$s, unrestricted comprehension. \TODO{But Russel's paradox $\varphi \defeq x \not\in x$.\footnotemark{}}

\footnotetext{By $x\not\in x$ we mean $\neg {(x \in x)}$. }

Problem is circularity.

There are different ways to restrict \SetCompr{} so to avoid Russel's paradox. \ZF's way is to only allow comprehension of \emph{guarded} proposition, of the form $x \in y \land \varphi$ for some $y\neq x$ and an arbitrary $\varphi$. \TODO{$\Compr {x \in y} \varphi$} No self-reference because carve only subsets of already existing sets, and other axioms of ZF force sets to be well-founded (\dots).

NF's way is \dots to restrict the axiom of comprehension to only \emph{stratified} formulas \ldots
The axiom of stratified comprehension does not force the collection $\Compr x {x \not\in x}$ to be a set, because the formula $x \not\in x$ is not stratified. As a consequence, Russel's paradox is avoided.

\subsection{Quantifier-free Comprehension}

In this paper we consider a weaker fragment of full-fledged \NF{}, obtained by restricting the axiom of stratified comprehension to only quantifier-free formulas.

\TODO{dire che: in this sub-case, we can actually replace stratified comprehension with an equivalent axiomatization with a few set operators ($\Universe, \Complement \Placeholder, \Union \Placeholder \Placeholder, \Sing\Placeholder, \Essence\Placeholder$) and charachterizing axioms.}

\TODO{Dire che si puo' fare per Skolemization/ some Skolem theorem?}

By a rapid inspection, we note that when $\varphi$ is quantifier-free, $\varphi$ is stratified if and only if it does not contain ``$x \in x$'' as a subformula.
\TODO{This means that we can drop the definition of stratified, and use a much simplier syntactic restriction on $\varphi$'s}

To understand what operations on sets are necessary to , we 
We show that $\Compr x \varphi$ for $\varphi$ stratified and quantifier-free can . We blah by cases on the structure of $\varphi$:

\[\begin{array}{lcll}
  \Compr x {x = x} & = & \Universe \\
  \Compr x {x = y} & = & \Sing y \\
  \Compr x {y = x} & = & \Sing y \\
  \Compr x {y = z} & = &
    \begin{cases}
      \Universe & \text{if } y = z\\
      \Complement \Universe & \text{otherwise}
    \end{cases}\\
  \Compr x {x \in y} & = & y \\
  \Compr x {y \in x} & = & \Essence y \\
  \Compr x {y \in z} & = &
    \begin{cases}
      \Universe & \text{if } y \in z\\
      \Complement \Universe & \text{otherwise}
    \end{cases}\\
  \Compr x {\neg\varphi} & = & \Complement {\Compr x \varphi} \\
  \Compr x {\varphi \lor \varphi'} & = & \Union {\Compr x \varphi} {\Compr x {\varphi'}} \\
\end{array}\]

where $x\neq y$ and $x \neq z$ above.

where
\begin{enumerate} \renewcommand\labelitemi{--}
  \item $\Universe$ is the universal set, \ie{} the set containing all sets:
    \[ \forall z.\, z \in \Universe \sctag{Universe} \]
  \item $\Complement \Placeholder$ is the set complement:
    \[ \forall z.\, z \in \Complement x \leftrightarrow \neg (z \in x) \sctag{Complement}\]
  \item $\Union \Placeholder \Placeholder$ is the set union:
    \[ \forall z.\, z \in \Union x y \leftrightarrow z \in x \lor z \in y \sctag{Union} \]
  \item $\Sing\Placeholder$ is singleton:
    \[ \forall z.\, z \in \Sing x \leftrightarrow x = z \sctag{Singleton} \]
  \item $\Essence\Placeholder$ is the essence:
    \[ \forall z.\, z \in \Essence{x} \leftrightarrow x \in z \sctag{Essence} \]
\end{enumerate}

\subsection{Sets in Coq}
\TODO{Descrivere Coq e la formalizzazione di ZF in Coq.}

\TODO{shallow embedding}

\section{Roadmap}
Dire:
\begin{itemize}
  \item Come si fara' in ogni sezione: prima definire il dominio, poi l'uguaglianza, poi membership, poi morphism, poi extensionality e set operators independently.
  \item Contenuto delle sezioni sotto
\end{itemize}

\section{\NFTWO}
%!TEX root = main.tex

\renewcommand\INX{\coqref{NF2.Model.IN}{{\IN}}~}
\renewcommand\EQX{\coqref{NF2.Model.EQ}{{\EQ}}~}

\NFTWO{} is the sub-theory of \NF{} characterized by the axiom of extensionality plus universe ($\Universe$), complement ($\Complement \Placeholder$), union ($\Union \Placeholder \Placeholder$), and singleton ($\Sing\Placeholder$) and their related axioms.

% $\Universe, \Complement \Placeholder, \Union \Placeholder \Placeholder, \Sing\Placeholder, \Essence\Placeholder$

\medskip

A first attempt at constructing a model for \NFTWO{} could be to take the \emph{free algebra} generated by these set constructors, basically identifying sets with free expressions of sets. Unfortunately, this approach is bound to fail:

\begin{itemize}
  \item 
  When defining the membership relation \coqdocdefinition{IN}, the singleton constructor $\Sing\Placeholder$ forces \coqdocdefinition{IN} to depend on \coqdocdefinition{EQ}, because \var s $\in \Sing{\text{\var t}}$ if and only if \var t $\equiv$ \var s.
  \item To define \coqdocdefinition{EQ} there is no other option than taking as definition exactly extensionality, \ie{} \\
  % 
  \centerline{
  \coqdocdefinition{EQ} \var s \var {s'} $\Leftrightarrow$ \coqdockw{\ensuremath{\forall}} \coqdocvar{t}, \coqdocvariable{t} \coqdocnotation{∈} \coqdocvariable{s} \coqexternalref{::type scope:x '<->' x}{http://coq.inria.fr/distrib/V8.11.0/stdlib//Coq.Init.Logic}{\coqdocnotation{\ensuremath{\leftrightarrow}}} \coqdocvariable{t} \coqdocnotation{∈} \coqdocvariable{s'}.}
  
  This makes \coqdocdefinition{EQ} depend on \coqdocdefinition{IN}, which makes both definitions invalid because no argument is decreasing.
\end{itemize}

The most promising way to construct a model is actually to first simplify set expressions, finding a ``normal form'' which makes extensional equality as close as possible to intensional equality.

Let us point out that the real difference between the set operators in \NFTWO{} and \ZF{} is the complement ($\Complement \Placeholder$), which is not definable in \ZF{}. A good bet may be that in \NFTWO{} there exist two kinds of sets, which we call \emph{low} and \emph{high} (following \cite{Forster2001}). A low set is similar to a \ZF{} set in that it is a positive collection of sets; a high set, instead, is the complement of a low set, and its extension includes all sets but the ones in a given collection --- in terms of \ZF{}, it actually consists of a proper class. This bet is correct: as we will see in the next sections, low and high sets are stable under boolean combinations. Moreover, they satisfy extensionality.

\subsection{The Type \texorpdfstring{\coqdocinductive{SET}}{SET}}

To define the type of \NFTWO{} sets, we extend Aczel's construction to accommodate high sets, thus introducing an additonal constructor:

\begin{coqdoccode}
  \coqdocnoindent
  \coqdockw{Inductive} \coqdefRef{NF2.Model.SET}{SET}{\coqdocinductive{SET}} :=\coqdoceol
  \coqdocindent{1.00em}
  \ensuremath{|} \coqdef{NF2.Model.Low}{Low}{\coqdocconstructor{Low}} : \coqdocnotation{\ensuremath{\forall}} \coqdocvar{X}, \coqexternalref{::type scope:x '->' x}{http://coq.inria.fr/distrib/V8.11.0/stdlib//Coq.Init.Logic}{\coqdocnotation{(}}\coqdocvariable{X} \coqexternalref{::type scope:x '->' x}{http://coq.inria.fr/distrib/V8.11.0/stdlib//Coq.Init.Logic}{\coqdocnotation{\ensuremath{\rightarrow}}} \coqref{NF2.Model.SET}{\coqdocinductive{SET}}\coqexternalref{::type scope:x '->' x}{http://coq.inria.fr/distrib/V8.11.0/stdlib//Coq.Init.Logic}{\coqdocnotation{)}} \coqexternalref{::type scope:x '->' x}{http://coq.inria.fr/distrib/V8.11.0/stdlib//Coq.Init.Logic}{\coqdocnotation{\ensuremath{\rightarrow}}} \coqref{NF2.Model.SET}{\coqdocinductive{SET}}\coqdoceol
  \coqdocindent{1.00em}
  \ensuremath{|} \coqdef{NF2.Model.High}{High}{\coqdocconstructor{High}} : \coqdocnotation{\ensuremath{\forall}} \coqdocvar{X}, \coqexternalref{::type scope:x '->' x}{http://coq.inria.fr/distrib/V8.11.0/stdlib//Coq.Init.Logic}{\coqdocnotation{(}}\coqdocvariable{X} \coqexternalref{::type scope:x '->' x}{http://coq.inria.fr/distrib/V8.11.0/stdlib//Coq.Init.Logic}{\coqdocnotation{\ensuremath{\rightarrow}}} \coqref{NF2.Model.SET}{\coqdocinductive{SET}}\coqexternalref{::type scope:x '->' x}{http://coq.inria.fr/distrib/V8.11.0/stdlib//Coq.Init.Logic}{\coqdocnotation{)}} \coqexternalref{::type scope:x '->' x}{http://coq.inria.fr/distrib/V8.11.0/stdlib//Coq.Init.Logic}{\coqdocnotation{\ensuremath{\rightarrow}}} \coqref{NF2.Model.SET}{\coqdocinductive{SET}}
  .\coqdoceol
  \coqdocemptyline
\end{coqdoccode}

Note that the indexing function \var f in the set \var s = \coqdef{NF2.Model.High}{High}{\coqdocconstructor{High}} \var X \var f does not actually index the sets in the extension of \var s, but the sets that are \emph{not} in the extension of \var s (because a high set is supposed to represent a complement).

\begin{example}[Empty set and Universe]\label{ex:EUNF2}
The empty set is defined as a low set with empty index:

  \begin{coqdoccode}
    \coqdocnoindent
    \coqdockw{Definition} \coqdocdefinition{Ø} := \coqdocconstructor{Low} \coqexternalref{False}{http://coq.inria.fr/distrib/V8.11.0/stdlib//Coq.Init.Logic}{\coqdocinductive{False}} (\coqdockw{fun} \coqdocvar{x} \ensuremath{\Rightarrow} \coqdockw{match} \coqdocvariable{x} \coqdockw{with} \coqdockw{end}).\coqdoceol
    \coqdocemptyline
  \end{coqdoccode}

  In a fully specular way, the universal set is defined as a \emph{high} set with empty index:

  \begin{coqdoccode}
    \coqdocnoindent
    \coqdockw{Definition} \coqdocdefinition{U} := \coqdocconstructor{High} \coqexternalref{False}{http://coq.inria.fr/distrib/V8.11.0/stdlib//Coq.Init.Logic}{\coqdocinductive{False}} (\coqdockw{fun} \coqdocvar{x} \ensuremath{\Rightarrow} \coqdockw{match} \coqdocvariable{x} \coqdockw{with} \coqdockw{end}).\coqdoceol
    \coqdocemptyline
  \end{coqdoccode}
\end{example}

\subsection{Set Equality}

To define equality of \NFO{} sets, we enforce our intuition that a low set cannot be equivalent to a high set (because high sets are much larger):

\begin{coqdoccode}
  \coqdocnoindent
\coqdockw{Fixpoint} \coqdefRef{NF2.Model.EQ}{EQ}{\coqdocdefinition{EQ}} \coqdocvar{s} \coqdocvar{t} := \coqdockw{match} \coqdocvariable{s},\coqdocvariable{t} \coqdockw{with}\coqdoceol
\coqdocindent{1.00em}
\ensuremath{|} \coqref{NF2.Model.Low}{\coqdocconstructor{Low}} \coqdocvar{\_} \coqdocvar{\_}, \coqref{NF2.Model.High}{\coqdocconstructor{High}} \coqdocvar{\_} \coqdocvar{\_} \ensuremath{\Rightarrow} \coqexternalref{False}{http://coq.inria.fr/distrib/V8.11.0/stdlib//Coq.Init.Logic}{\coqdocinductive{False}}\coqdoceol
\coqdocindent{1.00em}
\ensuremath{|} \coqref{NF2.Model.High}{\coqdocconstructor{High}} \coqdocvar{\_} \coqdocvar{\_}, \coqref{NF2.Model.Low}{\coqdocconstructor{Low}} \coqdocvar{\_} \coqdocvar{\_} \ensuremath{\Rightarrow} \coqexternalref{False}{http://coq.inria.fr/distrib/V8.11.0/stdlib//Coq.Init.Logic}{\coqdocinductive{False}}\coqdoceol
\coqdocindent{1.00em}
\ensuremath{|} \coqref{NF2.Model.Low}{\coqdocconstructor{Low}} \coqdocvar{\_} \coqdocvar{f}, \coqref{NF2.Model.Low}{\coqdocconstructor{Low}} \coqdocvar{\_} \coqdocvar{g} \ensuremath{\Rightarrow} \coqdocvar{f} $\AEQ$ \coqdocvar{g} \coqdoceol
\coqdocindent{1.00em}
\ensuremath{|} \coqref{NF2.Model.High}{\coqdocconstructor{High}} \coqdocvar{\_} \coqdocvar{f}, \coqref{NF2.Model.High}{\coqdocconstructor{High}} \coqdocvar{\_} \coqdocvar{g} \ensuremath{\Rightarrow} \coqdocvar{f} $\AEQ$ \coqdocvar{g} \coqdoceol
\coqdocnoindent
\coqdockw{end}.\coqdoceol

\end{coqdoccode}

We are a bit cheating here: in fact, the notation for \AEQ{} hides a dependency on \coqdocdefinition{EQ}, which would actually make the \coqdockw{Fixpoint} definition fail as it conceals what argument is decreasing. For the actual \Coq{} definition, we inlined \coqdocdefinition{eq\_aczel} so to make \Coq{} recognise the decreasing argument. However, we decided to just use the succint notation above, so to help the intuition.

\medskip

Proving that \coqdocdefinition{EQ} is an equivalence relation goes smoothly and just like in \ZF{}; the only difference here is the cases to consider are doubled because of the two constructors:

\begin{coqdoccode}
  \coqdocnoindent
  \coqdockw{Instance} \coqdefRef{NF2.Model.nf2 setoid}{nf2\_setoid}{\coqdocinstance{nf2\_setoid}} : \coqexternalref{Equivalence}{http://coq.inria.fr/distrib/V8.11.0/stdlib//Coq.Classes.RelationClasses}{\coqdocclass{Equivalence}} \coqref{NF2.Model.EQ}{\coqdocdefinition{EQ}}.
\end{coqdoccode}

\subsection{Set Membership}

To define membership of \NFTWO{} sets, 

\begin{coqdoccode}
  \coqdocnoindent
\coqdockw{Definition} \coqdefRef{NF2.Model.IN}{IN}{\coqdocdefinition{IN}} \coqdocvar{s} \coqdocvar{t} := \coqdockw{match} \coqdocvariable{t} \coqdockw{with}\coqdoceol
\coqdocindent{1.00em}
\ensuremath{|} \coqref{NF2.Model.Low}{\coqdocconstructor{Low}} \coqdocvar{\_} \coqdocvar{f} \ensuremath{\Rightarrow} \coqdocvariable{s} \coqref{Internal.Common.in aczel}{\AIN} \coqdocvariable{f} \coqdoceol
\coqdocindent{1.00em}
\ensuremath{|} \coqref{NF2.Model.High}{\coqdocconstructor{High}} \coqdocvar{\_} \coqdocvar{f} \ensuremath{\Rightarrow} \coqdocvariable{s} \coqref{Internal.Common.in aczel}{$\not\AIN$} \coqdocvariable{f} \coqdoceol
\coqdocnoindent
\coqdockw{end}.\coqdoceol
\end{coqdoccode}

\medskip

Proving that \coqdocdefinition{IN} is a morphism goes smoothly and just like in \ZF{}:

\begin{coqdoccode}
  \coqdocnoindent
\coqdockw{Add} \coqdockw{Morphism} \coqref{NF2.Model.IN}{\coqdocdefinition{IN}} \coqdockw{with} \coqdockw{signature} \coqref{NF2.Model.EQ}{\coqdocdefinition{EQ}} \coqexternalref{ProperNotations.::signature scope:x '==>' x}{http://coq.inria.fr/distrib/V8.11.0/stdlib//Coq.Classes.Morphisms}{\coqdocnotation{$\Rightarrow$}} \coqref{NF2.Model.EQ}{\coqdocdefinition{EQ}} \coqexternalref{ProperNotations.::signature scope:x '==>' x}{http://coq.inria.fr/distrib/V8.11.0/stdlib//Coq.Classes.Morphisms}{\coqdocnotation{$\Rightarrow$}} \coqexternalref{iff}{http://coq.inria.fr/distrib/V8.11.0/stdlib//Coq.Init.Logic}{\coqdocdefinition{iff}} \coqdockw{as} \coqdocdefinition{IN\_mor}.\coqdoceol
\coqdocnoindent
\end{coqdoccode}

\subsection{Set Operators}

Complementing a \NFTWO{} set is straighforward: the complement of a low set is a high set, and viceversa:

\begin{coqdoccode}
  \coqdockw{Definition} \coqdef{NF2.Sets.compl}{compl}{\coqdocdefinition{compl}} : \coqref{NF2.Model.SET}{\coqdocinductive{SET}} \coqexternalref{::type scope:x '->' x}{http://coq.inria.fr/distrib/V8.11.0/stdlib//Coq.Init.Logic}{\coqdocnotation{\ensuremath{\rightarrow}}} \coqref{NF2.Model.SET}{\coqdocinductive{SET}} :=\coqdoceol
\coqdocindent{1.00em}
\coqdockw{fun} \coqdocvar{s} \ensuremath{\Rightarrow} \coqdockw{match} \coqdocvariable{s} \coqdockw{with}\coqdoceol
\coqdocindent{1.00em}
\ensuremath{|} \coqref{NF2.Model.Low}{\coqdocconstructor{Low}} \coqdocvar{\_} \coqdocvar{f} \ensuremath{\Rightarrow} \coqref{NF2.Model.High}{\coqdocconstructor{High}} \coqdocvar{\_} \coqdocvar{f}\coqdoceol
\coqdocindent{1.00em}
\ensuremath{|} \coqref{NF2.Model.High}{\coqdocconstructor{High}} \coqdocvar{\_} \coqdocvar{f} \ensuremath{\Rightarrow} \coqref{NF2.Model.Low}{\coqdocconstructor{Low}} \coqdocvar{\_} \coqdocvar{f}\coqdoceol
\coqdocindent{1.00em}
\coqdockw{end}.\coqdoceol
\coqdocemptyline
\coqdocnoindent
\coqdockw{Lemma} \coqdef{NF2.Sets.compl ok}{compl\_ok}{\coqdoclemma{compl\_ok}} : \coqdockw{\ensuremath{\forall}} \coqdocvar{s} \coqdocvar{t}, \coqdocvariable{s} \coqref{NF2.Model.:::x 'xE2x88x88' x}{\coqdocnotation{∈}} \coqref{NF2.Sets.compl}{\coqdocdefinition{compl}} \coqdocvariable{t} \coqexternalref{::type scope:x '<->' x}{http://coq.inria.fr/distrib/V8.11.0/stdlib//Coq.Init.Logic}{\coqdocnotation{\ensuremath{\leftrightarrow}}} \coqexternalref{::type scope:x '<->' x}{http://coq.inria.fr/distrib/V8.11.0/stdlib//Coq.Init.Logic}{\coqdocnotation{(}}\coqdocvariable{s} \coqref{NF2.Model.:::x 'xE2x88x88' x}{\coqdocnotation{∈}} \coqdocvariable{t} \coqexternalref{::type scope:x '->' x}{http://coq.inria.fr/distrib/V8.11.0/stdlib//Coq.Init.Logic}{\coqdocnotation{\ensuremath{\rightarrow}}} \coqexternalref{False}{http://coq.inria.fr/distrib/V8.11.0/stdlib//Coq.Init.Logic}{\coqdocinductive{False}}\coqexternalref{::type scope:x '<->' x}{http://coq.inria.fr/distrib/V8.11.0/stdlib//Coq.Init.Logic}{\coqdocnotation{)}}.\coqdoceol
\end{coqdoccode}
\begin{proof}
  Just destruct \var t and apply few simplifications. Note: classical reasoning is required to turn a negated universal quantifier into an existential quantifier.
\end{proof}

A singleton is just a set whose indexing function is constant:

\begin{coqdoccode}
  \coqdocnoindent
\coqdockw{Definition} \coqdef{NF2.Sets.sing}{sing}{\coqdocdefinition{sing}} : \coqref{NF2.Model.SET}{\coqdocinductive{SET}} \coqexternalref{::type scope:x '->' x}{http://coq.inria.fr/distrib/V8.11.0/stdlib//Coq.Init.Logic}{\coqdocnotation{\ensuremath{\rightarrow}}} \coqref{NF2.Model.SET}{\coqdocinductive{SET}} :=\coqdoceol
\coqdocindent{1.00em}
\coqdockw{fun} \coqdocvar{s} \ensuremath{\Rightarrow} \coqref{NF2.Model.Low}{\coqdocconstructor{Low}} \coqexternalref{unit}{http://coq.inria.fr/distrib/V8.11.0/stdlib//Coq.Init.Datatypes}{\coqdocinductive{unit}} (\coqdockw{fun} \coqdocvar{\_} \ensuremath{\Rightarrow} \coqdocvariable{s}).\coqdoceol
\coqdocemptyline
\coqdocnoindent
\coqdockw{Definition} \coqdef{NF2.Sets.sing ok}{sing\_ok}{\coqdocdefinition{sing\_ok}} : \coqdockw{\ensuremath{\forall}} \coqdocvar{s} \coqdocvar{t}, \coqdocvariable{s} \coqref{NF2.Model.:::x 'xE2x88x88' x}{\coqdocnotation{∈}} \coqref{NF2.Sets.sing}{\coqdocdefinition{sing}} \coqdocvariable{t} \coqexternalref{::type scope:x '<->' x}{http://coq.inria.fr/distrib/V8.11.0/stdlib//Coq.Init.Logic}{\coqdocnotation{\ensuremath{\leftrightarrow}}} \coqdocvariable{t} \coqref{NF2.Model.:::x 'xE2x89xA1' x}{\coqdocnotation{≡}} \coqdocvariable{s}.\coqdoceol
\coqdocemptyline
\end{coqdoccode}

The definition of set union is slightly more involved, but straightforward. It relies on additional functions \coqdocdefinition{minus}, \coqdocdefinition{join} and \coqdocdefinition{meet} to perform respectively the set difference, intersection, and union of indexing functions. The various cases in the definition follow directly from the laws of boolean logic:

\begin{coqdoccode}
  \coqdocnoindent
  \coqdockw{Definition} \coqdef{NF2.Sets.cup}{cup}{\coqdocdefinition{cup}} \coqdocvar{s} \coqdocvar{s'} := \coqdockw{match} \coqdocvariable{s}, \coqdocvariable{s'} \coqdockw{with}\coqdoceol
\coqdocindent{1.00em}
\ensuremath{|} \coqref{NF2.Model.Low}{\coqdocconstructor{Low}} \coqdocvar{\_} \coqdocvar{f}, \coqref{NF2.Model.High}{\coqdocconstructor{High}} \coqdocvar{\_} \coqdocvar{g} \ensuremath{\Rightarrow} \coqref{NF2.Model.High}{\coqdocconstructor{High}} \coqdocvar{\_} (\coqref{NF2.Sets.minus}{\coqdocdefinition{minus}} \coqdocvar{g} \coqdocvar{f})\coqdoceol
\coqdocindent{1.00em}
\ensuremath{|} \coqref{NF2.Model.High}{\coqdocconstructor{High}} \coqdocvar{\_} \coqdocvar{f}, \coqref{NF2.Model.Low}{\coqdocconstructor{Low}} \coqdocvar{\_} \coqdocvar{g} \ensuremath{\Rightarrow} \coqref{NF2.Model.High}{\coqdocconstructor{High}} \coqdocvar{\_} (\coqref{NF2.Sets.minus}{\coqdocdefinition{minus}} \coqdocvar{f} \coqdocvar{g})\coqdoceol
\coqdocindent{1.00em}
\ensuremath{|} \coqref{NF2.Model.Low}{\coqdocconstructor{Low}} \coqdocvar{\_} \coqdocvar{f}, \coqref{NF2.Model.Low}{\coqdocconstructor{Low}} \coqdocvar{\_} \coqdocvar{g} \ensuremath{\Rightarrow} \coqref{NF2.Model.Low}{\coqdocconstructor{Low}} \coqdocvar{\_} (\coqref{NF2.Sets.join}{\coqdocdefinition{join}} \coqdocvar{f} \coqdocvar{g})\coqdoceol
\coqdocindent{1.00em}
\ensuremath{|} \coqref{NF2.Model.High}{\coqdocconstructor{High}} \coqdocvar{\_} \coqdocvar{f}, \coqref{NF2.Model.High}{\coqdocconstructor{High}} \coqdocvar{\_} \coqdocvar{g} \ensuremath{\Rightarrow} \coqref{NF2.Model.Low}{\coqdocconstructor{Low}} \coqdocvar{\_} (\coqref{NF2.Sets.meet}{\coqdocdefinition{meet}} \coqdocvar{f} \coqdocvar{g})\coqdoceol
\coqdocnoindent
\coqdockw{end}.\coqdoceol
\coqdocnoindent
\coqdockw{Notation} \coqdef{NF2.Sets.:::x 'xE2x88xAA' x}{"}{"}A ∪ B" := (\coqref{NF2.Sets.cup}{\coqdocdefinition{cup}} \coqdocvar{A} \coqdocvar{B}) (\coqdoctac{at} \coqdockw{level} 85).\coqdoceol
\coqdocemptyline
\end{coqdoccode}

As an example, we provide the definition of \coqdocdefinition{minus} (the others are similar):

\begin{coqdoccode}
  \coqdocnoindent
  \coqdockw{Definition} \coqdef{NF2.Sets.minus}{minus}{\coqdocdefinition{minus}} \{\coqdocvar{X} \coqdocvar{Y}\} \coqdocvar{f} \coqdocvar{g} \coqdoceol
  \coqdocindent{1.00em}
  : \coqexternalref{::type scope:'x7B' x ':' x 'x26' x 'x7D'}{http://coq.inria.fr/distrib/V8.11.0/stdlib//Coq.Init.Specif}{\coqdocnotation{\{}} \coqdocvar{x} \coqexternalref{::type scope:'x7B' x ':' x 'x26' x 'x7D'}{http://coq.inria.fr/distrib/V8.11.0/stdlib//Coq.Init.Specif}{\coqdocnotation{:}} \coqdocvariable{X} \coqexternalref{::type scope:'x7B' x ':' x 'x26' x 'x7D'}{http://coq.inria.fr/distrib/V8.11.0/stdlib//Coq.Init.Specif}{\coqdocnotation{\&}} \coqdockw{\ensuremath{\forall}} \coqdocvar{y} : \coqdocvariable{Y}, \coqexternalref{::type scope:'x7E' x}{http://coq.inria.fr/distrib/V8.11.0/stdlib//Coq.Init.Logic}{\coqdocnotation{\ensuremath{\lnot}}} \coqexternalref{::type scope:'x7E' x}{http://coq.inria.fr/distrib/V8.11.0/stdlib//Coq.Init.Logic}{\coqdocnotation{(}}\coqdocvariable{g} \coqdocvariable{y} \coqref{NF2.Model.:::x 'xE2x89xA1' x}{\coqdocnotation{≡}} \coqdocvariable{f} \coqdocvariable{x}\coqexternalref{::type scope:'x7E' x}{http://coq.inria.fr/distrib/V8.11.0/stdlib//Coq.Init.Logic}{\coqdocnotation{)}} \coqexternalref{::type scope:'x7B' x ':' x 'x26' x 'x7D'}{http://coq.inria.fr/distrib/V8.11.0/stdlib//Coq.Init.Specif}{\coqdocnotation{\}}} \coqexternalref{::type scope:x '->' x}{http://coq.inria.fr/distrib/V8.11.0/stdlib//Coq.Init.Logic}{\coqdocnotation{\ensuremath{\rightarrow}}} \coqref{NF2.Model.SET}{\coqdocinductive{SET}} \coqdoceol
\coqdocindent{1.00em}:=
\coqref{Internal.Misc.select}{\coqdocdefinition{select}} \coqdocvariable{f} (\coqdockw{fun} \coqdocvar{x} \ensuremath{\Rightarrow} \coqdockw{\ensuremath{\forall}} \coqdocvar{y}, \coqexternalref{::type scope:'x7E' x}{http://coq.inria.fr/distrib/V8.11.0/stdlib//Coq.Init.Logic}{\coqdocnotation{\ensuremath{\lnot}}} \coqexternalref{::type scope:'x7E' x}{http://coq.inria.fr/distrib/V8.11.0/stdlib//Coq.Init.Logic}{\coqdocnotation{(}}\coqdocvariable{g} \coqdocvariable{y} \coqref{NF2.Model.:::x 'xE2x89xA1' x}{\coqdocnotation{≡}} \coqdocvariable{f} \coqdocvariable{x}\coqexternalref{::type scope:'x7E' x}{http://coq.inria.fr/distrib/V8.11.0/stdlib//Coq.Init.Logic}{\coqdocnotation{)}}).\coqdoceol
\coqdocemptyline
\end{coqdoccode}

It requires an auxiliary function \coqdocdefinition{select} \var f \var P that restricts the domain of a function \var f according to a predicate \var P:

\begin{coqdoccode}
  \coqdocnoindent
  \coqdockw{Definition} \coqdef{Internal.Misc.select}{select}{\coqdocdefinition{select}} \{\coqdocvar{X} \coqdocvar{Y}\} (\coqdocvar{f}: \coqdocvariable{X} \coqexternalref{::type scope:x '->' x}{http://coq.inria.fr/distrib/V8.11.0/stdlib//Coq.Init.Logic}{\coqdocnotation{\ensuremath{\rightarrow}}} \coqdocvariable{Y}) (\coqdocvar{P}: \coqdocvariable{X} \coqexternalref{::type scope:x '->' x}{http://coq.inria.fr/distrib/V8.11.0/stdlib//Coq.Init.Logic}{\coqdocnotation{\ensuremath{\rightarrow}}} \coqdockw{Prop}) \coqdoceol
  \coqdocindent{1.00em} : \coqexternalref{::type scope:'x7B' x ':' x 'x26' x 'x7D'}{http://coq.inria.fr/distrib/V8.11.0/stdlib//Coq.Init.Specif}{\coqdocnotation{\{}}\coqdocvar{x}\coqexternalref{::type scope:'x7B' x ':' x 'x26' x 'x7D'}{http://coq.inria.fr/distrib/V8.11.0/stdlib//Coq.Init.Specif}{\coqdocnotation{:}} \coqdocvariable{X} \coqexternalref{::type scope:'x7B' x ':' x 'x26' x 'x7D'}{http://coq.inria.fr/distrib/V8.11.0/stdlib//Coq.Init.Specif}{\coqdocnotation{\&}} \coqdocvariable{P} \coqdocvariable{x}\coqexternalref{::type scope:'x7B' x ':' x 'x26' x 'x7D'}{http://coq.inria.fr/distrib/V8.11.0/stdlib//Coq.Init.Specif}{\coqdocnotation{\}}} \coqexternalref{::type scope:x '->' x}{http://coq.inria.fr/distrib/V8.11.0/stdlib//Coq.Init.Logic}{\coqdocnotation{\ensuremath{\rightarrow}}} \coqdocvariable{Y}\coqdoceol
  \coqdocindent{1.00em}
  := \coqdockw{fun} \coqdocvar{x} \ensuremath{\Rightarrow} \coqdocvariable{f} (\coqexternalref{projT1}{http://coq.inria.fr/distrib/V8.11.0/stdlib//Coq.Init.Specif}{\coqdocdefinition{projT1}} \coqdocvariable{x}).\coqdoceol
  \coqdocemptyline
\end{coqdoccode}

\begin{coqdoccode}
\coqdocnoindent
\coqdockw{Lemma} \coqdef{NF2.Sets.cup ok}{cup\_ok}{\coqdoclemma{cup\_ok}} : \coqdockw{\ensuremath{\forall}} \coqdocvar{s} \coqdocvar{s'} \coqdocvar{t}, \coqdocvariable{t} \coqref{NF2.Model.:::x 'xE2x88x88' x}{\coqdocnotation{∈}} \coqref{NF2.Model.:::x 'xE2x88x88' x}{\coqdocnotation{(}}\coqdocvariable{s} \coqref{NF2.Sets.:::x 'xE2x88xAA' x}{\coqdocnotation{∪}} \coqdocvariable{s'}\coqref{NF2.Model.:::x 'xE2x88x88' x}{\coqdocnotation{)}} \coqexternalref{::type scope:x '<->' x}{http://coq.inria.fr/distrib/V8.11.0/stdlib//Coq.Init.Logic}{\coqdocnotation{\ensuremath{\leftrightarrow}}} \coqexternalref{::type scope:x 'x5C/' x}{http://coq.inria.fr/distrib/V8.11.0/stdlib//Coq.Init.Logic}{\coqdocnotation{(}}\coqdocvariable{t} \coqref{NF2.Model.:::x 'xE2x88x88' x}{\coqdocnotation{∈}} \coqdocvariable{s}\coqexternalref{::type scope:x 'x5C/' x}{http://coq.inria.fr/distrib/V8.11.0/stdlib//Coq.Init.Logic}{\coqdocnotation{)}} \coqexternalref{::type scope:x 'x5C/' x}{http://coq.inria.fr/distrib/V8.11.0/stdlib//Coq.Init.Logic}{\coqdocnotation{\ensuremath{\lor}}} \coqexternalref{::type scope:x 'x5C/' x}{http://coq.inria.fr/distrib/V8.11.0/stdlib//Coq.Init.Logic}{\coqdocnotation{(}}\coqdocvariable{t} \coqref{NF2.Model.:::x 'xE2x88x88' x}{\coqdocnotation{∈}} \coqdocvariable{s'}\coqexternalref{::type scope:x 'x5C/' x}{http://coq.inria.fr/distrib/V8.11.0/stdlib//Coq.Init.Logic}{\coqdocnotation{)}}.\coqdoceol
\end{coqdoccode}

\subsection{Extensionality}
Proving extensionality in \NFTWO{} is slightly more involved than in \ZF{}. Like in \ZF{}, one direction of the double implication is easy and follows directly from \coqdocdefinition{in\_sound\_right}.

As for the other direction, \ie{} \\
\centerline{
  (\coqdockw{\ensuremath{\forall}} \coqdocvar{t}, \coqdocvariable{t} \INX \coqdocvariable{s} \coqexternalref{::type scope:x '<->' x}{http://coq.inria.fr/distrib/V8.11.0/stdlib//Coq.Init.Logic}{\coqdocnotation{\ensuremath{\leftrightarrow}}} \coqdocvariable{t} \INX \coqdocvariable{s'}) %
  \coqexternalref{::type scope:x '<->' x}{http://coq.inria.fr/distrib/V8.11.0/stdlib//Coq.Init.Logic}{\coqdocnotation{\ensuremath{\to}}} %
  \coqdocvariable{s} \EQX \coqdocvariable{s'}
}

\noindent
we proceed by cases over \var s and \var{s'}. When their constructor matches (cases \coqdocconstructor{Low}/\coqdocconstructor{Low} and \coqdocconstructor{High}/\coqdocconstructor{High}) the proof proceeds just like in \ZF{}. The new cases \coqdocconstructor{Low} \emph{vs} \coqdocconstructor{High} require some more work. In these cases, \coqdocvariable{s} \EQX \coqdocvariable{s'} does not hold because low sets and high sets are not equivalent by definition of \coqdocdefinition{EQ}. Therefore to conclude it suffices to assume \coqdockw{\ensuremath{\forall}} \coqdocvar{t}, \coqdocvariable{t} \INX \coqdocvariable{s} \coqexternalref{::type scope:x '<->' x}{http://coq.inria.fr/distrib/V8.11.0/stdlib//Coq.Init.Logic}{\coqdocnotation{\ensuremath{\leftrightarrow}}} \coqdocvariable{t} \INX \coqdocvariable{s'} and derive a contradiction.

Without loss of generality, let us say that \var s = \coqref{NF2.Model.Low}{\coqdocconstructor{Low}} \coqdocvar{X} \coqdocvar{f} and \var {s'} = \coqref{NF2.Model.High}{\coqdocconstructor{High}} \coqdocvar{Y} \coqdocvar{g} (the other case is specular). Basically, what we have to prove is that a low set and a high set cannot have the same extension (upcoming lemma \coqref{NF2.Ext.pos neg ext neq}{\coqdoclemma{pos\_neg\_ext\_neq}}), which is justified by our initial intuition that low sets are ``small'', while high sets correspond to proper classes.

\medskip

The crucial intuition is that when a low set has the same extension as a high set, it is actually possible to construct a universal set that is low (contrarily to the universal set defined in \Cref{ex:EUNF2}, which is correctly high). In fact, the images of \var f and \var g are complementary, and it suffices to take their disjoin union:

\begin{coqdoccode}
  \coqdocnoindent
  \coqdockw{Lemma} \coqdefRef{NF2.Ext.pos univ}{pos\_univ}{\coqdoclemma{pos\_univ}}: \coqdockw{\ensuremath{\forall}} \coqdocvar{X} \coqdocvar{f} \coqdocvar{Y} \coqdocvar{g},\coqdoceol
\coqdocindent{1.00em}
\coqexternalref{::type scope:x '->' x}{http://coq.inria.fr/distrib/V8.11.0/stdlib//Coq.Init.Logic}{\coqdocnotation{(}}\coqdockw{\ensuremath{\forall}} \coqdocvar{s}, \coqdocvariable{s} \INX \coqref{NF2.Model.Low}{\coqdocconstructor{Low}} \coqdocvariable{X} \coqdocvariable{f} \coqexternalref{::type scope:x '<->' x}{http://coq.inria.fr/distrib/V8.11.0/stdlib//Coq.Init.Logic}{\coqdocnotation{\ensuremath{\leftrightarrow}}} \coqdocvariable{s} \INX \coqref{NF2.Model.High}{\coqdocconstructor{High}} \coqdocvariable{Y} \coqdocvariable{g}\coqexternalref{::type scope:x '->' x}{http://coq.inria.fr/distrib/V8.11.0/stdlib//Coq.Init.Logic}{\coqdocnotation{)}}\coqdoceol
\coqdocindent{1.00em}
\coqexternalref{::type scope:x '->' x}{http://coq.inria.fr/distrib/V8.11.0/stdlib//Coq.Init.Logic}{\coqdocnotation{\ensuremath{\rightarrow}}} \coqdockw{\ensuremath{\forall}} \coqdocvar{s}, \coqdocvariable{s} \INX \coqref{NF2.Model.Low}{\coqdocconstructor{Low}} (\coqdocvariable{X} \coqexternalref{::type scope:x '+' x}{http://coq.inria.fr/distrib/V8.11.0/stdlib//Coq.Init.Datatypes}{\coqdocnotation{+}} \coqdocvariable{Y}) (\coqdocvariable{f} \coqref{Internal.Misc.:::x 'xE2xA8x81' x}{\coqdocnotation{⨁}} \coqdocvariable{g}).\coqdoceol
\end{coqdoccode}
\begin{proof}
  Note: classical logic is required to prove this lemma.
  Assume $\coqdocvariable{s} \colon \coqdocinductive{SET}$. To show that \coqdocvariable{s} \INX \coqref{NF2.Model.Low}{\coqdocconstructor{Low}} (\coqdocvariable{X} \coqexternalref{::type scope:x '+' x}{http://coq.inria.fr/distrib/V8.11.0/stdlib//Coq.Init.Datatypes}{\coqdocnotation{+}} \coqdocvariable{Y}) (\coqdocvariable{f} \coqref{Internal.Misc.:::x 'xE2xA8x81' x}{\coqdocnotation{⨁}} \coqdocvariable{g}), we need to supply an element of type \coqdocvariable{X} \coqexternalref{::type scope:x '+' x}{http://coq.inria.fr/distrib/V8.11.0/stdlib//Coq.Init.Datatypes}{\coqdocnotation{+}} \coqdocvariable{Y}. Choosing the left or right injection into \coqdocvariable{X} \coqexternalref{::type scope:x '+' x}{http://coq.inria.fr/distrib/V8.11.0/stdlib//Coq.Init.Datatypes}{\coqdocnotation{+}} \coqdocvariable{Y} is equivalent to deciding whether \coqdocvariable{s} \INX \coqref{NF2.Model.Low}{\coqdocconstructor{Low}} \coqdocvariable{X} \coqdocvariable{f} holds.
  % \TODO{}
\end{proof}

To conclude, we just need to show that a universal set cannot be low. We resort to a form of ``regularity'' of low sets. Recall that regularity is an axiom of \ZF{} set theory which fundamentally forces sets to be well-founded. Clearly regularity does not hold in \NFTWO{}, since set membership is clearly not well-founded --- for instance, $\Universe \in \Universe \in \ldots$. However, a weaker form of regularity holds for low sets, namely that a low set cannot contain itself:

\begin{coqdoccode}
  \coqdocnoindent
\coqdockw{Theorem} \coqdefRef{NF2.Ext.weak regularity}{weak\_regularity}{\coqdoclemma{weak\_regularity}}: \coqdockw{\ensuremath{\forall}} \coqdocvar{s}, \coqref{NF2.Model.low}{\coqdocdefinition{low}} \coqdocvariable{s} \coqexternalref{::type scope:x '->' x}{http://coq.inria.fr/distrib/V8.11.0/stdlib//Coq.Init.Logic}{\coqdocnotation{\ensuremath{\rightarrow}}} \coqdocvariable{s} \INX \coqdocvariable{s} \coqexternalref{::type scope:x '->' x}{http://coq.inria.fr/distrib/V8.11.0/stdlib//Coq.Init.Logic}{\coqdocnotation{\ensuremath{\rightarrow}}} \coqexternalref{False}{http://coq.inria.fr/distrib/V8.11.0/stdlib//Coq.Init.Logic}{\coqdocinductive{False}}.\coqdoceol
\end{coqdoccode}
\begin{proof}
  By structural induction on \coqdocvariable{s}, low by hypothesis. Let \coqdocvariable{s} = \coqref{NF2.Model.Low}{\coqdocconstructor{Low}} \coqdocvariable{X} \coqdocvariable{f}. From \coqdocvariable{s} \INX \coqdocvariable{s} it follows that there exists \coqdocvariable{x} such that \coqdocvariable{s} \EQX \coqdocvariable{f}~\coqdocvariable{x}. Note that \coqdocvariable{f}~\coqdocvariable{x} must be a low set by the definition of \coqdocdefinition{EQ}. Since \coqdocdefinition{IN} is a \coqdocdefinition{EQ}-morphism, it follows that \coqdocvariable{f}~\coqdocvariable{x} \INX \coqdocvariable{f}~\coqdocvariable{x}. We conclude by the inductive hypothesis.
\end{proof}

It directly follows:

\begin{coqdoccode}
  \coqdocnoindent
\coqdockw{Lemma} \coqdefRef{NF2.Ext.pos neg ext neq}{pos\_neg\_ext\_neq}{\coqdoclemma{pos\_neg\_ext\_neq}}: \coqdockw{\ensuremath{\forall}} \coqdocvar{X} \coqdocvar{f} \coqdocvar{Y} \coqdocvar{g},\coqdoceol
\coqdocindent{1.00em}
\coqexternalref{::type scope:x '->' x}{http://coq.inria.fr/distrib/V8.11.0/stdlib//Coq.Init.Logic}{\coqdocnotation{(}}\coqdockw{\ensuremath{\forall}} \coqdocvar{s}, \coqdocvariable{s} \INX \coqref{NF2.Model.Low}{\coqdocconstructor{Low}} \coqdocvariable{X} \coqdocvariable{f} \coqexternalref{::type scope:x '<->' x}{http://coq.inria.fr/distrib/V8.11.0/stdlib//Coq.Init.Logic}{\coqdocnotation{\ensuremath{\leftrightarrow}}} \coqdocvariable{s} \INX \coqref{NF2.Model.High}{\coqdocconstructor{High}} \coqdocvariable{Y} \coqdocvariable{g} \coqexternalref{::type scope:x '->' x}{http://coq.inria.fr/distrib/V8.11.0/stdlib//Coq.Init.Logic}{\coqdocnotation{)}} \coqexternalref{::type scope:x '->' x}{http://coq.inria.fr/distrib/V8.11.0/stdlib//Coq.Init.Logic}{\coqdocnotation{\ensuremath{\rightarrow}}} \coqexternalref{False}{http://coq.inria.fr/distrib/V8.11.0/stdlib//Coq.Init.Logic}{\coqdocinductive{False}}.\coqdoceol
\end{coqdoccode}
\begin{proof}
  By \coqref{NF2.Ext.pos univ}{\coqdoclemma{pos\_univ}} and \coqref{NF2.Ext.weak regularity}{\coqdoclemma{weak\_regularity}}.
\end{proof}

We are now ready to prove extensionality for \NFTWO:

\begin{coqdoccode}
  \coqdocnoindent
\coqdockw{Theorem} \coqdefRef{NF2.Ext.ext}{ext}{\coqdoclemma{ext}}: \coqdockw{\ensuremath{\forall}} \coqdocvar{s} \coqdocvar{s'}, \coqdocvariable{s} \EQX \coqdocvariable{s'} \coqexternalref{::type scope:x '<->' x}{http://coq.inria.fr/distrib/V8.11.0/stdlib//Coq.Init.Logic}{\coqdocnotation{\ensuremath{\leftrightarrow}}} \coqdockw{\ensuremath{\forall}} \coqdocvar{t}, \coqdocvariable{t} \INX \coqdocvariable{s} \coqexternalref{::type scope:x '<->' x}{http://coq.inria.fr/distrib/V8.11.0/stdlib//Coq.Init.Logic}{\coqdocnotation{\ensuremath{\leftrightarrow}}} \coqdocvariable{t} \INX \coqdocvariable{s'}.\coqdoceol
\end{coqdoccode}
\begin{proof}
  The direction $(\Rightarrow)$ of the double implication follows from \coqdocdefinition{IN} being an \coqdocdefinition{EQ}-morphism. As for the other direction, we proceed by cases on \coqdocvariable{s} and \coqdocvariable{s'}:
  \begin{itemize}
    \item If \coqdocvariable{s} and \coqdocvariable{s'} are both low sets or both high sets, then the proof carries on exactly as in \ZF.
    \item The cases when one is a low set and the other one is a high set are ruled out by lemma \coqref{NF2.Ext.pos neg ext neq}{\coqdoclemma{pos\_neg\_ext\_neq}}.
    \qedhere
  \end{itemize}
\end{proof}


\section{\NFO}

Test:

\begin{coqdoccode}
  \coqdocnoindent
  \Definition{} \coqdocdefinition{extP} \{\coqdocvar{X}\} \coqdocvar{P} \coqdocvar{Q} := \Math\forall \Var{x}: \Var{X}, \Var{P} \Var{x} \Math\leftrightarrow{} \Var{Q} \Var{x}.\coqdoceol
  \coqdocemptyline
\end{coqdoccode}
% \coqdocindent{1.00em}


\section{Discussion}
\subsection{Related Work}
\subsection{Future Work}



%%
%% The acknowledgments section is defined using the "acks" environment
%% (and NOT an unnumbered section). This ensures the proper
%% identification of the section in the article metadata, and the
%% consistent spelling of the heading.
% \begin{acks}
% To Robert, for the bagels and explaining CMYK and color spaces.
% \end{acks}

%%
%% The next two lines define the bibliography style to be used, and
%% the bibliography file.
\bibliographystyle{ACM-Reference-Format}
\bibliography{main}

%%
%% If your work has an appendix, this is the place to put it.
\appendix


\end{document}
\endinput
